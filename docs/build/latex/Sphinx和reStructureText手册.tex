%% Generated by Sphinx.
\def\sphinxdocclass{report}
\documentclass[a4paper,10pt,english]{sphinxmanual}
\ifdefined\pdfpxdimen
   \let\sphinxpxdimen\pdfpxdimen\else\newdimen\sphinxpxdimen
\fi \sphinxpxdimen=.75bp\relax
\ifdefined\pdfimageresolution
    \pdfimageresolution= \numexpr \dimexpr1in\relax/\sphinxpxdimen\relax
\fi
%% let collapsible pdf bookmarks panel have high depth per default
\PassOptionsToPackage{bookmarksdepth=5}{hyperref}

\PassOptionsToPackage{booktabs}{sphinx}
\PassOptionsToPackage{colorrows}{sphinx}

\PassOptionsToPackage{warn}{textcomp}

\catcode`^^^^00a0\active\protected\def^^^^00a0{\leavevmode\nobreak\ }
\usepackage{cmap}
\usepackage{xeCJK}
\usepackage{amsmath,amssymb,amstext}
\usepackage{babel}



\setmainfont{FreeSerif}[
  Extension      = .otf,
  UprightFont    = *,
  ItalicFont     = *Italic,
  BoldFont       = *Bold,
  BoldItalicFont = *BoldItalic
]
\setsansfont{FreeSans}[
  Extension      = .otf,
  UprightFont    = *,
  ItalicFont     = *Oblique,
  BoldFont       = *Bold,
  BoldItalicFont = *BoldOblique,
]
\setmonofont{FreeMono}[
  Extension      = .otf,
  UprightFont    = *,
  ItalicFont     = *Oblique,
  BoldFont       = *Bold,
  BoldItalicFont = *BoldOblique,
]



\usepackage[Sonny]{fncychap}
\usepackage{sphinx}

\fvset{fontsize=\small,formatcom=\xeCJKVerbAddon}
\usepackage[left=2cm,right=2cm,top=2cm,bottom=2cm]{geometry}


% Include hyperref last.
\usepackage{hyperref}
% Fix anchor placement for figures with captions.
\usepackage{hypcap}% it must be loaded after hyperref.
% Set up styles of URL: it should be placed after hyperref.
\urlstyle{same}

\addto\captionsenglish{\renewcommand{\contentsname}{全部章节目录}}

\usepackage{sphinxmessages}
\setcounter{tocdepth}{1}

\setcounter{secnumdepth}{2}
\setcounter{tocdepth}{2}

\usepackage{amsmath,amsfonts,amssymb,amsthm}
\usepackage{graphicx}
%%% reduce spaces for Table of contents, figures and tables
%%% it is used "\addtocontents{toc}{\vskip -1.2cm}" etc. in the document
\usepackage[notlot,nottoc,notlof]{}

\usepackage{color}
\usepackage{transparent}
\usepackage{eso-pic}
\usepackage{lipsum}

%%% Needed for displaying Chinese in English documentation
\usepackage{xeCJK}
\usepackage[UTF8]{ctex}

\setCJKmainfont{SourceHanSansSC-Regular}[
    ExternalLocation=../../latex_templates/fonts/,
    Extension       = .otf,
    AutoFakeBold,
    AutoFakeSlant
]

\setCJKsansfont{SourceHanSansSC-Regular}[
    ExternalLocation=../../latex_templates/fonts/,
    Extension       = .otf,
    AutoFakeBold,
    AutoFakeSlant
]

\setCJKmonofont{SourceHanSansSC-Regular}[
    ExternalLocation=../../latex_templates/fonts/,
    Extension       = .otf,
    AutoFakeBold,
    AutoFakeSlant
]

\setmainfont{DejaVuSans}[
    ExternalLocation=../../latex_templates/fonts/,
    Extension       = .ttf,
    AutoFakeBold,
    AutoFakeSlant
]

\setsansfont{DejaVuSans}[
    ExternalLocation=../../latex_templates/fonts/,
    Extension       = .ttf,
    AutoFakeBold,
    AutoFakeSlant
]

\setmonofont{DejaVuSans}[
    ExternalLocation=../../latex_templates/fonts/,
    Extension       = .ttf,
    AutoFakeBold,
    AutoFakeSlant
]

%% spacing between line
\usepackage{setspace}
\singlespacing


\definecolor{myred}{RGB}{229, 32, 26}
\definecolor{myblue}{RGB}{47, 85, 151}
\definecolor{mygray}{RGB}{127, 127, 127}
\definecolor{myblack}{RGB}{64, 64, 64}


%%%%%%%%%%% datetime
\usepackage{datetime}

\newdateformat{MonthYearFormat}{%
    \monthname[\THEMONTH], \THEYEAR}


%% RO, LE will not work for 'oneside' layout.
%% Change oneside to twoside in document class
\usepackage{fancyhdr}
\pagestyle{fancy}
\fancyhf{}

% Header and footer
\makeatletter
\def\@dotsep{2}
\fancypagestyle{normal}{
    \fancyhf{}
    \fancyhead[L]{\nouppercase{\leftmark}}
    \fancyfoot[C]{\py@HeaderFamily\thepage \\ \href{https://www.evas.ai/}}
    \fancyfoot[L]{\@author}
    \fancyfoot[R]{V1.0}
    \renewcommand{\headrulewidth}{0.4pt}
    \renewcommand{\footrulewidth}{0.4pt}
}
\makeatother

\renewcommand{\headrulewidth}{0.5pt}
\renewcommand{\footrulewidth}{0.5pt}

\setlength{\headheight}{27pt}

% Define a spacing for section, subsection and subsubsection
% http://tex.stackexchange.com/questions/108684/spacing-before-and-after-section-titles

\titlespacing*{\section}{0pt}{6pt plus 0pt minus 0pt}{6pt plus 0pt minus 0pt}
\titlespacing*{\subsection}{0pt}{18pt plus 64pt minus 0pt}{0pt}
\titlespacing*{\subsubsection}{0pt}{12pt plus 0pt minus 0pt}{0pt}
\titlespacing*{\paragraph}    {0pt}{3.25ex plus 1ex minus .2ex}{1.5ex plus .2ex}
\titlespacing*{\subparagraph} {0pt}{3.25ex plus 1ex minus .2ex}{1.5ex plus .2ex}

% Define the colors of table of contents
% This is helpful to understand http://tex.stackexchange.com/questions/110253/what-the-first-argument-for-lsubsection-actually-is
\definecolor{LochmaraColor}{HTML}{000000}

% Hyperlinks
\hypersetup{
    colorlinks = true,
    allcolors = {LochmaraColor},
}


\RequirePackage{tocbibind} %%% comment this to remove page number for following
\addto\captionsenglish{\renewcommand{\contentsname}{Table of contents}}
\addto\captionsenglish{\renewcommand{\listfigurename}{List of figures}}
\addto\captionsenglish{\renewcommand{\listtablename}{List of tables}}
% \addto\captionsenglish{\renewcommand{\chaptername}{Chapter}}




%%reduce spacing for itemize
\usepackage{enumitem}
\setlist{nosep}

%%%%%%%%%%% Quote Styles at the top of chapter
\usepackage{epigraph}
\setlength{\epigraphwidth}{0.8\columnwidth}
\newcommand{\chapterquote}[2]{\epigraphhead[60]{\epigraph{\textit{#1}}{\textbf {\textit{--#2}}}}}
%%%%%%%%%%% Quote for all places except Chapter
\newcommand{\sectionquote}[2]{{\quote{\textit{``#1''}}{\textbf {\textit{--#2}}}}}

% Insert 22pt white space before roc title. \titlespacing at line 65 changes it by -22 later on.
\renewcommand*\contentsname{\hspace{0pt}Contents}


% Define section, subsection and subsubsection font size and color
\usepackage{sectsty}
\definecolor{AllportsColor}{HTML}{000000}
\allsectionsfont{\color{AllportsColor}}

\usepackage{titlesec}
\titleformat{\section}
{\color{AllportsColor}\LARGE\bfseries}{\thesection.}{1em}{}

\titleformat{\subsection}
{\color{AllportsColor}\Large\bfseries}{\thesubsection.}{1em}{}

\titleformat{\subsubsection}
{\color{AllportsColor}\large\bfseries}{\thesubsubsection.}{1em}{}

\titleformat{\paragraph}
{\color{AllportsColor}\large\bfseries}{\theparagraph}{1em}{}

\titleformat{\subparagraph}
{\normalfont\normalsize\bfseries}{\thesubparagraph}{1em}{}

\titleformat{\subsubparagraph}
{\normalfont\normalsize\bfseries}{\thesubsubparagraph}{1em}{}


\title{Sphinx和reStructureText手册}
\date{2024 年 02 月 24 日}
\release{V1.0}
\author{奕行智能科技有限公司}
\newcommand{\sphinxlogo}{\sphinxincludegraphics{evas-logo.pdf}\par}
\renewcommand{\releasename}{V1.0}
\makeindex
\begin{document}

\ifdefined\shorthandoff
  \ifnum\catcode`\=\string=\active\shorthandoff{=}\fi
  \ifnum\catcode`\"=\active\shorthandoff{"}\fi
\fi

\pagestyle{empty}
\makeatletter
\newgeometry{left=0cm,right=0cm,bottom=2cm}


\cfoot{www.evas.ai}

\renewcommand{\headrulewidth}{0pt}

{\color{myblue}\rule{30pt}{2.1cm}}
\hspace{1cm}
\begin{minipage}[b]{18cm}
    {\fontsize{28pt}{18pt}\textbf{\color{mygray}\@title}}
\end{minipage}
\hspace{\stretch{1}}

\vspace{48em}


\begin{flushright}
    \setlength\parindent{8em}
    \begin{minipage}[b]{6cm}
        \sphinxlogo
    \end{minipage}
    \hspace{0.2cm}
    \rule{3pt}{1.9cm}
    \hspace{0.2cm}
    \begin{minipage}[b]{6cm}
        {\large{V1.0}}\smallskip\newline
        {\large{\@author}}\smallskip\newline
        {\large{\@date}}\smallskip
    \end{minipage}
    {\color{myblue}\rule{30pt}{1.9cm}}
\end{flushright}


\restoregeometry
\makeatother

\pagestyle{plain}
\sphinxtableofcontents
\pagestyle{normal}
\phantomsection\label{\detokenize{index::doc}}


\sphinxAtStartPar
用于学习sphinx和reStructureText

\sphinxstepscope


\chapter{sphinx介绍}
\label{\detokenize{sphinx_intro:sphinx}}\label{\detokenize{sphinx_intro::doc}}

\section{基本介绍}
\label{\detokenize{sphinx_intro:id1}}
\sphinxAtStartPar
Sphinx 是一种文档工具,它可以令人轻松的撰写出清晰且优美的文档, 由 Georg Brandl 在BSD 许可证下开发。

\sphinxAtStartPar
它最初是为Python文档创建的,它具有出色的工具,可用于各种语言的软件项目文档。
\begin{itemize}
\item {} 
\sphinxAtStartPar
输出格式: HTML(包括Windows HTML帮助),LaTeX(适用于可打印的PDF版本),ePub,Texinfo,手册页,纯文本

\item {} 
\sphinxAtStartPar
广泛的交叉引用: 语义标记和功能,类,引用,词汇表术语和类似信息的自动链接

\item {} 
\sphinxAtStartPar
分层结构: 轻松定义文档树,自动链接到平级,上级和下级

\item {} 
\sphinxAtStartPar
自动索引: 一般索引以及特定于语言的模块索引

\item {} 
\sphinxAtStartPar
代码处理: 使用 \sphinxhref{https://pygments.org/}{Pygments} 荧光笔自动突出显示

\item {} 
\sphinxAtStartPar
扩展: 自动测试代码片段,包含Python模块(API文档)中的文档字符串等

\item {} 
\sphinxAtStartPar
贡献的扩展: 用户在第二个存储库中贡献了50多个扩展;其中大多数可以从PyPI安装

\end{itemize}

\sphinxAtStartPar
Sphinx使用reStructuredText作为其标记语言,其许多优点来自reStructuredText及其解析
和翻译套件 \sphinxhref{https://docutils.sourceforge.io/}{Docutils} 的强大功能和直接性。


\section{安装}
\label{\detokenize{sphinx_intro:id2}}
\sphinxAtStartPar
Sphinx是用 Python 编写的,支持Python 3.5+。


\subsection{Linux}
\label{\detokenize{sphinx_intro:linux}}

\subsubsection{Debian/Ubuntu}
\label{\detokenize{sphinx_intro:debian-ubuntu}}
\sphinxAtStartPar
使用 apt\sphinxhyphen{}get 安装 python3\sphinxhyphen{}sphinx:

\begin{sphinxVerbatim}[commandchars=\\\{\}]
\PYGZdl{} apt\PYGZhy{}get install python3\PYGZhy{}sphinx
\end{sphinxVerbatim}


\subsubsection{RHEL, CentOS}
\label{\detokenize{sphinx_intro:rhel-centos}}
\sphinxAtStartPar
使用 yum 安装 python\sphinxhyphen{}sphinx :

\begin{sphinxVerbatim}[commandchars=\\\{\}]
\PYGZdl{} yum install python\PYGZhy{}sphinx
\end{sphinxVerbatim}


\subsection{PyPI}
\label{\detokenize{sphinx_intro:pypi}}
\begin{sphinxVerbatim}[commandchars=\\\{\}]
\PYG{n}{pip} \PYG{n}{install} \PYG{o}{\PYGZhy{}}\PYG{n}{U} \PYG{n}{sphinx}
\end{sphinxVerbatim}


\subsection{源码安装}
\label{\detokenize{sphinx_intro:id3}}
\begin{sphinxVerbatim}[commandchars=\\\{\}]
git clone https://github.com/sphinx\PYGZhy{}doc/sphinx
\PYGZdl{} cd sphinx
\PYGZdl{} pip install .
\end{sphinxVerbatim}


\section{与其他标记语言互相转换}
\label{\detokenize{sphinx_intro:id4}}\begin{itemize}
\item {} 
\sphinxAtStartPar
Gerard Flanagan编写了一个脚本,将纯HTML转换为reST;它可以在 \sphinxhref{https://pypi.org/project/html2rest/}{Python包索引} 找到。

\item {} 
\sphinxAtStartPar
为了将旧的Python文档转换为Sphinx,编写了一个转换器,可以在 \sphinxhref{https://svn.python.org/projects/doctools/converter/}{Python SVN存储库} 中找到。
它包含将Python\sphinxhyphen{}doc样式的LaTeX标记转换为Sphinx reST的通用代码。

\item {} 
\sphinxAtStartPar
Marcin Wojdyr编写了一个脚本,将Docbook转换为使用Sphinx标记的reST; 它位于 \sphinxhref{https://github.com/wojdyr/db2rst}{GitHub} 。

\item {} 
\sphinxAtStartPar
Christophe de Vienne编写了一个工具,用于将Open/LibreOffice文档转换为Sphinx: \sphinxhref{https://pypi.org/project/odt2sphinx/}{odt2sphinx} 。

\item {} 
\sphinxAtStartPar
要转换不同的标记语言文本,\sphinxhref{https://pandoc.org/}{Pandoc} 是一个非常有用的工具。

\end{itemize}


\section{快速开始}
\label{\detokenize{sphinx_intro:id5}}
\sphinxAtStartPar
为了简化入门过程,Sphinx提供了一个工具 sphinx\sphinxhyphen{}quickstart,
它将生成一个文档源目录并用一些默认值填充它。

\begin{sphinxVerbatim}[commandchars=\\\{\}]
\PYG{c+c1}{\PYGZsh{} 如果网速堪忧,增加 \PYGZhy{}i https://pypi.douban.com/simple}
\PYG{n}{pip} \PYG{n}{install} \PYG{n}{sphinx} \PYG{n}{sphinx}\PYG{o}{\PYGZhy{}}\PYG{n}{autobuild} \PYG{n}{sphinx\PYGZus{}rtd\PYGZus{}theme}
\end{sphinxVerbatim}

\begin{sphinxVerbatim}[commandchars=\\\{\}]
\PYG{c+c1}{\PYGZsh{} 从命令行进入你的初始化目录,如果要托管到github,则为本地仓目录}
\PYG{n}{sphinx}\PYG{o}{\PYGZhy{}}\PYG{n}{quickstart}
\end{sphinxVerbatim}

\begin{sphinxVerbatim}[commandchars=\\\{\}]
\PYG{c+c1}{\PYGZsh{} 初始化过程中的配置项}
\PYG{n}{Separate} \PYG{n}{source} \PYG{o+ow}{and}  \PYG{n}{directories} \PYG{p}{(}\PYG{n}{y}\PYG{o}{/}\PYG{n}{n}\PYG{p}{)} \PYG{p}{[}\PYG{n}{n}\PYG{p}{]}\PYG{p}{:}\PYG{n}{y}  \PYG{c+c1}{\PYGZsh{} 是否分离build和source目录}
\PYG{n}{Project} \PYG{n}{name}\PYG{p}{:} \PYG{n}{sphinx\PYGZus{}handbook}  \PYG{c+c1}{\PYGZsh{} 项目名}
\PYG{n}{Author} \PYG{n}{name}\PYG{p}{(}\PYG{n}{s}\PYG{p}{)}\PYG{p}{:} \PYG{n}{firewang}  \PYG{c+c1}{\PYGZsh{} 作者}
\PYG{n}{Project} \PYG{n}{version} \PYG{p}{[}\PYG{p}{]}\PYG{p}{:}  \PYG{c+c1}{\PYGZsh{} 默认没有版本,按Enter跳过}
\PYG{n}{Project} \PYG{n}{release} \PYG{p}{[}\PYG{p}{]}\PYG{p}{:}  \PYG{c+c1}{\PYGZsh{} 默认没有发布版本,按Enter跳过}
\PYG{n}{Project} \PYG{n}{language} \PYG{p}{[}\PYG{n}{en}\PYG{p}{]}\PYG{p}{:} \PYG{n}{zh\PYGZus{}CN}  \PYG{c+c1}{\PYGZsh{} 默认英文,指定为中文}
\end{sphinxVerbatim}

\begin{sphinxVerbatim}[commandchars=\\\{\}]
\PYG{c+c1}{\PYGZsh{} Sphinx中的插件配置}
\PYG{o}{\PYGZgt{}} \PYG{n}{autodoc}\PYG{p}{:} \PYG{n}{automatically} \PYG{n}{insert} \PYG{n}{docstrings} \PYG{k+kn}{from} \PYG{n+nn}{modules} \PYG{p}{(}\PYG{n}{y}\PYG{o}{/}\PYG{n}{n}\PYG{p}{)} \PYG{p}{[}\PYG{n}{n}\PYG{p}{]}\PYG{p}{:}
\PYG{o}{\PYGZgt{}} \PYG{n}{doctest}\PYG{p}{:} \PYG{n}{automatically} \PYG{n}{test} \PYG{n}{code} \PYG{n}{snippets} \PYG{o+ow}{in} \PYG{n}{doctest} \PYG{n}{blocks} \PYG{p}{(}\PYG{n}{y}\PYG{o}{/}\PYG{n}{n}\PYG{p}{)} \PYG{p}{[}\PYG{n}{n}\PYG{p}{]}\PYG{p}{:}
\PYG{o}{\PYGZgt{}} \PYG{n}{intersphinx}\PYG{p}{:} \PYG{n}{link} \PYG{n}{between} \PYG{n}{Sphinx} \PYG{n}{documentation} \PYG{n}{of} \PYG{n}{different} \PYG{n}{projects} \PYG{p}{(}\PYG{n}{y}\PYG{o}{/}\PYG{n}{n}\PYG{p}{)} \PYG{p}{[}\PYG{n}{n}\PYG{p}{]}\PYG{p}{:}
\PYG{o}{\PYGZgt{}} \PYG{n}{todo}\PYG{p}{:} \PYG{n}{write} \PYG{l+s+s2}{\PYGZdq{}}\PYG{l+s+s2}{todo}\PYG{l+s+s2}{\PYGZdq{}} \PYG{n}{entries} \PYG{n}{that} \PYG{n}{can} \PYG{n}{be} \PYG{n}{shown} \PYG{o+ow}{or} \PYG{n}{hidden} \PYG{n}{on} \PYG{n}{build} \PYG{p}{(}\PYG{n}{y}\PYG{o}{/}\PYG{n}{n}\PYG{p}{)} \PYG{p}{[}\PYG{n}{n}\PYG{p}{]}\PYG{p}{:}
\PYG{o}{\PYGZgt{}} \PYG{n}{coverage}\PYG{p}{:} \PYG{n}{checks} \PYG{k}{for} \PYG{n}{documentation} \PYG{n}{coverage} \PYG{p}{(}\PYG{n}{y}\PYG{o}{/}\PYG{n}{n}\PYG{p}{)} \PYG{p}{[}\PYG{n}{n}\PYG{p}{]}\PYG{p}{:}
\PYG{o}{\PYGZgt{}} \PYG{n}{imgmath}\PYG{p}{:} \PYG{n}{include} \PYG{n}{math}\PYG{p}{,} \PYG{n}{rendered} \PYG{k}{as} \PYG{n}{PNG} \PYG{o+ow}{or} \PYG{n}{SVG} \PYG{n}{images} \PYG{p}{(}\PYG{n}{y}\PYG{o}{/}\PYG{n}{n}\PYG{p}{)} \PYG{p}{[}\PYG{n}{n}\PYG{p}{]}\PYG{p}{:}
\PYG{o}{\PYGZgt{}} \PYG{n}{mathjax}\PYG{p}{:} \PYG{n}{include} \PYG{n}{math}\PYG{p}{,} \PYG{n}{rendered} \PYG{o+ow}{in} \PYG{n}{the} \PYG{n}{browser} \PYG{n}{by} \PYG{n}{MathJax} \PYG{p}{(}\PYG{n}{y}\PYG{o}{/}\PYG{n}{n}\PYG{p}{)} \PYG{p}{[}\PYG{n}{n}\PYG{p}{]}\PYG{p}{:}
\PYG{o}{\PYGZgt{}} \PYG{n}{ifconfig}\PYG{p}{:} \PYG{n}{conditional} \PYG{n}{inclusion} \PYG{n}{of} \PYG{n}{content} \PYG{n}{based} \PYG{n}{on} \PYG{n}{config} \PYG{n}{values} \PYG{p}{(}\PYG{n}{y}\PYG{o}{/}\PYG{n}{n}\PYG{p}{)} \PYG{p}{[}\PYG{n}{n}\PYG{p}{]}\PYG{p}{:}
\PYG{o}{\PYGZgt{}} \PYG{n}{viewcode}\PYG{p}{:} \PYG{n}{include} \PYG{n}{links} \PYG{n}{to} \PYG{n}{the} \PYG{n}{source} \PYG{n}{code} \PYG{n}{of} \PYG{n}{documented} \PYG{n}{Python} \PYG{n}{objects} \PYG{p}{(}\PYG{n}{y}\PYG{o}{/}\PYG{n}{n}\PYG{p}{)} \PYG{p}{[}\PYG{n}{n}\PYG{p}{]}\PYG{p}{:}
\PYG{o}{\PYGZgt{}} \PYG{n}{githubpages}\PYG{p}{:} \PYG{n}{create} \PYG{o}{.}\PYG{n}{nojekyll} \PYG{n}{file} \PYG{n}{to} \PYG{n}{publish} \PYG{n}{the} \PYG{n}{document} \PYG{n}{on} \PYG{n}{GitHub} \PYG{n}{pages} \PYG{p}{(}\PYG{n}{y}\PYG{o}{/}\PYG{n}{n}\PYG{p}{)} \PYG{p}{[}\PYG{n}{n}\PYG{p}{]}\PYG{p}{:}
\PYG{c+c1}{\PYGZsh{}autodoc:从模块中自动插入代码}
\PYG{c+c1}{\PYGZsh{}doctest:在文档中自动测试代码段}
\PYG{c+c1}{\PYGZsh{}intersphinx:在不同的项目文档中的链接}
\PYG{c+c1}{\PYGZsh{}todo:写入“todo”条目,可以在构建文档中显示或隐藏}
\PYG{c+c1}{\PYGZsh{}coverage:检查文档覆盖率}
\PYG{c+c1}{\PYGZsh{}imgmath:提供PNG或SVG图像(包含数学)}
\PYG{c+c1}{\PYGZsh{}mathjax:提供浏览器插件MathJax(包含数学)}
\PYG{c+c1}{\PYGZsh{}ifconfig:根据配置值的内容纳入条件}
\PYG{c+c1}{\PYGZsh{}viewcode:包含指向Python对象源代码的链接}
\PYG{c+c1}{\PYGZsh{}githubpages:创建.nojekyll文件以在GitHub页面上发布文档}
\end{sphinxVerbatim}

\begin{sphinxVerbatim}[commandchars=\\\{\}]
\PYGZsh{} 初始化完成以后,在目录下就会生成以下内容
├── build
├── make.bat
├── Makefile
└── source
    ├── conf.py
    ├── index.rst
    ├── \PYGZus{}static
    └── \PYGZus{}templates

\PYGZsh{} build 为编译后生成的文档
\PYGZsh{} source 为文档目录,其中index.rst 为索引目录,conf.py 是配置文件
\end{sphinxVerbatim}

\sphinxAtStartPar
之后新增、修改文件后更新编译文档,有两种方式
\begin{itemize}
\item {} 
\sphinxAtStartPar
\sphinxstylestrong{使用 sphinx\sphinxhyphen{}build 程序启动构建}

\end{itemize}

\begin{sphinxVerbatim}[commandchars=\\\{\}]
\PYGZdl{} sphinx\PYGZhy{}build \PYGZhy{}b html sourcedir builddir
\end{sphinxVerbatim}

\sphinxAtStartPar
其中 sourcedir 是 source directory ,builddir 是您要在其中放置构建文档的目录。 \sphinxhyphen{}b 选项选择一个构建器。
\begin{itemize}
\item {} 
\sphinxAtStartPar
\sphinxstylestrong{通过make}

\end{itemize}

\sphinxAtStartPar
sphinx\sphinxhyphen{}quickstart 脚本创建了一个 Makefile 和一个 make.bat,它让你的生活更加轻松。

\begin{sphinxVerbatim}[commandchars=\\\{\}]
\PYG{n}{make} \PYG{n}{html}
\PYG{c+c1}{\PYGZsh{} make pdf}
\PYG{c+c1}{\PYGZsh{} make epub}
\end{sphinxVerbatim}

\sphinxstepscope


\chapter{sphinx配置}
\label{\detokenize{sphinx_conf:sphinx}}\label{\detokenize{sphinx_conf::doc}}
\sphinxAtStartPar
配置文件 /source/conf.py 在构建时作为Python代码执行
(使用 execfile() , 并且当前目录设置为其包含目录)。


\section{项目信息}
\label{\detokenize{sphinx_conf:id1}}

\begin{savenotes}\sphinxattablestart
\sphinxthistablewithglobalstyle
\centering
\begin{tabulary}{\linewidth}[t]{TT}
\sphinxtoprule
\sphinxtableatstartofbodyhook
\sphinxAtStartPar
project
&
\sphinxAtStartPar
记录的项目名称。
\\
\sphinxhline
\sphinxAtStartPar
author
&
\sphinxAtStartPar
该文件的作者姓名。 默认值为 unknown
\\
\sphinxhline
\sphinxAtStartPar
copyright
&
\sphinxAtStartPar
2020, Author Name 风格的版权声明
\\
\sphinxhline
\sphinxAtStartPar
version
&
\sphinxAtStartPar
主要项目版本, 用作 \sphinxtitleref{|version|} 的替代品。 例如, 对于Python文档, 这可能类似于 2.6 。
\\
\sphinxhline
\sphinxAtStartPar
release
&
\sphinxAtStartPar
完整的项目版本, 用作 \sphinxtitleref{|release|} 的替代品, 例如在HTML模板中。 例如, 对于Python文档, 这可能类似于 2.6.0rc1 。

\sphinxAtStartPar
如果你不需要在 version 和 release 之间提供分隔, 只需将它们设置为相同的值即可。
\\
\sphinxbottomrule
\end{tabulary}
\sphinxtableafterendhook\par
\sphinxattableend\end{savenotes}


\section{一般配置项}
\label{\detokenize{sphinx_conf:id2}}

\subsection{extensions}
\label{\detokenize{sphinx_conf:extensions}}
\sphinxAtStartPar
可以是Sphinx(名为 sphinx.ext.*)或自定义的扩展。

\sphinxAtStartPar
请注意, 如果扩展名位于另一个目录中, 则可以在conf文件中扩展 sys.path(\sphinxstylestrong{使用绝对路径})。

\sphinxAtStartPar
如果扩展路径是相对于 configuration directory , 使用 os.path.abspath()

\begin{sphinxVerbatim}[commandchars=\\\{\}]
\PYG{k+kn}{import} \PYG{n+nn}{sys}\PYG{o}{,} \PYG{n+nn}{os}
\PYG{n}{sys}\PYG{o}{.}\PYG{n}{path}\PYG{o}{.}\PYG{n}{append}\PYG{p}{(}\PYG{n}{os}\PYG{o}{.}\PYG{n}{path}\PYG{o}{.}\PYG{n}{abspath}\PYG{p}{(}\PYG{l+s+s1}{\PYGZsq{}}\PYG{l+s+s1}{sphinxext}\PYG{l+s+s1}{\PYGZsq{}}\PYG{p}{)}\PYG{p}{)}
\PYG{n}{extensions} \PYG{o}{=} \PYG{p}{[}\PYG{l+s+s1}{\PYGZsq{}}\PYG{l+s+s1}{extname}\PYG{l+s+s1}{\PYGZsq{}}\PYG{p}{]}
\end{sphinxVerbatim}

\sphinxAtStartPar
这样, 你可以从子目录 sphinxext 加载一个名为 extname 的扩展名。


\subsection{source\_suffix}
\label{\detokenize{sphinx_conf:source-suffix}}
\sphinxAtStartPar
源文件的文件扩展名。 Sphinx将具有此后缀的文件视为源。

\begin{sphinxVerbatim}[commandchars=\\\{\}]
\PYG{n}{source\PYGZus{}suffix} \PYG{o}{=} \PYG{p}{\PYGZob{}}
    \PYG{l+s+s1}{\PYGZsq{}}\PYG{l+s+s1}{.rst}\PYG{l+s+s1}{\PYGZsq{}}\PYG{p}{:} \PYG{l+s+s1}{\PYGZsq{}}\PYG{l+s+s1}{restructuredtext}\PYG{l+s+s1}{\PYGZsq{}}\PYG{p}{,}
    \PYG{l+s+s1}{\PYGZsq{}}\PYG{l+s+s1}{.txt}\PYG{l+s+s1}{\PYGZsq{}}\PYG{p}{:} \PYG{l+s+s1}{\PYGZsq{}}\PYG{l+s+s1}{restructuredtext}\PYG{l+s+s1}{\PYGZsq{}}\PYG{p}{,}
    \PYG{l+s+s1}{\PYGZsq{}}\PYG{l+s+s1}{.md}\PYG{l+s+s1}{\PYGZsq{}}\PYG{p}{:} \PYG{l+s+s1}{\PYGZsq{}}\PYG{l+s+s1}{markdown}\PYG{l+s+s1}{\PYGZsq{}}\PYG{p}{,}
\PYG{p}{\PYGZcb{}}
\end{sphinxVerbatim}

\sphinxAtStartPar
可以使用源解析器扩展添加新文件类型。


\subsection{source\_encoding}
\label{\detokenize{sphinx_conf:source-encoding}}
\sphinxAtStartPar
所有reST源文件的编码。推荐的编码和默认值是 ‘utf\sphinxhyphen{}8\sphinxhyphen{}sig’ 。


\subsection{source\_parsers}
\label{\detokenize{sphinx_conf:source-parsers}}
\sphinxAtStartPar
如果给出, 则不同源的解析器类字典就足够了。键是后缀, 值可以是类或字符串, 给出解析器类的完全限定名称。
解析器类可以是 docutils.parsers.Parser 或 sphinx.parsers.Parser 。
不在字典中的后缀的文件将使用默认的reStructuredText解析器进行解析。

\begin{sphinxVerbatim}[commandchars=\\\{\}]
\PYG{n}{source\PYGZus{}parsers} \PYG{o}{=} \PYG{p}{\PYGZob{}}\PYG{l+s+s1}{\PYGZsq{}}\PYG{l+s+s1}{.md}\PYG{l+s+s1}{\PYGZsq{}}\PYG{p}{:} \PYG{l+s+s1}{\PYGZsq{}}\PYG{l+s+s1}{recommonmark.parser.CommonMarkParser}\PYG{l+s+s1}{\PYGZsq{}}\PYG{p}{\PYGZcb{}}
\end{sphinxVerbatim}


\subsection{master\_doc}
\label{\detokenize{sphinx_conf:master-doc}}
\sphinxAtStartPar
主目录文件名,默认为index


\subsection{exclude\_patterns}
\label{\detokenize{sphinx_conf:exclude-patterns}}
\sphinxAtStartPar
查找源文件时应排除的glob样式模式列表。 它们与源目录相对于源目录进行匹配, 在所有平台上使用斜杠作为目录分隔符。

\sphinxAtStartPar
示例模式:

\begin{sphinxVerbatim}[commandchars=\\\{\}]
\PYGZsq{}library/xml.rst\PYGZsq{} – 忽略 library/xml.rst 文件(替换条目 unused\PYGZus{}docs)
\PYGZsq{}library/xml\PYGZsq{} – 忽略 library/xml 目录
\PYGZsq{}library/xml*\PYGZsq{} – 忽略以 library/xml 开头的所有文件和目录
\PYGZsq{}**/.svn\PYGZsq{} – 忽略所有 .svn 目录
\end{sphinxVerbatim}


\subsection{templates\_path}
\label{\detokenize{sphinx_conf:templates-path}}
\sphinxAtStartPar
包含额外模板的路径列表(覆盖内置/主题特定模板的模板)。相对路径被视为相对于配置目录。

\sphinxAtStartPar
由于这些文件不是要构建的, 因此它们会自动添加到 exclude\_patterns 中.


\subsection{rst\_epilog}
\label{\detokenize{sphinx_conf:rst-epilog}}
\sphinxAtStartPar
一串reStructuredText, 它将包含在每个读取的源文件的末尾。

\begin{sphinxVerbatim}[commandchars=\\\{\}]
\PYG{n}{rst\PYGZus{}epilog} \PYG{o}{=} \PYG{l+s+s2}{\PYGZdq{}\PYGZdq{}\PYGZdq{}}
\PYG{l+s+s2}{.. |psf| replace:: Python Software Foundation}
\PYG{l+s+s2}{\PYGZdq{}\PYGZdq{}\PYGZdq{}}
\end{sphinxVerbatim}


\subsection{rst\_prolog}
\label{\detokenize{sphinx_conf:rst-prolog}}
\sphinxAtStartPar
一串reStructuredText, 它将包含在每个读取的源文件的开头。

\begin{sphinxVerbatim}[commandchars=\\\{\}]
\PYG{n}{rst\PYGZus{}prolog} \PYG{o}{=} \PYG{l+s+s2}{\PYGZdq{}\PYGZdq{}\PYGZdq{}}
\PYG{l+s+s2}{.. |psf| replace:: Python Software Foundation}
\PYG{l+s+s2}{\PYGZdq{}\PYGZdq{}\PYGZdq{}}
\end{sphinxVerbatim}


\subsection{primary\_domain}
\label{\detokenize{sphinx_conf:primary-domain}}

\subsection{default\_role}
\label{\detokenize{sphinx_conf:default-role}}

\subsection{keep\_warnings}
\label{\detokenize{sphinx_conf:keep-warnings}}

\subsection{suppress\_warnings}
\label{\detokenize{sphinx_conf:suppress-warnings}}
\sphinxAtStartPar
用于禁止任意警告消息的警告类型列表。

\sphinxAtStartPar
Sphinx支持以下警告类型:
\begin{itemize}
\item {} 
\sphinxAtStartPar
app.add\_node

\item {} 
\sphinxAtStartPar
app.add\_directive

\item {} 
\sphinxAtStartPar
app.add\_role

\item {} 
\sphinxAtStartPar
app.add\_generic\_role

\item {} 
\sphinxAtStartPar
app.add\_source\_parser

\item {} 
\sphinxAtStartPar
download.not\_readable

\item {} 
\sphinxAtStartPar
image.not\_readable

\item {} 
\sphinxAtStartPar
ref.term

\item {} 
\sphinxAtStartPar
ref.ref

\item {} 
\sphinxAtStartPar
ref.numref

\item {} 
\sphinxAtStartPar
ref.keyword

\item {} 
\sphinxAtStartPar
ref.option

\item {} 
\sphinxAtStartPar
ref.citation

\item {} 
\sphinxAtStartPar
ref.footnote

\item {} 
\sphinxAtStartPar
ref.doc

\item {} 
\sphinxAtStartPar
ref.python

\item {} 
\sphinxAtStartPar
misc.highlighting\_failure

\item {} 
\sphinxAtStartPar
toc.secnum

\item {} 
\sphinxAtStartPar
epub.unknown\_project\_files

\end{itemize}


\subsection{needs\_sphinx}
\label{\detokenize{sphinx_conf:needs-sphinx}}
\sphinxAtStartPar
指定用来构建的sphinx版本,如果设置为 major.minor 版本字符串, 如 ‘1.1’ ,
Sphinx会将其与版本进行比较, 如果它太旧则拒绝构建。 默认是没有要求的。


\subsection{needs\_extensions}
\label{\detokenize{sphinx_conf:needs-extensions}}
\sphinxAtStartPar
指定扩展的版本 , 例如: needs\_extensions = \{‘sphinxcontrib.something’:’1.5’\} 。
版本字符串应采用 major.minor 形式。不必为所有扩展指定要求, 仅适用于您要检查的扩展。


\subsection{manpages\_url}
\label{\detokenize{sphinx_conf:manpages-url}}
\sphinxAtStartPar
交叉引用的URL manpage 指令。
如果将其定义为 \sphinxurl{https://manpages.debian.org}/\{path\} , 则 \sphinxstyleliteralemphasis{\sphinxupquote{man(1)}} 角色将链接到 <\sphinxurl{https://manpages.debian.org/man(1)}> 。

\sphinxAtStartPar
可用的模式是:
\begin{itemize}
\item {} 
\sphinxAtStartPar
page \sphinxhyphen{} 手册页 (man)

\item {} 
\sphinxAtStartPar
section \sphinxhyphen{} 手册部分 (1)

\item {} 
\sphinxAtStartPar
path \sphinxhyphen{} 原始的手册页和指定的部分(man(1))

\end{itemize}


\subsection{numfig}
\label{\detokenize{sphinx_conf:numfig}}
\sphinxAtStartPar
如果为true, 则数字, 表格和代码块如果有标题则会自动编号。 numref 角色已启用。


\subsection{numfig\_format}
\label{\detokenize{sphinx_conf:numfig-format}}
\sphinxAtStartPar
一个字典将 ‘figure’ , ‘table’ , ‘code\sphinxhyphen{}block’ 和 ‘section’ 映射到用于图号格式的字符串。作为一个特殊字符, %s 将被替换为图号。
默认是使用 ‘Fig. \%s’ 为 ‘figure’ , ‘Table %s’ 为 ‘table’ , ‘Listing %s’ 为 ‘code\sphinxhyphen{}block’ 和 ‘Section’ 为 ‘section’


\subsection{numfig\_secnum\_depth}
\label{\detokenize{sphinx_conf:numfig-secnum-depth}}\begin{itemize}
\item {} 
\sphinxAtStartPar
如果设置为 0 , 则数字, 表格和代码块从 1 开始连续编号。

\item {} 
\sphinxAtStartPar
如果 1 (默认)数字将是 x.1 , x.2 , … 与 x 的节号(顶级切片;没有 x 如果没有部分)。只有当通过 toctree 指令的 :numbered: 选项激活了段号时, 这才自然适用。

\item {} 
\sphinxAtStartPar
2 表示数字将是 xy1 , xy2 , …如果位于子区域(但仍然是 x.1 , x.2 , … 如果直接位于一个部分和 1 , 2 , … 如果不在任何顶级部分。)

\end{itemize}


\subsection{smartquotes}
\label{\detokenize{sphinx_conf:smartquotes}}

\subsection{smartquotes\_action}
\label{\detokenize{sphinx_conf:smartquotes-action}}

\subsection{smartquotes\_excludes}
\label{\detokenize{sphinx_conf:smartquotes-excludes}}

\subsection{tls\_verify}
\label{\detokenize{sphinx_conf:tls-verify}}
\sphinxAtStartPar
如果为true, Sphinx将验证服务器认证。默认为 True 。


\subsection{tls\_cacerts}
\label{\detokenize{sphinx_conf:tls-cacerts}}
\sphinxAtStartPar
CA的证书文件的路径或包含证书的目录的路径。这也允许字典映射主机名到证书文件的路径。证书用于验证服务器认证。


\subsection{highlight\_language}
\label{\detokenize{sphinx_conf:highlight-language}}
\sphinxAtStartPar
用于突出显示源代码的默认语言。默认语言为 ‘python3’ 。
\sphinxhref{https://www.sphinx.org.cn/usage/restructuredtext/directives.html\#code-examples}{代码高亮}


\subsection{highlight\_options}
\label{\detokenize{sphinx_conf:highlight-options}}
\sphinxAtStartPar
\sphinxhref{http://pygments.org/docs/lexers/}{Pygments documentation}


\subsection{pygments\_style}
\label{\detokenize{sphinx_conf:pygments-style}}
\sphinxAtStartPar
用于Pygments突出显示源代码的样式名称。如果未设置, 则为HTML输出选择主题的默认样式或 ‘sphinx’ 。


\subsection{add\_function\_parentheses}
\label{\detokenize{sphinx_conf:add-function-parentheses}}
\sphinxAtStartPar
一个布尔值, 决定是否将括号附加到函数和方法角色文本(例如 \sphinxcode{\sphinxupquote{input()}} 的内容)以表示该名称是可调用的。默认为 True 。


\subsection{add\_module\_names}
\label{\detokenize{sphinx_conf:add-module-names}}
\sphinxAtStartPar
A boolean that decides whether module names are prepended to all object names
(for object types where a “module” of some kind is defined), e.g. for py:function directives. Default is True.


\subsection{show\_authors}
\label{\detokenize{sphinx_conf:show-authors}}
\sphinxAtStartPar
一个布尔值, 决定 codeauthor 和 sectionauthor 指令在构建的文件中产生任何输出。


\subsection{modindex\_common\_prefix}
\label{\detokenize{sphinx_conf:modindex-common-prefix}}
\sphinxAtStartPar
为了对Python模块索引进行排序而忽略的前缀列表(例如, 如果将其设置为 {[}‘foo.’{]} , 那么 foo.bar 将显示在 B 下, 而不是 F )。
如果您记录包含单个包的项目, 这可能很方便。仅适用于当前的HTML构建器。默认是 {[}{]} 。


\subsection{trim\_footnote\_reference\_space}
\label{\detokenize{sphinx_conf:trim-footnote-reference-space}}
\sphinxAtStartPar
在脚注引用之前修剪空格, 这是reST解析器识别脚注所必需的, 但在输出中看起来不太好。


\subsection{trim\_doctest\_flags}
\label{\detokenize{sphinx_conf:trim-doctest-flags}}
\sphinxAtStartPar
如果为true, 则删除doctest标志(在行的末尾看起来像 doctest:FLAG, … 的注释)和 <BLANKLINE> 标记,
以显示交互式Python会话的所有代码块(即doctests) 。
默认为 True 。有关包含doctests的更多可能性, 请参阅扩展名 doctest 。


\section{国际化选项}
\label{\detokenize{sphinx_conf:id4}}
\sphinxAtStartPar
影响Sphinx的 母语支持

\sphinxAtStartPar
language
文档编写语言的代码。Sphinx自动生成的任何文本都将使用该语言。

\sphinxAtStartPar
目前Sphinx支持的语言是:
\begin{itemize}
\item {} 
\sphinxAtStartPar
bn – 孟加拉语

\item {} 
\sphinxAtStartPar
ca – 加泰罗尼亚语

\item {} 
\sphinxAtStartPar
cs – 捷克语

\item {} 
\sphinxAtStartPar
da – 丹麦语

\item {} 
\sphinxAtStartPar
de – 德语

\item {} 
\sphinxAtStartPar
en – 英语

\item {} 
\sphinxAtStartPar
es – 西班牙语

\item {} 
\sphinxAtStartPar
et – 爱沙尼亚语

\item {} 
\sphinxAtStartPar
eu – 巴斯克语

\item {} 
\sphinxAtStartPar
fa – 伊朗语

\item {} 
\sphinxAtStartPar
fi – 芬兰语

\item {} 
\sphinxAtStartPar
fr – 法语

\item {} 
\sphinxAtStartPar
he – 希伯来语

\item {} 
\sphinxAtStartPar
hr – 克罗地亚语

\item {} 
\sphinxAtStartPar
hu – 匈牙利语

\item {} 
\sphinxAtStartPar
id – 印度尼西亚语

\item {} 
\sphinxAtStartPar
it – 意大利语

\item {} 
\sphinxAtStartPar
ja – 日语

\item {} 
\sphinxAtStartPar
ko – 朝鲜语

\item {} 
\sphinxAtStartPar
lt – 立陶宛语

\item {} 
\sphinxAtStartPar
lv – 拉脱维亚语

\item {} 
\sphinxAtStartPar
mk – 马其顿语

\item {} 
\sphinxAtStartPar
nb\_NO – 挪威博克马尔语

\item {} 
\sphinxAtStartPar
ne – 尼泊尔语

\item {} 
\sphinxAtStartPar
nl – 荷兰语

\item {} 
\sphinxAtStartPar
pl – 波兰语

\item {} 
\sphinxAtStartPar
pt\_BR – 巴西葡萄牙语

\item {} 
\sphinxAtStartPar
pt\_PT – 欧洲葡萄牙语

\item {} 
\sphinxAtStartPar
ru – 俄语

\item {} 
\sphinxAtStartPar
si – 僧伽罗语

\item {} 
\sphinxAtStartPar
sk – 斯洛伐克语

\item {} 
\sphinxAtStartPar
sl – 斯洛文尼亚语

\item {} 
\sphinxAtStartPar
sv – 瑞典语

\item {} 
\sphinxAtStartPar
tr – 土耳其语

\item {} 
\sphinxAtStartPar
uk\_UA – 乌克兰语

\item {} 
\sphinxAtStartPar
vi – 越南语

\item {} 
\sphinxAtStartPar
zh\_CN – 简体中文

\item {} 
\sphinxAtStartPar
zh\_TW – 繁体中文

\end{itemize}


\subsection{locale\_dirs}
\label{\detokenize{sphinx_conf:locale-dirs}}
\sphinxAtStartPar
相对于源目录, 在其中搜索其他消息目录(请参阅 language )的目录。此路径上的目录由标准 gettext 模块搜索。

\sphinxAtStartPar
内部消息是从 sphinx 的文本域中获取的;因此, 如果将目录 。/locale 添加到此设置, 则消息目录(使用 msgfmt 编译为 .po 格式)必须位于 ./locale/language/LC\_MESSAGES/sphinx.mo 。
单个文档的文本域取决于 gettext\_compact 。

\sphinxAtStartPar
默认是 {[}‘locales’{]}.


\subsection{gettext\_compact}
\label{\detokenize{sphinx_conf:gettext-compact}}
\sphinxAtStartPar
如果为true, 则文档的文本域是其docname, 如果它是顶级项目文件, 则为其基本目录。

\sphinxAtStartPar
默认情况下, 文档 markup/code.rst 最终出现在 markup 文本域中。将此选项设置为 False , 它是 标记/代码 。


\subsection{gettext\_uuid}
\label{\detokenize{sphinx_conf:gettext-uuid}}
\sphinxAtStartPar
如果为true, 则Sphinx会在消息目录中生成用于版本跟踪的uuid信息。它用于:
\begin{itemize}
\item {} 
\sphinxAtStartPar
在.pot文件中为每个msgids添加uid行。

\item {} 
\sphinxAtStartPar
计算新msgids和以前保存的旧msgids之间的相似性。(计算时间长)

\end{itemize}

\sphinxAtStartPar
如果想加速计算, 可以使用 pip install python\sphinxhyphen{}levenshtein 来使用C编写的 python\sphinxhyphen{}levenshtein 第三方包。

\sphinxAtStartPar
默认是 False.


\subsection{gettext\_location}
\label{\detokenize{sphinx_conf:gettext-location}}
\sphinxAtStartPar
如果为true, 则Sphinx为消息目录中的消息生成位置信息。默认 True.


\subsection{gettext\_auto\_build}
\label{\detokenize{sphinx_conf:gettext-auto-build}}
\sphinxAtStartPar
如果为true, 则Sphinx为每个翻译目录文件构建mo文件。

\sphinxAtStartPar
默认是 True.


\subsection{gettext\_additional\_targets}
\label{\detokenize{sphinx_conf:gettext-additional-targets}}
\sphinxAtStartPar
指定名称以启用gettext提取和转换。可以指定以下名称:


\begin{savenotes}\sphinxattablestart
\sphinxthistablewithglobalstyle
\centering
\begin{tabulary}{\linewidth}[t]{TT}
\sphinxtoprule
\sphinxtableatstartofbodyhook
\sphinxAtStartPar
索引:
&
\sphinxAtStartPar
索引条款
\\
\sphinxhline
\sphinxAtStartPar
Literal\sphinxhyphen{}block:
&
\sphinxAtStartPar
文字块和code\sphinxhyphen{}block
\\
\sphinxhline
\sphinxAtStartPar
Doctest\sphinxhyphen{}block:
&
\sphinxAtStartPar
doctest块
\\
\sphinxhline
\sphinxAtStartPar
Raw:
&
\sphinxAtStartPar
原始内容
\\
\sphinxhline
\sphinxAtStartPar
图片:
&
\sphinxAtStartPar
image/figure uri 和 alt
\\
\sphinxbottomrule
\end{tabulary}
\sphinxtableafterendhook\par
\sphinxattableend\end{savenotes}

\sphinxAtStartPar
例如: gettext\_additional\_targets = {[}‘literal\sphinxhyphen{}block’, ‘image’{]}
默认是 {[}{]}


\subsection{figure\_language\_filename}
\label{\detokenize{sphinx_conf:figure-language-filename}}
\sphinxAtStartPar
语言特定数字的文件名格式。默认值为 \sphinxtitleref{\{root\}.\{language\}\{ext\}}
它将从 \sphinxtitleref{.. image:: dirname/filename.png} 扩展为 \sphinxtitleref{dirname/filename.en.png} 可用的格式标记是:
\begin{itemize}
\item {} 
\sphinxAtStartPar
\{root\} \sphinxhyphen{} 文件名, 包括任何路径组件, 没有文件扩展名, 例如 dirname/filename

\item {} 
\sphinxAtStartPar
\{path\} \sphinxhyphen{} 文件名的目录路径组件, 如果非空, 则带有斜杠, 例如 dirname/

\item {} 
\sphinxAtStartPar
\{basename\} \sphinxhyphen{} 没有目录路径或文件扩展名组件的文件名, 例如 filename

\item {} 
\sphinxAtStartPar
\{ext\} \sphinxhyphen{} 文件扩展名, 例如 .png

\item {} 
\sphinxAtStartPar
\{language\} \sphinxhyphen{} 翻译语言, 例如 en

\end{itemize}

\sphinxAtStartPar
例如, 将其设置为 \sphinxtitleref{\{path\}\{language\}/\{basename\}\{ext\}} 将扩展为 \sphinxtitleref{dirname/en/filename.png}


\section{数学选项}
\label{\detokenize{sphinx_conf:id5}}

\begin{savenotes}\sphinxattablestart
\sphinxthistablewithglobalstyle
\centering
\begin{tabulary}{\linewidth}[t]{TT}
\sphinxtoprule
\sphinxtableatstartofbodyhook
\sphinxAtStartPar
math\_number\_all
&
\sphinxAtStartPar
如果要对所有显示的数学项进行编号, 请将此选项设置为 True 。默认为 False 。
\\
\sphinxhline
\sphinxAtStartPar
math\_eqref\_format
&
\sphinxAtStartPar
用于格式化方程式引用标签的字符串。 \{number\} 占位符代表等式编号。
例: ‘Eq.\{number\}’ 被渲染为, 例如, Eq.10.
\\
\sphinxhline
\sphinxAtStartPar
math\_numfig
&
\sphinxAtStartPar
如果为 True , 则在页面中下显示的数学公式编号。默认为 True 。
\\
\sphinxbottomrule
\end{tabulary}
\sphinxtableafterendhook\par
\sphinxattableend\end{savenotes}


\section{HTML输出选项}
\label{\detokenize{sphinx_conf:html}}
\sphinxAtStartPar
这些选项会影响HTML以及HTML帮助输出, 以及使用Sphinx的HTMLWriter类的其他构建器。


\subsection{html主题html\_theme}
\label{\detokenize{sphinx_conf:htmlhtml-theme}}
\sphinxAtStartPar
页面主题模板,默认 alabaster 。 \sphinxhref{https://www.sphinx.org.cn/usage/theming.html\#builtin-themes}{内置主题信息}


\subsection{html\_theme\_options}
\label{\detokenize{sphinx_conf:html-theme-options}}
\sphinxAtStartPar
所选主题外观的选项字典。


\subsection{html\_theme\_path}
\label{\detokenize{sphinx_conf:html-theme-path}}
\sphinxAtStartPar
包含自定义主题的路径列表, 可以是子目录, 也可以是zip文件。相对路径被视为相对于配置目录。


\subsection{html\_style}
\label{\detokenize{sphinx_conf:html-style}}
\sphinxAtStartPar
用于HTML页面的样式表。该名称的文件必须存在于Sphinx的 static/ 路径中, 或者存在于 html\_static\_path 中给出的自定义路径之一。
默认值是所选主题给出的样式表。如果您只想添加或覆盖与主题样式表相比的一些内容, 使用CSS @import 导入主题的样式表。


\subsection{html\_title}
\label{\detokenize{sphinx_conf:html-title}}
\sphinxAtStartPar
使用Sphinx内置模板生成的html页面的title。默认 \sphinxtitleref{<project> v<revision> documentation}

\sphinxAtStartPar
html\_short\_title
在HTML docs 和 HTML Help docs 使用的 html title。
默认使用 html\_title 的设置值


\subsection{html\_baseurl}
\label{\detokenize{sphinx_conf:html-baseurl}}
\sphinxAtStartPar
指向HTML文档根目录的URL。它用于表示文档的位置, 如 canonical\_url 。

\sphinxAtStartPar
html\_context
要传递到所有页面的模板引擎上下文的值字典。
单个值也可以使用sphinx\sphinxhyphen{}build 的\sphinxhyphen{}A命令行选项来设置。


\subsection{html\_logo}
\label{\detokenize{sphinx_conf:html-logo}}
\sphinxAtStartPar
文档的徽标,位于侧边栏的顶部,宽度不应超过200像素。默认值:None。

\sphinxAtStartPar
如果给定, 则必须是图像文件的名称(相对于 configuration directory 的路径)。

\sphinxAtStartPar
若图片文件不存在于 \_static目录,将被复制到输出HTML的 \_static 目录中。


\subsection{html\_favicon}
\label{\detokenize{sphinx_conf:html-favicon}}
\sphinxAtStartPar
文档的favicon,现代浏览器使用它作为标签, 窗口和书签的图标。它应该是一个Windows风格的图标文件(.ico), 大小为16x16或32x32像素。默认值: None 。

\sphinxAtStartPar
如果给定, 则必须是图像文件的名称(相对于 configuration directory 的路径)。

\sphinxAtStartPar
若图片文件不存在于 \_static目录,将被复制到输出HTML的 \_static 目录中。


\subsection{html\_css\_files}
\label{\detokenize{sphinx_conf:html-css-files}}
\sphinxAtStartPar
CSS文件列表。该条目必须是filename字符串或包含filename 字符串和attributes字典的元组。
filename 必须相对于 html\_static\_path , 或者是一个完整的URI, 如 \sphinxurl{http://example.org/style.css} 。
attributes 用于 <link> 标签的属性。默认为空列表。

\begin{sphinxVerbatim}[commandchars=\\\{\}]
\PYG{n}{html\PYGZus{}css\PYGZus{}files} \PYG{o}{=} \PYG{p}{[}\PYG{l+s+s1}{\PYGZsq{}}\PYG{l+s+s1}{custom.css}\PYG{l+s+s1}{\PYGZsq{}}
          \PYG{l+s+s1}{\PYGZsq{}}\PYG{l+s+s1}{https://example.com/css/custom.css}\PYG{l+s+s1}{\PYGZsq{}}\PYG{p}{,}
          \PYG{p}{(}\PYG{l+s+s1}{\PYGZsq{}}\PYG{l+s+s1}{print.css}\PYG{l+s+s1}{\PYGZsq{}}\PYG{p}{,} \PYG{p}{\PYGZob{}}\PYG{l+s+s1}{\PYGZsq{}}\PYG{l+s+s1}{media}\PYG{l+s+s1}{\PYGZsq{}}\PYG{p}{:} \PYG{l+s+s1}{\PYGZsq{}}\PYG{l+s+s1}{print}\PYG{l+s+s1}{\PYGZsq{}}\PYG{p}{\PYGZcb{}}\PYG{p}{)}\PYG{p}{]}
\end{sphinxVerbatim}


\subsection{html\_js\_files}
\label{\detokenize{sphinx_conf:html-js-files}}
\sphinxAtStartPar
JavaScript filename 列表。该条目必须是 filename 字符串或包含 filename 字符串和 attributes 字典的元组。
filename 必须相对于 html\_static\_path , 或者是一个完整的URI, 如 \sphinxurl{http://example.org/script.js} 。
attributes 用于 <script> 标签的属性。默认为空列表。

\begin{sphinxVerbatim}[commandchars=\\\{\}]
\PYG{n}{html\PYGZus{}js\PYGZus{}files} \PYG{o}{=} \PYG{p}{[}\PYG{l+s+s1}{\PYGZsq{}}\PYG{l+s+s1}{script.js}\PYG{l+s+s1}{\PYGZsq{}}\PYG{p}{,}
             \PYG{l+s+s1}{\PYGZsq{}}\PYG{l+s+s1}{https://example.com/scripts/custom.js}\PYG{l+s+s1}{\PYGZsq{}}\PYG{p}{,}
             \PYG{p}{(}\PYG{l+s+s1}{\PYGZsq{}}\PYG{l+s+s1}{custom.js}\PYG{l+s+s1}{\PYGZsq{}}\PYG{p}{,} \PYG{p}{\PYGZob{}}\PYG{l+s+s1}{\PYGZsq{}}\PYG{l+s+s1}{async}\PYG{l+s+s1}{\PYGZsq{}}\PYG{p}{:} \PYG{l+s+s1}{\PYGZsq{}}\PYG{l+s+s1}{async}\PYG{l+s+s1}{\PYGZsq{}}\PYG{p}{\PYGZcb{}}\PYG{p}{)}\PYG{p}{]}
\end{sphinxVerbatim}


\subsection{html\_static\_path}
\label{\detokenize{sphinx_conf:html-static-path}}
\sphinxAtStartPar
包含自定义静态文件(例如样式表css或脚本文件js)的路径列表。

\sphinxAtStartPar
相对路径被视为相对于配置目录。它们被复制到主题的静态文件之后的输出的 \_static 目录中,
因此名为 default.css 的文件将覆盖主题的 default.css 。

\sphinxAtStartPar
由于这些文件不是要构建的, 因此它们会自动从源文件中排除。


\subsection{html\_extra\_path}
\label{\detokenize{sphinx_conf:html-extra-path}}
\sphinxAtStartPar
包含与文档无直接关系的额外文件的路径列表, 例如 robots.txt 或 .htaccess 。

\sphinxAtStartPar
相对路径被视为相对于配置目录。它们被复制到输出目录。它们将覆盖任何同名的现有文件。

\sphinxAtStartPar
由于这些文件不是要构建的, 因此它们会自动从源文件中排除。


\subsection{html\_last\_updated\_fmt}
\label{\detokenize{sphinx_conf:html-last-updated-fmt}}
\sphinxAtStartPar
如果这不是None, 则使用给定的 strftime() 格式在每个页面底部插入 ‘Last updated on:’ 时间戳。空字符串相当于 ‘%b%d, %Y’ (或依赖于语言环境的等价物)。


\subsection{html\_use\_smartypants}
\label{\detokenize{sphinx_conf:html-use-smartypants}}
\sphinxAtStartPar
如果为true, 则引号和短划线将转换为印刷正确的实体。默认值: True 。


\subsection{html\_add\_permalinks}
\label{\detokenize{sphinx_conf:html-add-permalinks}}
\sphinxAtStartPar
Sphinx will add “permalinks” for each heading and description environment as paragraph signs that become visible when the mouse hovers over them.

\sphinxAtStartPar
This value determines the text for the permalink; it defaults to “¶”. Set it to None or the empty string to disable permalinks.


\subsection{html\_sidebars}
\label{\detokenize{sphinx_conf:html-sidebars}}
\sphinxAtStartPar
自定义侧边栏模板必须是将文档名称映射到模板名称的字典。

\sphinxAtStartPar
键可以包含glob样式的模式, 所有匹配的文档都将获得指定的侧边栏。(当多个glob样式模式与任何文档匹配时会发出警告)

\sphinxAtStartPar
值是列表,它指定要包括的侧边栏模板的完整列表。如果要包含所有或部分默认侧边栏, 则必须将它们放入此列表中。
默认侧边栏(适用于与任何模式不匹配的文档)由主题本身定义。
内置主题默认使用这些模板: {[}‘localtoc.html’, ‘relations.html’ , ‘sourcelink.html’ , ‘searchbox.html’{]}

\sphinxAtStartPar
可呈现的内置侧边栏模板是:
\begin{itemize}
\item {} 
\sphinxAtStartPar
localtoc.html \sphinxhyphen{} 当前文档的细粒度目录

\item {} 
\sphinxAtStartPar
globaltoc.html – 折叠整个文档集的粗粒度目录

\item {} 
\sphinxAtStartPar
relations.html – 两个指向上一个和下一个文档的链接

\item {} 
\sphinxAtStartPar
sourcelink.html – 指向当前文档源的链接(如果在 html\_show\_sourcelink 中启用)

\item {} 
\sphinxAtStartPar
searchbox.html – the “quick search” box

\end{itemize}

\begin{sphinxVerbatim}[commandchars=\\\{\}]
 \PYG{n}{html\PYGZus{}sidebars} \PYG{o}{=} \PYG{p}{\PYGZob{}}
\PYG{l+s+s1}{\PYGZsq{}}\PYG{l+s+s1}{**}\PYG{l+s+s1}{\PYGZsq{}}\PYG{p}{:} \PYG{p}{[}\PYG{l+s+s1}{\PYGZsq{}}\PYG{l+s+s1}{globaltoc.html}\PYG{l+s+s1}{\PYGZsq{}}\PYG{p}{,} \PYG{l+s+s1}{\PYGZsq{}}\PYG{l+s+s1}{sourcelink.html}\PYG{l+s+s1}{\PYGZsq{}}\PYG{p}{,} \PYG{l+s+s1}{\PYGZsq{}}\PYG{l+s+s1}{searchbox.html}\PYG{l+s+s1}{\PYGZsq{}}\PYG{p}{]}\PYG{p}{,}
\PYG{l+s+s1}{\PYGZsq{}}\PYG{l+s+s1}{using/windows}\PYG{l+s+s1}{\PYGZsq{}}\PYG{p}{:} \PYG{p}{[}\PYG{l+s+s1}{\PYGZsq{}}\PYG{l+s+s1}{windowssidebar.html}\PYG{l+s+s1}{\PYGZsq{}}\PYG{p}{,} \PYG{l+s+s1}{\PYGZsq{}}\PYG{l+s+s1}{searchbox.html}\PYG{l+s+s1}{\PYGZsq{}}\PYG{p}{]}\PYG{p}{,}\PYG{p}{\PYGZcb{}}
\end{sphinxVerbatim}

\sphinxAtStartPar
这将呈现自定义模板 windowssidebar.html 和给定文档侧边栏内的快速搜索框, 并呈现所有其他页面的默认侧边栏(除了本地TOC被全局TOC替换)。

\begin{DUlineblock}{0em}
\item[] 如果所选主题不具有侧边栏, 则此值仅无效, 例如内置 scrolls 和 haiku 。
\end{DUlineblock}


\subsection{html\_additional\_pages}
\label{\detokenize{sphinx_conf:html-additional-pages}}
\sphinxAtStartPar
为HTML页面指定其他模板,为文档名称映射到模板名称的字典。

\begin{sphinxVerbatim}[commandchars=\\\{\}]
\PYG{n}{html\PYGZus{}additional\PYGZus{}pages} \PYG{o}{=} \PYG{p}{\PYGZob{}}
  \PYG{l+s+s1}{\PYGZsq{}}\PYG{l+s+s1}{download}\PYG{l+s+s1}{\PYGZsq{}}\PYG{p}{:} \PYG{l+s+s1}{\PYGZsq{}}\PYG{l+s+s1}{customdownload.html}\PYG{l+s+s1}{\PYGZsq{}}\PYG{p}{,}\PYG{p}{\PYGZcb{}}
\end{sphinxVerbatim}

\sphinxAtStartPar
这将把模板 customdownload.html 渲染为页面 download.html 。


\subsection{html\_domain\_indices}
\label{\detokenize{sphinx_conf:html-domain-indices}}
\sphinxAtStartPar
如果为true, 则除了常规索引外, 还会生成特定于域的索引。对于例如Python域, 这是全局模块索引。默认为 True 。

\sphinxAtStartPar
此值可以是bool或应生成的索引名称列表。要查找特定索引的索引名称, 请查看HTML文件名。
例如, Python模块索引的名称为 ‘py\sphinxhyphen{}modindex’ 。


\subsection{html\_use\_index}
\label{\detokenize{sphinx_conf:html-use-index}}
\sphinxAtStartPar
默认为True,为HTML文档添加索引。


\subsection{html\_split\_index}
\label{\detokenize{sphinx_conf:html-split-index}}
\sphinxAtStartPar
默认False,如果为true, 则索引生成两次:一次作为包含所有条目的单个页面, 一次作为每个起始字母的一个页面。


\subsection{html\_copy\_source}
\label{\detokenize{sphinx_conf:html-copy-source}}
\sphinxAtStartPar
默认为true, reST源包含在HTML构建中 \_sources/name


\subsection{html\_show\_sourcelink}
\label{\detokenize{sphinx_conf:html-show-sourcelink}}
\sphinxAtStartPar
默认为true(并且 html\_copy\_source 也为 true ), 则指向reST源的链接将添加到侧栏。


\subsection{html\_sourcelink\_suffix}
\label{\detokenize{sphinx_conf:html-sourcelink-suffix}}
\sphinxAtStartPar
附加到源链接的后缀(参见html\_show\_sourcelink), 除非它们已经有这个后缀。默认是 ‘.txt’ 。


\subsection{html\_use\_opensearch}
\label{\detokenize{sphinx_conf:html-use-opensearch}}
\sphinxAtStartPar
If nonempty, an OpenSearch description file will be output, and all pages will contain a <link> tag referring to it.
Since OpenSearch doesn’t support relative URLs for its search page location,
the value of this option must be the base URL from which these documents are served (without trailing slash), e.g. “\sphinxurl{https://docs.python.org}”. The default is ‘’.


\subsection{html\_file\_suffix}
\label{\detokenize{sphinx_conf:html-file-suffix}}
\sphinxAtStartPar
生成的HTML文件后缀,默认为 “.html”


\subsection{html\_link\_suffix}
\label{\detokenize{sphinx_conf:html-link-suffix}}
\sphinxAtStartPar
生成HTML文件链接的后缀。默认值为 html\_file\_suffix 设置值;它可以设置不同(例如, 支持不同的Web服务器设置)。


\subsection{html\_show\_copyright}
\label{\detokenize{sphinx_conf:html-show-copyright}}
\sphinxAtStartPar
在HTML Footer显示 “(C) Copyright …”, 默认True


\subsection{html\_show\_sphinx}
\label{\detokenize{sphinx_conf:html-show-sphinx}}
\sphinxAtStartPar
在HTML Footer显示 “Created using Sphinx” ,”Built with Sphinx”,默认True


\subsection{html\_output\_encoding}
\label{\detokenize{sphinx_conf:html-output-encoding}}
\sphinxAtStartPar
HTML输出文件的编码。默认为 ‘utf\sphinxhyphen{}8’


\subsection{html\_compact\_lists}
\label{\detokenize{sphinx_conf:html-compact-lists}}
\sphinxAtStartPar
默认为True, 如列表中包含单个段落和/或子列表的所有项目等等…(递归定义),将不会对其任何项目使用 <p> 元素。
这是标准的docutils行为。


\subsection{html\_secnumber\_suffix}
\label{\detokenize{sphinx_conf:html-secnumber-suffix}}
\sphinxAtStartPar
章节编号的后缀(最后一个,与章节标题相连的部分),默认是”. “, 比如 “2.1.1. 章节标题”,可设置为” “(空格)。


\subsection{html\_search\_language}
\label{\detokenize{sphinx_conf:html-search-language}}
\sphinxAtStartPar
全文检索使用的语言,默认为en

\sphinxAtStartPar
支持这些语言:
\begin{itemize}
\item {} 
\sphinxAtStartPar
da – 丹麦语

\item {} 
\sphinxAtStartPar
nl – 荷兰语

\item {} 
\sphinxAtStartPar
en – 英语

\item {} 
\sphinxAtStartPar
fi – 芬兰语

\item {} 
\sphinxAtStartPar
fr – 法语

\item {} 
\sphinxAtStartPar
de – 德语

\item {} 
\sphinxAtStartPar
hu – 匈牙利语

\item {} 
\sphinxAtStartPar
it – 意大利语

\item {} 
\sphinxAtStartPar
ja – 日语

\item {} 
\sphinxAtStartPar
no – 挪威语

\item {} 
\sphinxAtStartPar
pt – 葡萄牙语

\item {} 
\sphinxAtStartPar
ro – 罗马尼亚语

\item {} 
\sphinxAtStartPar
ru – 俄语

\item {} 
\sphinxAtStartPar
es – 西班牙语

\item {} 
\sphinxAtStartPar
sv – 瑞典语

\item {} 
\sphinxAtStartPar
tr – 土耳其语

\item {} 
\sphinxAtStartPar
zh – 中文

\end{itemize}

\sphinxAtStartPar
每种语言(日语除外)都提供自己的词干算法。 Sphinx默认使用Python实现。您可以使用C实现来加速构建索引文件。
\sphinxhref{https://pypi.org/project/PorterStemmer/}{PorterStemmer} (en),
\sphinxhref{https://pypi.org/project/PyStemmer/}{PyStemmer} (所有语言)


\subsection{html\_search\_options}
\label{\detokenize{sphinx_conf:html-search-options}}
\sphinxAtStartPar
带有搜索语言支持选项的字典, 默认为空。这些选项的含义取决于所选语言。

\sphinxAtStartPar
英语支持没有选择。

\sphinxAtStartPar
日本的支持有这些选择:

\sphinxAtStartPar
Type: type 是点模块路径字符串, 用于指定应该从哪实现 sphinx.search.ja.BaseSplitter。
如果未指定或指定None, 将使用 ‘sphinx.search.ja.DefaultSplitter’ 。

\sphinxAtStartPar
可以从以下模块中进行选择:
\begin{itemize}
\item {} 
\sphinxAtStartPar
‘sphinx.search.ja.DefaultSplitter’: TinySegmenter algorithm. 这是默认分割器。

\item {} 
\sphinxAtStartPar
‘sphinx.search.ja.MecabSplitter’: MeCab绑定。要使用这个拆分器, 需要 ‘mecab’ python绑定或动态链接库( ‘libmecab.so’ 用于linux, ‘libmecab.dll’ 用于windows)。

\item {} 
\sphinxAtStartPar
‘sphinx.search.ja.JanomeSplitter’: Janome绑定。要使用这个分离器, 需要 Janome 。

\end{itemize}

\sphinxAtStartPar
1.6 版后已移除: ‘mecab’, ‘janome’ and ‘default’ 已弃用. 为了保持兼容性, ‘mecab’, ‘janome’ and ‘default’ 也可以接受。

\sphinxAtStartPar
其他选项值取决于您选择的拆分器值。

\sphinxAtStartPar
‘mecab’ 的选项:
+ dic\_enc: MeCab算法的编码。
+ dict: 用于MeCab算法的字典。
+ lib: 用于在未安装Python绑定的情况下通过ctypes查找MeCab库的库名。

\begin{sphinxVerbatim}[commandchars=\\\{\}]
\PYG{n}{html\PYGZus{}search\PYGZus{}options} \PYG{o}{=} \PYG{p}{\PYGZob{}}
\PYG{l+s+s1}{\PYGZsq{}}\PYG{l+s+s1}{type}\PYG{l+s+s1}{\PYGZsq{}}\PYG{p}{:} \PYG{l+s+s1}{\PYGZsq{}}\PYG{l+s+s1}{mecab}\PYG{l+s+s1}{\PYGZsq{}}\PYG{p}{,}
\PYG{l+s+s1}{\PYGZsq{}}\PYG{l+s+s1}{dic\PYGZus{}enc}\PYG{l+s+s1}{\PYGZsq{}}\PYG{p}{:} \PYG{l+s+s1}{\PYGZsq{}}\PYG{l+s+s1}{utf\PYGZhy{}8}\PYG{l+s+s1}{\PYGZsq{}}\PYG{p}{,}
\PYG{l+s+s1}{\PYGZsq{}}\PYG{l+s+s1}{dict}\PYG{l+s+s1}{\PYGZsq{}}\PYG{p}{:} \PYG{l+s+s1}{\PYGZsq{}}\PYG{l+s+s1}{/path/to/mecab.dic}\PYG{l+s+s1}{\PYGZsq{}}\PYG{p}{,}
\PYG{l+s+s1}{\PYGZsq{}}\PYG{l+s+s1}{lib}\PYG{l+s+s1}{\PYGZsq{}}\PYG{p}{:} \PYG{l+s+s1}{\PYGZsq{}}\PYG{l+s+s1}{/path/to/libmecab.so}\PYG{l+s+s1}{\PYGZsq{}}\PYG{p}{,}\PYG{p}{\PYGZcb{}}
\end{sphinxVerbatim}

\sphinxAtStartPar
‘janome’ 的选项:
\begin{itemize}
\item {} 
\sphinxAtStartPar
user\_dic : Janome的用户词典文件路径。

\item {} 
\sphinxAtStartPar
user\_dic\_enc : user\_dic选项指定的用户词典文件的编码。默认为 \sphinxtitleref{utf8}

\end{itemize}

\sphinxAtStartPar
中文的支持有这些选择:
dict – 如果想使用自定义词典, jieba 字典路径。


\subsection{html\_search\_scorer}
\label{\detokenize{sphinx_conf:html-search-scorer}}
\sphinxAtStartPar
实现搜索结果记分器的JavaScript文件的名称(相对于配置目录)。如果为空, 则使用默认值。


\subsection{html\_scaled\_image\_link}
\label{\detokenize{sphinx_conf:html-scaled-image-link}}
\sphinxAtStartPar
默认为True,图像本身会链接到原始图像(如果它没有target选项或缩放相关选项: \sphinxtitleref{scale} , \sphinxtitleref{width} , \sphinxtitleref{height}


\subsection{html\_math\_renderer}
\label{\detokenize{sphinx_conf:html-math-renderer}}
\sphinxAtStartPar
HTML输出的math\_renderer扩展名。默认为 \sphinxcode{\sphinxupquote{mathjax}} 。


\subsection{singlehtml\_sidebars}
\label{\detokenize{sphinx_conf:singlehtml-sidebars}}
\sphinxAtStartPar
单个HTML页面输出选项,自定义侧边栏模板必须是将文档名称映射到模板名称的字典。它只允许一个名为 “index” 的键。
所有其他键都被忽略。默认情况下,与 html\_sidebars 相同。


\subsection{htmlhelp\_basename}
\label{\detokenize{sphinx_conf:htmlhelp-basename}}
\sphinxAtStartPar
HTML帮助构建器的输出文件基名。默认是 \sphinxcode{\sphinxupquote{pydoc}}


\subsection{htmlhelp\_file\_suffix}
\label{\detokenize{sphinx_conf:htmlhelp-file-suffix}}
\sphinxAtStartPar
HTML帮助文档文件名后缀,默认 \sphinxcode{\sphinxupquote{.html}}


\subsection{htmlhelp\_link\_suffix}
\label{\detokenize{sphinx_conf:htmlhelp-link-suffix}}
\sphinxAtStartPar
HTML帮助文档链接后缀,默认 \sphinxtitleref{.html}


\section{EPUB输出配置项}
\label{\detokenize{sphinx_conf:epub}}
\sphinxAtStartPar
\sphinxhref{https://www.sphinx.org.cn/usage/configuration.html\#options-for-epub-output}{EPUB输出配置项}


\section{LaTeX输出配置项}
\label{\detokenize{sphinx_conf:latex}}
\sphinxAtStartPar
\sphinxhref{https://www.sphinx.org.cn/usage/configuration.html\#options-for-latex-output}{LaTeX输出配置项}


\section{文本输出选项}
\label{\detokenize{sphinx_conf:id9}}

\subsection{text\_newlines}
\label{\detokenize{sphinx_conf:text-newlines}}
\sphinxAtStartPar
确定在文本输出中使用哪个行尾字符。
\begin{itemize}
\item {} 
\sphinxAtStartPar
‘unix’: 使用Unix风格的行结尾(\sphinxcode{\sphinxupquote{\textbackslash{}n}})

\item {} 
\sphinxAtStartPar
‘windows’: 使用Windows风格的行结尾(\sphinxcode{\sphinxupquote{\textbackslash{}r\textbackslash{}n}})

\item {} 
\sphinxAtStartPar
‘native’: 使用构建文档的平台的行结束样式

\end{itemize}

\sphinxAtStartPar
默认值: ‘unix’ 。


\subsection{text\_sectionchars}
\label{\detokenize{sphinx_conf:text-sectionchars}}
\sphinxAtStartPar
一个7个字符的字符串, 应该用于下划线部分。第一个字符用于第一级标题, 第二个字符用于第二级标题, 依此类推。

\sphinxAtStartPar
默认为 \sphinxcode{\sphinxupquote{*=\sphinxhyphen{}\textasciitilde{}"+`}}


\subsection{text\_add\_secnumbers}
\label{\detokenize{sphinx_conf:text-add-secnumbers}}
\sphinxAtStartPar
一个布尔值, 用于决定文本输出中是否包含节号。默认为 True 。


\subsection{text\_secnumber\_suffix}
\label{\detokenize{sphinx_conf:text-secnumber-suffix}}
\sphinxAtStartPar
章节编号的后缀(最后一个,与章节标题相连的部分),默认是 “. “ ,比如 \sphinxcode{\sphinxupquote{2.1.1. 章节标题}} ,可设置为 “ “(空格)。


\section{HTML主题}
\label{\detokenize{sphinx_conf:id10}}
\sphinxAtStartPar
Sphinx支持通过 themes 更改其HTML输出的外观。

\sphinxAtStartPar
主题是HTML模板, 样式表和其他静态文件的集合。此外, 它还有一个配置文件, 用于指定要继承的主题, 要使用的突出显示样式以及用于自定义主题外观的选项。

\sphinxAtStartPar
也可以自己 \sphinxhref{https://www.sphinx.org.cn/theming.html}{制作主题}


\subsection{使用(配置)主题}
\label{\detokenize{sphinx_conf:id12}}

\subsubsection{使用内置主题}
\label{\detokenize{sphinx_conf:id13}}
\sphinxAtStartPar
设置 \sphinxcode{\sphinxupquote{conf.py}} 中 \sphinxcode{\sphinxupquote{html\_theme}} 的值即可

\begin{sphinxVerbatim}[commandchars=\\\{\}]
\PYG{n}{html\PYGZus{}theme} \PYG{o}{=} \PYG{l+s+s1}{\PYGZsq{}}\PYG{l+s+s1}{classic}\PYG{l+s+s1}{\PYGZsq{}}
\end{sphinxVerbatim}

\sphinxAtStartPar
修改主题的一些配置选项,修改 \sphinxcode{\sphinxupquote{html\_theme\_options}} 配置项,
对于使用的主题适用于哪些修改,视具体主题而定。

\begin{sphinxVerbatim}[commandchars=\\\{\}]
\PYG{n}{html\PYGZus{}theme\PYGZus{}options} \PYG{o}{=} \PYG{p}{\PYGZob{}}
 \PYG{l+s+s2}{\PYGZdq{}}\PYG{l+s+s2}{rightsidebar}\PYG{l+s+s2}{\PYGZdq{}}\PYG{p}{:} \PYG{l+s+s2}{\PYGZdq{}}\PYG{l+s+s2}{true}\PYG{l+s+s2}{\PYGZdq{}}\PYG{p}{,}
 \PYG{l+s+s2}{\PYGZdq{}}\PYG{l+s+s2}{relbarbgcolor}\PYG{l+s+s2}{\PYGZdq{}}\PYG{p}{:} \PYG{l+s+s2}{\PYGZdq{}}\PYG{l+s+s2}{black}\PYG{l+s+s2}{\PYGZdq{}}\PYG{p}{\PYGZcb{}}
\end{sphinxVerbatim}


\subsubsection{使用自定义主题}
\label{\detokenize{sphinx_conf:id14}}
\sphinxAtStartPar
自定义主题可以是静态文件形式或Python包。
对于静态表单, 支持目录(包含 theme.conf 和其他所需文件)或具有相同内容的zip文件。
有配置值 html\_theme\_path,路径为相对conf.py所在目录的**相对路径** ,

\sphinxAtStartPar
例如, 如果文件中有一个主题 blue.zip, 则可以将其放在包含 conf.py 的目录中并使用此配置

\begin{sphinxVerbatim}[commandchars=\\\{\}]
\PYG{n}{html\PYGZus{}theme} \PYG{o}{=} \PYG{l+s+s2}{\PYGZdq{}}\PYG{l+s+s2}{blue}\PYG{l+s+s2}{\PYGZdq{}}
\PYG{n}{html\PYGZus{}theme\PYGZus{}path} \PYG{o}{=} \PYG{p}{[}\PYG{l+s+s2}{\PYGZdq{}}\PYG{l+s+s2}{.}\PYG{l+s+s2}{\PYGZdq{}}\PYG{p}{]}
\end{sphinxVerbatim}


\subsubsection{python包主题}
\label{\detokenize{sphinx_conf:python}}
\sphinxAtStartPar
使用 pip 安装主题包之后,和上一小节一样的流程配置即可。

\begin{sphinxVerbatim}[commandchars=\\\{\}]
\PYG{n}{pip} \PYG{n}{install} \PYG{n}{sphinxjp}\PYG{o}{.}\PYG{n}{themes}\PYG{o}{.}\PYG{n}{dotted}
\end{sphinxVerbatim}


\subsection{内置主题}
\label{\detokenize{sphinx_conf:id15}}

\begin{wrapfigure}{l}{0.450\linewidth}
\centering
\noindent\sphinxincludegraphics{{alabaster}.png}
\caption{alabaster}\label{\detokenize{sphinx_conf:id23}}\end{wrapfigure}

\mbox{}\par\vskip-\dimexpr\baselineskip+\parskip\relax


\begin{wrapfigure}{r}{0.450\linewidth}
\centering
\noindent\sphinxincludegraphics{{agogo}.png}
\caption{agogo}\label{\detokenize{sphinx_conf:id24}}\end{wrapfigure}

\mbox{}\par\vskip-\dimexpr\baselineskip+\parskip\relax


\begin{wrapfigure}{l}{0.450\linewidth}
\centering
\noindent\sphinxincludegraphics{{bizstyle}.png}
\caption{bizstyle}\label{\detokenize{sphinx_conf:id25}}\end{wrapfigure}

\mbox{}\par\vskip-\dimexpr\baselineskip+\parskip\relax


\begin{wrapfigure}{r}{0.450\linewidth}
\centering
\noindent\sphinxincludegraphics{{classic}.png}
\caption{classic}\label{\detokenize{sphinx_conf:id26}}\end{wrapfigure}

\mbox{}\par\vskip-\dimexpr\baselineskip+\parskip\relax


\begin{wrapfigure}{l}{0.450\linewidth}
\centering
\noindent\sphinxincludegraphics{{haiku}.png}
\caption{haiku}\label{\detokenize{sphinx_conf:id27}}\end{wrapfigure}

\mbox{}\par\vskip-\dimexpr\baselineskip+\parskip\relax


\begin{wrapfigure}{r}{0.450\linewidth}
\centering
\noindent\sphinxincludegraphics{{nature}.png}
\caption{nature}\label{\detokenize{sphinx_conf:id28}}\end{wrapfigure}

\mbox{}\par\vskip-\dimexpr\baselineskip+\parskip\relax


\begin{wrapfigure}{l}{0.450\linewidth}
\centering
\noindent\sphinxincludegraphics{{pyramid}.png}
\caption{pyramid}\label{\detokenize{sphinx_conf:id29}}\end{wrapfigure}

\mbox{}\par\vskip-\dimexpr\baselineskip+\parskip\relax


\begin{wrapfigure}{r}{0.450\linewidth}
\centering
\noindent\sphinxincludegraphics{{sphinxdoc}.png}
\caption{sphinxdoc}\label{\detokenize{sphinx_conf:id30}}\end{wrapfigure}

\mbox{}\par\vskip-\dimexpr\baselineskip+\parskip\relax


\begin{wrapfigure}{l}{0.450\linewidth}
\centering
\noindent\sphinxincludegraphics{{scrolls}.png}
\caption{scrolls}\label{\detokenize{sphinx_conf:id31}}\end{wrapfigure}

\mbox{}\par\vskip-\dimexpr\baselineskip+\parskip\relax


\begin{wrapfigure}{r}{0.450\linewidth}
\centering
\noindent\sphinxincludegraphics{{traditional}.png}
\caption{traditional}\label{\detokenize{sphinx_conf:id32}}\end{wrapfigure}

\mbox{}\par\vskip-\dimexpr\baselineskip+\parskip\relax


\subsubsection{basic}
\label{\detokenize{sphinx_conf:basic}}
\sphinxAtStartPar
基本上没有样式的布局, 用作其他主题的基础, 也可用作自定义主题的基础
\begin{itemize}
\item {} 
\sphinxAtStartPar
nosidebar (true or false): 不包括侧边栏. 默认为 False .

\item {} 
\sphinxAtStartPar
sidebarwidth (int或str): 侧边栏的宽度(以像素为单位). 这可以是 int, 它被解释为像素或有效的CSS维度字符串, 例如 ‘70em’ 或 ‘50%’. 默认为230像素.

\item {} 
\sphinxAtStartPar
body\_min\_width (int或str):文档正文的最小宽度. 这可以是int, 它被解释为像素或有效的CSS维度字符串, 例如’70em’或’50%’. 如果您不想要宽度限制, 请使用0. 默认值可能取决于主题(通常为450px).

\item {} 
\sphinxAtStartPar
body\_max\_width (int或str):文档正文的最大宽度. 这可以是int, 它被解释为像素或有效的CSS维度字符串, 例如’70em’或’50%’. 如果您不想要宽度限制, 请使用 none . 默认值可能取决于主题(通常为800px).

\end{itemize}


\subsubsection{alabaster}
\label{\detokenize{sphinx_conf:alabaster}}
\sphinxAtStartPar
来自@kennethreitz的修改后的 Kr Sphinx主题(Requests项目中使用), 它本身最初基于@mitsuhiko用于Flask及相关项目的主题。
\sphinxhref{https://alabaster.readthedocs.io/en/latest/installation.html}{配置信息}


\subsubsection{classic}
\label{\detokenize{sphinx_conf:classic}}
\sphinxAtStartPar
经典主题
\begin{itemize}
\item {} 
\sphinxAtStartPar
rightsidebar (true or false):将侧边栏放在右侧。默认为 False

\item {} 
\sphinxAtStartPar
stickysidebar (true or false):使侧边栏固定。默认为False

\item {} 
\sphinxAtStartPar
collapsiblesidebar (true or false):添加一个实验性 JavaScript代码段, 通过侧面的按钮使侧边栏可折叠。默认为False

\item {} 
\sphinxAtStartPar
externalrefs (true或false):显示外部链接与内部链接不同。默认为 False

\end{itemize}

\sphinxAtStartPar
还有各种颜色和字体选项可以更改颜色方案, 而无需编写自定义样式表:
\begin{itemize}
\item {} 
\sphinxAtStartPar
footerbgcolor (CSS颜色):页脚行的背景颜色.

\item {} 
\sphinxAtStartPar
footertextcolor (CSS颜色):页脚行的文本颜色.

\item {} 
\sphinxAtStartPar
sidebarbgcolor (CSS颜色):侧边栏的背景颜色.

\item {} 
\sphinxAtStartPar
sidebarbtncolor (CSS颜色:侧边栏折叠按钮的背景颜色(当 collapsiblesidebar 为 True 时使用)。

\item {} 
\sphinxAtStartPar
sidebartextcolor (CSS颜色):侧边栏的文本颜色.

\item {} 
\sphinxAtStartPar
sidebarlinkcolor (CSS颜色):侧边栏的链接颜色.

\item {} 
\sphinxAtStartPar
relbarbgcolor (CSS颜色):关系栏的背景颜色.

\item {} 
\sphinxAtStartPar
relbartextcolor (CSS颜色): 关系栏的文本颜色.

\item {} 
\sphinxAtStartPar
relbarlinkcolor (CSS颜色):关系栏的链接颜色.

\item {} 
\sphinxAtStartPar
bgcolor (CSS颜色):身体背景颜色.

\item {} 
\sphinxAtStartPar
textcolor (CSS颜色):正文文本颜色.

\item {} 
\sphinxAtStartPar
linkcolor (CSS颜色):正文链接颜色.

\item {} 
\sphinxAtStartPar
visitedlinkcolor (CSS颜色):访问过的链接的正文颜色.

\item {} 
\sphinxAtStartPar
headbgcolor (CSS颜色):标题的背景颜色.

\item {} 
\sphinxAtStartPar
headtextcolor (CSS颜色):标题的文本颜色.

\item {} 
\sphinxAtStartPar
headlinkcolor (CSS颜色):标题的链接颜色.

\item {} 
\sphinxAtStartPar
codebgcolor (CSS颜色):代码块的背景颜色.

\item {} 
\sphinxAtStartPar
codetextcolor (CSS颜色): 代码块的默认文本颜色,如果没有通过突出显示样式设置不同.

\item {} 
\sphinxAtStartPar
bodyfont (CSS字体系列):普通文本的字体.

\item {} 
\sphinxAtStartPar
headfont (CSS字体系列):标题的字体.

\end{itemize}


\subsubsection{sphinxdoc}
\label{\detokenize{sphinx_conf:sphinxdoc}}
\sphinxAtStartPar
可配置nosidebar, sidebarwidth


\subsubsection{scrolls}
\label{\detokenize{sphinx_conf:scrolls}}
\sphinxAtStartPar
一个更轻量级的主题, 基于 \sphinxhref{http://jinja.pocoo.org/}{Jinja} 文档。有以下颜色选项:
\begin{itemize}
\item {} 
\sphinxAtStartPar
headerbordercolor

\item {} 
\sphinxAtStartPar
subheadlinecolor

\item {} 
\sphinxAtStartPar
linkcolor

\item {} 
\sphinxAtStartPar
visitedlinkcolor

\item {} 
\sphinxAtStartPar
admonitioncolor

\end{itemize}


\subsubsection{agogo}
\label{\detokenize{sphinx_conf:agogo}}\begin{itemize}
\item {} 
\sphinxAtStartPar
bodyfont (CSS字体系列):普通文本的字体.

\item {} 
\sphinxAtStartPar
headerfont (CSS字体系列):标题字体.

\item {} 
\sphinxAtStartPar
pagewidth (CSS长度):页面内容的宽度, 默认为70em.

\item {} 
\sphinxAtStartPar
documentwidth (CSS长度):文档的宽度(没有侧边栏), 默认为50em.

\item {} 
\sphinxAtStartPar
sidebarwidth (CSS长度):侧边栏的宽度, 默认为20em.

\item {} 
\sphinxAtStartPar
bgcolor (CSS color): 背景颜色.

\item {} 
\sphinxAtStartPar
headerbg (“background” 的CSS值):标题区域的背景, 默认为浅灰色渐变.

\item {} 
\sphinxAtStartPar
footerbg (“background” 的CSS值):页脚区域的背景, 默认为浅灰色渐变.

\item {} 
\sphinxAtStartPar
linkcolor (CSS颜色):正文链接颜色.

\item {} 
\sphinxAtStartPar
headercolor1, headercolor2 (CSS颜色):<h1>和<h2>标题的颜色.

\item {} 
\sphinxAtStartPar
headerlinkcolor (CSS颜色):标题中后向引用链接的颜色.

\item {} 
\sphinxAtStartPar
textalign (CSS text\sphinxhyphen{}align 值):正文的文本对齐方式, 默认为 justify.

\end{itemize}


\subsubsection{nature}
\label{\detokenize{sphinx_conf:nature}}
\sphinxAtStartPar
一个绿色的主题。可配置nosidebar, sidebarwidth


\subsubsection{pyramid}
\label{\detokenize{sphinx_conf:pyramid}}
\sphinxAtStartPar
由Blaise Laflamme设计的金字塔网络框架项目的主题。

\sphinxAtStartPar
可配置nosidebar, sidebarwidth


\subsubsection{haiku}
\label{\detokenize{sphinx_conf:haiku}}
\sphinxAtStartPar
没有侧栏的主题
\begin{itemize}
\item {} 
\sphinxAtStartPar
full\_logo (true 或 false, 默认为 False):如果True, 标题只会显示 html\_logo. 用于大型徽标。 如果为False, 则徽标(如果存在)将浮动右侧显示, 文档标题将放在标题中.

\item {} 
\sphinxAtStartPar
textcolor, headingcolor, linkcolor, visitedlinkcolor, hoverlinkcolor (CSS颜色):各种身体元素的颜色.

\end{itemize}


\subsubsection{traditional}
\label{\detokenize{sphinx_conf:traditional}}
\sphinxAtStartPar
一个类似于旧Python文档的主题。可配置nosidebar, sidebarwidth


\subsubsection{epub}
\label{\detokenize{sphinx_conf:id17}}
\sphinxAtStartPar
epub构建器的主题。 这个主题试图保留视觉空间, 这是电子书阅读器上的稀疏资源。
\begin{itemize}
\item {} 
\sphinxAtStartPar
relbar1 (true 或 false, 默认为 True): 如果为true, 则将 relbar1 块插入epub输出中

\item {} 
\sphinxAtStartPar
footer (true 或 false, 默认为 True):如果为true, 则在脚本输出中插入 footer 块

\end{itemize}


\subsubsection{bizstyle}
\label{\detokenize{sphinx_conf:bizstyle}}
\sphinxAtStartPar
一个简单的蓝色主题。 可配置nosidebar, sidebarwidth, rightsidebar


\subsection{第三方主题}
\label{\detokenize{sphinx_conf:id18}}
\sphinxAtStartPar
有许多第三方主题可用。其中一些是一般用途, 而另一些则是针对单个项目的。

\sphinxAtStartPar
可在 \sphinxhref{https://pypi.org/search/?q=\&o=\&c=Framework+\%3A\%3A+Sphinx+\%3A\%3A+Theme}{PyPI} ,
\sphinxhref{https://github.com/search?utf8=\%E2\%9C\%93\&q=sphinx+theme\&type=}{GitHub}
和 \sphinxhref{https://sphinx-themes.org/}{sphinx\sphinxhyphen{}themes.org} 上可以找到更多第三方主题。


\subsubsection{sphinx\_rtd\_theme}
\label{\detokenize{sphinx_conf:sphinx-rtd-theme}}
\sphinxAtStartPar
\sphinxhref{https://pypi.org/project/sphinx\_rtd\_theme/}{Read the Docs Sphinx Theme}.

\sphinxAtStartPar
这是一个针对readthedocs.org制作的适合移动设备的sphinx主题.


\section{Markdown支持}
\label{\detokenize{sphinx_conf:markdown}}
\sphinxAtStartPar
为了支持基于Markdown的文档,Sphinx使用 \sphinxhref{https://recommonmark.readthedocs.io/en/latest/index.html}{recommonmark} 。

\sphinxAtStartPar
recommonmark是一个Docutils桥接器,是用于解析 \sphinxhref{https://commonmark.org/}{CommonMark} Markdown风格的Python包。
\begin{enumerate}
\sphinxsetlistlabels{\arabic}{enumi}{enumii}{}{.}%
\item {} 
\sphinxAtStartPar
安装Markdown解析器 recommonmark

\end{enumerate}

\begin{sphinxVerbatim}[commandchars=\\\{\}]
\PYG{n}{pip} \PYG{n}{install} \PYG{o}{\PYGZhy{}}\PYG{o}{\PYGZhy{}}\PYG{n}{upgrade} \PYG{n}{recommonmark}
\end{sphinxVerbatim}
\begin{enumerate}
\sphinxsetlistlabels{\arabic}{enumi}{enumii}{}{.}%
\setcounter{enumi}{1}
\item {} 
\sphinxAtStartPar
将 recommonmark 添加到扩展名列表

\end{enumerate}

\begin{sphinxVerbatim}[commandchars=\\\{\}]
\PYG{n}{extensions} \PYG{o}{=} \PYG{p}{[}\PYG{l+s+s1}{\PYGZsq{}}\PYG{l+s+s1}{recommonmark}\PYG{l+s+s1}{\PYGZsq{}}\PYG{p}{]}
\end{sphinxVerbatim}
\begin{enumerate}
\sphinxsetlistlabels{\arabic}{enumi}{enumii}{}{.}%
\setcounter{enumi}{2}
\item {} 
\sphinxAtStartPar
调整source\_suffix变量。

\end{enumerate}

\sphinxAtStartPar
下面的示例配置Sphinx将所有扩展名为 .md 和 .txt 的文件解析为 Markdown

\begin{sphinxVerbatim}[commandchars=\\\{\}]
    \PYG{n}{source\PYGZus{}suffix} \PYG{o}{=} \PYG{p}{\PYGZob{}}
\PYG{l+s+s1}{\PYGZsq{}}\PYG{l+s+s1}{.rst}\PYG{l+s+s1}{\PYGZsq{}}\PYG{p}{:} \PYG{l+s+s1}{\PYGZsq{}}\PYG{l+s+s1}{restructuredtext}\PYG{l+s+s1}{\PYGZsq{}}\PYG{p}{,}
\PYG{l+s+s1}{\PYGZsq{}}\PYG{l+s+s1}{.txt}\PYG{l+s+s1}{\PYGZsq{}}\PYG{p}{:} \PYG{l+s+s1}{\PYGZsq{}}\PYG{l+s+s1}{markdown}\PYG{l+s+s1}{\PYGZsq{}}\PYG{p}{,}
\PYG{l+s+s1}{\PYGZsq{}}\PYG{l+s+s1}{.md}\PYG{l+s+s1}{\PYGZsq{}}\PYG{p}{:} \PYG{l+s+s1}{\PYGZsq{}}\PYG{l+s+s1}{markdown}\PYG{l+s+s1}{\PYGZsq{}}\PYG{p}{,}\PYG{p}{\PYGZcb{}}
\end{sphinxVerbatim}
\begin{enumerate}
\sphinxsetlistlabels{\arabic}{enumi}{enumii}{}{.}%
\setcounter{enumi}{3}
\item {} 
\sphinxAtStartPar
进一步配置 recommonmark 以允许标准 CommonMark 不支持的自定义语法

\end{enumerate}

\sphinxAtStartPar
详阅 \sphinxhref{https://recommonmark.readthedocs.io/en/latest/auto\_structify.html}{recommonmark documentation}


\section{HTML模板}
\label{\detokenize{sphinx_conf:id19}}

\section{sphinx扩展ext}
\label{\detokenize{sphinx_conf:sphinxext}}

\subsection{内置扩展}
\label{\detokenize{sphinx_conf:id20}}

\subsection{外部扩展}
\label{\detokenize{sphinx_conf:id21}}

\section{配置文件示例}
\label{\detokenize{sphinx_conf:id22}}
\begin{sphinxVerbatim}[commandchars=\\\{\}]
\PYG{c+c1}{\PYGZsh{} \PYGZhy{}*\PYGZhy{} coding: utf\PYGZhy{}8 \PYGZhy{}*\PYGZhy{}}
\PYG{c+c1}{\PYGZsh{} test documentation build configuration file, created by}
\PYG{c+c1}{\PYGZsh{} sphinx\PYGZhy{}quickstart on Sun Jun 26 00:00:43 2016.}
\PYG{c+c1}{\PYGZsh{}}
\PYG{c+c1}{\PYGZsh{} This file is execfile()d with the current directory set to its}
\PYG{c+c1}{\PYGZsh{} containing dir.}
\PYG{c+c1}{\PYGZsh{}}
\PYG{c+c1}{\PYGZsh{} Note that not all possible configuration values are present in this}
\PYG{c+c1}{\PYGZsh{} autogenerated file.}
\PYG{c+c1}{\PYGZsh{}}
\PYG{c+c1}{\PYGZsh{} All configuration values have a default; values that are commented out}
\PYG{c+c1}{\PYGZsh{} serve to show the default.}

\PYG{c+c1}{\PYGZsh{} If extensions (or modules to document with autodoc) are in another directory,}
\PYG{c+c1}{\PYGZsh{} add these directories to sys.path here. If the directory is relative to the}
\PYG{c+c1}{\PYGZsh{} documentation root, use os.path.abspath to make it absolute, like shown here.}
\PYG{c+c1}{\PYGZsh{}}
\PYG{c+c1}{\PYGZsh{} import os}
\PYG{c+c1}{\PYGZsh{} import sys}
\PYG{c+c1}{\PYGZsh{} sys.path.insert(0, os.path.abspath(\PYGZsq{}.\PYGZsq{}))}

\PYG{c+c1}{\PYGZsh{} \PYGZhy{}\PYGZhy{} General configuration \PYGZhy{}\PYGZhy{}\PYGZhy{}\PYGZhy{}\PYGZhy{}\PYGZhy{}\PYGZhy{}\PYGZhy{}\PYGZhy{}\PYGZhy{}\PYGZhy{}\PYGZhy{}\PYGZhy{}\PYGZhy{}\PYGZhy{}\PYGZhy{}\PYGZhy{}\PYGZhy{}\PYGZhy{}\PYGZhy{}\PYGZhy{}\PYGZhy{}\PYGZhy{}\PYGZhy{}\PYGZhy{}\PYGZhy{}\PYGZhy{}\PYGZhy{}\PYGZhy{}\PYGZhy{}\PYGZhy{}\PYGZhy{}\PYGZhy{}\PYGZhy{}\PYGZhy{}\PYGZhy{}\PYGZhy{}\PYGZhy{}\PYGZhy{}\PYGZhy{}\PYGZhy{}\PYGZhy{}\PYGZhy{}\PYGZhy{}\PYGZhy{}\PYGZhy{}\PYGZhy{}\PYGZhy{}}

\PYG{c+c1}{\PYGZsh{} If your documentation needs a minimal Sphinx version, state it here.}
\PYG{c+c1}{\PYGZsh{}}
\PYG{c+c1}{\PYGZsh{} needs\PYGZus{}sphinx = \PYGZsq{}1.0\PYGZsq{}}

\PYG{c+c1}{\PYGZsh{} Add any Sphinx extension module names here, as strings. They can be}
\PYG{c+c1}{\PYGZsh{} extensions coming with Sphinx (named \PYGZsq{}sphinx.ext.*\PYGZsq{}) or your custom}
\PYG{c+c1}{\PYGZsh{} ones.}
\PYG{n}{extensions} \PYG{o}{=} \PYG{p}{[}\PYG{p}{]}

\PYG{c+c1}{\PYGZsh{} Add any paths that contain templates here, relative to this directory.}
\PYG{n}{templates\PYGZus{}path} \PYG{o}{=} \PYG{p}{[}\PYG{l+s+s1}{\PYGZsq{}}\PYG{l+s+s1}{\PYGZus{}templates}\PYG{l+s+s1}{\PYGZsq{}}\PYG{p}{]}

\PYG{c+c1}{\PYGZsh{} The suffix(es) of source filenames.}
\PYG{c+c1}{\PYGZsh{} You can specify multiple suffix as a list of string:}
\PYG{c+c1}{\PYGZsh{}}
\PYG{c+c1}{\PYGZsh{} source\PYGZus{}suffix = [\PYGZsq{}.rst\PYGZsq{}, \PYGZsq{}.md\PYGZsq{}]}
\PYG{n}{source\PYGZus{}suffix} \PYG{o}{=} \PYG{l+s+s1}{\PYGZsq{}}\PYG{l+s+s1}{.rst}\PYG{l+s+s1}{\PYGZsq{}}

\PYG{c+c1}{\PYGZsh{} The encoding of source files.}
\PYG{c+c1}{\PYGZsh{}}
\PYG{c+c1}{\PYGZsh{} source\PYGZus{}encoding = \PYGZsq{}utf\PYGZhy{}8\PYGZhy{}sig\PYGZsq{}}

\PYG{c+c1}{\PYGZsh{} The master toctree document.}
\PYG{n}{master\PYGZus{}doc} \PYG{o}{=} \PYG{l+s+s1}{\PYGZsq{}}\PYG{l+s+s1}{index}\PYG{l+s+s1}{\PYGZsq{}}

\PYG{c+c1}{\PYGZsh{} General information about the project.}
\PYG{n}{project} \PYG{o}{=} \PYG{l+s+sa}{u}\PYG{l+s+s1}{\PYGZsq{}}\PYG{l+s+s1}{test}\PYG{l+s+s1}{\PYGZsq{}}
\PYG{n}{copyright} \PYG{o}{=} \PYG{l+s+sa}{u}\PYG{l+s+s1}{\PYGZsq{}}\PYG{l+s+s1}{2016, test}\PYG{l+s+s1}{\PYGZsq{}}
\PYG{n}{author} \PYG{o}{=} \PYG{l+s+sa}{u}\PYG{l+s+s1}{\PYGZsq{}}\PYG{l+s+s1}{test}\PYG{l+s+s1}{\PYGZsq{}}

\PYG{c+c1}{\PYGZsh{} The version info for the project you\PYGZsq{}re documenting, acts as replacement for}
\PYG{c+c1}{\PYGZsh{} |version| and |release|, also used in various other places throughout the}
\PYG{c+c1}{\PYGZsh{} built documents.}
\PYG{c+c1}{\PYGZsh{}}
\PYG{c+c1}{\PYGZsh{} The short X.Y version.}
\PYG{n}{version} \PYG{o}{=} \PYG{l+s+sa}{u}\PYG{l+s+s1}{\PYGZsq{}}\PYG{l+s+s1}{test}\PYG{l+s+s1}{\PYGZsq{}}
\PYG{c+c1}{\PYGZsh{} The full version, including alpha/beta/rc tags.}
\PYG{n}{release} \PYG{o}{=} \PYG{l+s+sa}{u}\PYG{l+s+s1}{\PYGZsq{}}\PYG{l+s+s1}{test}\PYG{l+s+s1}{\PYGZsq{}}

\PYG{c+c1}{\PYGZsh{} The language for content autogenerated by Sphinx. Refer to documentation}
\PYG{c+c1}{\PYGZsh{} for a list of supported languages.}
\PYG{c+c1}{\PYGZsh{}}
\PYG{c+c1}{\PYGZsh{} This is also used if you do content translation via gettext catalogs.}
\PYG{c+c1}{\PYGZsh{} Usually you set \PYGZdq{}language\PYGZdq{} from the command line for these cases.}
\PYG{n}{language} \PYG{o}{=} \PYG{k+kc}{None}

\PYG{c+c1}{\PYGZsh{} There are two options for replacing |today|: either, you set today to some}
\PYG{c+c1}{\PYGZsh{} non\PYGZhy{}false value, then it is used:}
\PYG{c+c1}{\PYGZsh{}}
\PYG{c+c1}{\PYGZsh{} today = \PYGZsq{}\PYGZsq{}}
\PYG{c+c1}{\PYGZsh{}}
\PYG{c+c1}{\PYGZsh{} Else, today\PYGZus{}fmt is used as the format for a strftime call.}
\PYG{c+c1}{\PYGZsh{}}
\PYG{c+c1}{\PYGZsh{} today\PYGZus{}fmt = \PYGZsq{}\PYGZpc{}B \PYGZpc{}d, \PYGZpc{}Y\PYGZsq{}}

\PYG{c+c1}{\PYGZsh{} List of patterns, relative to source directory, that match files and}
\PYG{c+c1}{\PYGZsh{} directories to ignore when looking for source files.}
\PYG{c+c1}{\PYGZsh{} These patterns also affect html\PYGZus{}static\PYGZus{}path and html\PYGZus{}extra\PYGZus{}path}
\PYG{n}{exclude\PYGZus{}patterns} \PYG{o}{=} \PYG{p}{[}\PYG{l+s+s1}{\PYGZsq{}}\PYG{l+s+s1}{\PYGZus{}build}\PYG{l+s+s1}{\PYGZsq{}}\PYG{p}{,} \PYG{l+s+s1}{\PYGZsq{}}\PYG{l+s+s1}{Thumbs.db}\PYG{l+s+s1}{\PYGZsq{}}\PYG{p}{,} \PYG{l+s+s1}{\PYGZsq{}}\PYG{l+s+s1}{.DS\PYGZus{}Store}\PYG{l+s+s1}{\PYGZsq{}}\PYG{p}{]}

\PYG{c+c1}{\PYGZsh{} The reST default role (used for this markup: `text`) to use for all}
\PYG{c+c1}{\PYGZsh{} documents.}
\PYG{c+c1}{\PYGZsh{}}
\PYG{c+c1}{\PYGZsh{} default\PYGZus{}role = None}

\PYG{c+c1}{\PYGZsh{} If true, \PYGZsq{}()\PYGZsq{} will be appended to :func: etc. cross\PYGZhy{}reference text.}
\PYG{c+c1}{\PYGZsh{}}
\PYG{c+c1}{\PYGZsh{} add\PYGZus{}function\PYGZus{}parentheses = True}

\PYG{c+c1}{\PYGZsh{} If true, the current module name will be prepended to all description}
\PYG{c+c1}{\PYGZsh{} unit titles (such as .. function::).}
\PYG{c+c1}{\PYGZsh{}}
\PYG{c+c1}{\PYGZsh{} add\PYGZus{}module\PYGZus{}names = True}

\PYG{c+c1}{\PYGZsh{} If true, sectionauthor and moduleauthor directives will be shown in the}
\PYG{c+c1}{\PYGZsh{} output. They are ignored by default.}
\PYG{c+c1}{\PYGZsh{}}
\PYG{c+c1}{\PYGZsh{} show\PYGZus{}authors = False}

\PYG{c+c1}{\PYGZsh{} The name of the Pygments (syntax highlighting) style to use.}
\PYG{n}{pygments\PYGZus{}style} \PYG{o}{=} \PYG{l+s+s1}{\PYGZsq{}}\PYG{l+s+s1}{sphinx}\PYG{l+s+s1}{\PYGZsq{}}

\PYG{c+c1}{\PYGZsh{} A list of ignored prefixes for module index sorting.}
\PYG{c+c1}{\PYGZsh{} modindex\PYGZus{}common\PYGZus{}prefix = []}

\PYG{c+c1}{\PYGZsh{} If true, keep warnings as \PYGZdq{}system message\PYGZdq{} paragraphs in the built documents.}
\PYG{c+c1}{\PYGZsh{} keep\PYGZus{}warnings = False}

\PYG{c+c1}{\PYGZsh{} If true, `todo` and `todoList` produce output, else they produce nothing.}
\PYG{n}{todo\PYGZus{}include\PYGZus{}todos} \PYG{o}{=} \PYG{k+kc}{False}


\PYG{c+c1}{\PYGZsh{} \PYGZhy{}\PYGZhy{} Options for HTML output \PYGZhy{}\PYGZhy{}\PYGZhy{}\PYGZhy{}\PYGZhy{}\PYGZhy{}\PYGZhy{}\PYGZhy{}\PYGZhy{}\PYGZhy{}\PYGZhy{}\PYGZhy{}\PYGZhy{}\PYGZhy{}\PYGZhy{}\PYGZhy{}\PYGZhy{}\PYGZhy{}\PYGZhy{}\PYGZhy{}\PYGZhy{}\PYGZhy{}\PYGZhy{}\PYGZhy{}\PYGZhy{}\PYGZhy{}\PYGZhy{}\PYGZhy{}\PYGZhy{}\PYGZhy{}\PYGZhy{}\PYGZhy{}\PYGZhy{}\PYGZhy{}\PYGZhy{}\PYGZhy{}\PYGZhy{}\PYGZhy{}\PYGZhy{}\PYGZhy{}\PYGZhy{}\PYGZhy{}\PYGZhy{}\PYGZhy{}\PYGZhy{}\PYGZhy{}}

\PYG{c+c1}{\PYGZsh{} The theme to use for HTML and HTML Help pages.  See the documentation for}
\PYG{c+c1}{\PYGZsh{} a list of builtin themes.}
\PYG{c+c1}{\PYGZsh{}}
\PYG{n}{html\PYGZus{}theme} \PYG{o}{=} \PYG{l+s+s1}{\PYGZsq{}}\PYG{l+s+s1}{alabaster}\PYG{l+s+s1}{\PYGZsq{}}

\PYG{c+c1}{\PYGZsh{} Theme options are theme\PYGZhy{}specific and customize the look and feel of a theme}
\PYG{c+c1}{\PYGZsh{} further.  For a list of options available for each theme, see the}
\PYG{c+c1}{\PYGZsh{} documentation.}
\PYG{c+c1}{\PYGZsh{}}
\PYG{c+c1}{\PYGZsh{} html\PYGZus{}theme\PYGZus{}options = \PYGZob{}\PYGZcb{}}

\PYG{c+c1}{\PYGZsh{} Add any paths that contain custom themes here, relative to this directory.}
\PYG{c+c1}{\PYGZsh{} html\PYGZus{}theme\PYGZus{}path = []}

\PYG{c+c1}{\PYGZsh{} The name for this set of Sphinx documents.}
\PYG{c+c1}{\PYGZsh{} \PYGZdq{}\PYGZlt{}project\PYGZgt{} v\PYGZlt{}release\PYGZgt{} documentation\PYGZdq{} by default.}
\PYG{c+c1}{\PYGZsh{}}
\PYG{c+c1}{\PYGZsh{} html\PYGZus{}title = u\PYGZsq{}test vtest\PYGZsq{}}

\PYG{c+c1}{\PYGZsh{} A shorter title for the navigation bar.  Default is the same as html\PYGZus{}title.}
\PYG{c+c1}{\PYGZsh{}}
\PYG{c+c1}{\PYGZsh{} html\PYGZus{}short\PYGZus{}title = None}

\PYG{c+c1}{\PYGZsh{} The name of an image file (relative to this directory) to place at the top}
\PYG{c+c1}{\PYGZsh{} of the sidebar.}
\PYG{c+c1}{\PYGZsh{}}
\PYG{c+c1}{\PYGZsh{} html\PYGZus{}logo = None}

\PYG{c+c1}{\PYGZsh{} The name of an image file (relative to this directory) to use as a favicon of}
\PYG{c+c1}{\PYGZsh{} the docs.  This file should be a Windows icon file (.ico) being 16x16 or 32x32}
\PYG{c+c1}{\PYGZsh{} pixels large.}
\PYG{c+c1}{\PYGZsh{}}
\PYG{c+c1}{\PYGZsh{} html\PYGZus{}favicon = None}

\PYG{c+c1}{\PYGZsh{} Add any paths that contain custom static files (such as style sheets) here,}
\PYG{c+c1}{\PYGZsh{} relative to this directory. They are copied after the builtin static files,}
\PYG{c+c1}{\PYGZsh{} so a file named \PYGZdq{}default.css\PYGZdq{} will overwrite the builtin \PYGZdq{}default.css\PYGZdq{}.}
\PYG{n}{html\PYGZus{}static\PYGZus{}path} \PYG{o}{=} \PYG{p}{[}\PYG{l+s+s1}{\PYGZsq{}}\PYG{l+s+s1}{\PYGZus{}static}\PYG{l+s+s1}{\PYGZsq{}}\PYG{p}{]}

\PYG{c+c1}{\PYGZsh{} Add any extra paths that contain custom files (such as robots.txt or}
\PYG{c+c1}{\PYGZsh{} .htaccess) here, relative to this directory. These files are copied}
\PYG{c+c1}{\PYGZsh{} directly to the root of the documentation.}
\PYG{c+c1}{\PYGZsh{}}
\PYG{c+c1}{\PYGZsh{} html\PYGZus{}extra\PYGZus{}path = []}

\PYG{c+c1}{\PYGZsh{} If not None, a \PYGZsq{}Last updated on:\PYGZsq{} timestamp is inserted at every page}
\PYG{c+c1}{\PYGZsh{} bottom, using the given strftime format.}
\PYG{c+c1}{\PYGZsh{} The empty string is equivalent to \PYGZsq{}\PYGZpc{}b \PYGZpc{}d, \PYGZpc{}Y\PYGZsq{}.}
\PYG{c+c1}{\PYGZsh{}}
\PYG{c+c1}{\PYGZsh{} html\PYGZus{}last\PYGZus{}updated\PYGZus{}fmt = None}

\PYG{c+c1}{\PYGZsh{} Custom sidebar templates, maps document names to template names.}
\PYG{c+c1}{\PYGZsh{}}
\PYG{c+c1}{\PYGZsh{} html\PYGZus{}sidebars = \PYGZob{}\PYGZcb{}}

\PYG{c+c1}{\PYGZsh{} Additional templates that should be rendered to pages, maps page names to}
\PYG{c+c1}{\PYGZsh{} template names.}
\PYG{c+c1}{\PYGZsh{}}
\PYG{c+c1}{\PYGZsh{} html\PYGZus{}additional\PYGZus{}pages = \PYGZob{}\PYGZcb{}}

\PYG{c+c1}{\PYGZsh{} If false, no module index is generated.}
\PYG{c+c1}{\PYGZsh{}}
\PYG{c+c1}{\PYGZsh{} html\PYGZus{}domain\PYGZus{}indices = True}

\PYG{c+c1}{\PYGZsh{} If false, no index is generated.}
\PYG{c+c1}{\PYGZsh{}}
\PYG{c+c1}{\PYGZsh{} html\PYGZus{}use\PYGZus{}index = True}

\PYG{c+c1}{\PYGZsh{} If true, the index is split into individual pages for each letter.}
\PYG{c+c1}{\PYGZsh{}}
\PYG{c+c1}{\PYGZsh{} html\PYGZus{}split\PYGZus{}index = False}

\PYG{c+c1}{\PYGZsh{} If true, links to the reST sources are added to the pages.}
\PYG{c+c1}{\PYGZsh{}}
\PYG{c+c1}{\PYGZsh{} html\PYGZus{}show\PYGZus{}sourcelink = True}

\PYG{c+c1}{\PYGZsh{} If true, \PYGZdq{}Created using Sphinx\PYGZdq{} is shown in the HTML footer. Default is True.}
\PYG{c+c1}{\PYGZsh{}}
\PYG{c+c1}{\PYGZsh{} html\PYGZus{}show\PYGZus{}sphinx = True}

\PYG{c+c1}{\PYGZsh{} If true, \PYGZdq{}(C) Copyright ...\PYGZdq{} is shown in the HTML footer. Default is True.}
\PYG{c+c1}{\PYGZsh{}}
\PYG{c+c1}{\PYGZsh{} html\PYGZus{}show\PYGZus{}copyright = True}

\PYG{c+c1}{\PYGZsh{} If true, an OpenSearch description file will be output, and all pages will}
\PYG{c+c1}{\PYGZsh{} contain a \PYGZlt{}link\PYGZgt{} tag referring to it.  The value of this option must be the}
\PYG{c+c1}{\PYGZsh{} base URL from which the finished HTML is served.}
\PYG{c+c1}{\PYGZsh{}}
\PYG{c+c1}{\PYGZsh{} html\PYGZus{}use\PYGZus{}opensearch = \PYGZsq{}\PYGZsq{}}

\PYG{c+c1}{\PYGZsh{} This is the file name suffix for HTML files (e.g. \PYGZdq{}.xhtml\PYGZdq{}).}
\PYG{c+c1}{\PYGZsh{} html\PYGZus{}file\PYGZus{}suffix = None}

\PYG{c+c1}{\PYGZsh{} Language to be used for generating the HTML full\PYGZhy{}text search index.}
\PYG{c+c1}{\PYGZsh{} Sphinx supports the following languages:}
\PYG{c+c1}{\PYGZsh{}   \PYGZsq{}da\PYGZsq{}, \PYGZsq{}de\PYGZsq{}, \PYGZsq{}en\PYGZsq{}, \PYGZsq{}es\PYGZsq{}, \PYGZsq{}fi\PYGZsq{}, \PYGZsq{}fr\PYGZsq{}, \PYGZsq{}hu\PYGZsq{}, \PYGZsq{}it\PYGZsq{}, \PYGZsq{}ja\PYGZsq{}}
\PYG{c+c1}{\PYGZsh{}   \PYGZsq{}nl\PYGZsq{}, \PYGZsq{}no\PYGZsq{}, \PYGZsq{}pt\PYGZsq{}, \PYGZsq{}ro\PYGZsq{}, \PYGZsq{}ru\PYGZsq{}, \PYGZsq{}sv\PYGZsq{}, \PYGZsq{}tr\PYGZsq{}, \PYGZsq{}zh\PYGZsq{}}
\PYG{c+c1}{\PYGZsh{}}
\PYG{c+c1}{\PYGZsh{} html\PYGZus{}search\PYGZus{}language = \PYGZsq{}en\PYGZsq{}}

\PYG{c+c1}{\PYGZsh{} A dictionary with options for the search language support, empty by default.}
\PYG{c+c1}{\PYGZsh{} \PYGZsq{}ja\PYGZsq{} uses this config value.}
\PYG{c+c1}{\PYGZsh{} \PYGZsq{}zh\PYGZsq{} user can custom change `jieba` dictionary path.}
\PYG{c+c1}{\PYGZsh{}}
\PYG{c+c1}{\PYGZsh{} html\PYGZus{}search\PYGZus{}options = \PYGZob{}\PYGZsq{}type\PYGZsq{}: \PYGZsq{}default\PYGZsq{}\PYGZcb{}}

\PYG{c+c1}{\PYGZsh{} The name of a javascript file (relative to the configuration directory) that}
\PYG{c+c1}{\PYGZsh{} implements a search results scorer. If empty, the default will be used.}
\PYG{c+c1}{\PYGZsh{}}
\PYG{c+c1}{\PYGZsh{} html\PYGZus{}search\PYGZus{}scorer = \PYGZsq{}scorer.js\PYGZsq{}}

\PYG{c+c1}{\PYGZsh{} Output file base name for HTML help builder.}
\PYG{n}{htmlhelp\PYGZus{}basename} \PYG{o}{=} \PYG{l+s+s1}{\PYGZsq{}}\PYG{l+s+s1}{testdoc}\PYG{l+s+s1}{\PYGZsq{}}

\PYG{c+c1}{\PYGZsh{} \PYGZhy{}\PYGZhy{} Options for LaTeX output \PYGZhy{}\PYGZhy{}\PYGZhy{}\PYGZhy{}\PYGZhy{}\PYGZhy{}\PYGZhy{}\PYGZhy{}\PYGZhy{}\PYGZhy{}\PYGZhy{}\PYGZhy{}\PYGZhy{}\PYGZhy{}\PYGZhy{}\PYGZhy{}\PYGZhy{}\PYGZhy{}\PYGZhy{}\PYGZhy{}\PYGZhy{}\PYGZhy{}\PYGZhy{}\PYGZhy{}\PYGZhy{}\PYGZhy{}\PYGZhy{}\PYGZhy{}\PYGZhy{}\PYGZhy{}\PYGZhy{}\PYGZhy{}\PYGZhy{}\PYGZhy{}\PYGZhy{}\PYGZhy{}\PYGZhy{}\PYGZhy{}\PYGZhy{}\PYGZhy{}\PYGZhy{}\PYGZhy{}\PYGZhy{}\PYGZhy{}\PYGZhy{}}

\PYG{n}{latex\PYGZus{}elements} \PYG{o}{=} \PYG{p}{\PYGZob{}}
    \PYG{c+c1}{\PYGZsh{} The paper size (\PYGZsq{}letterpaper\PYGZsq{} or \PYGZsq{}a4paper\PYGZsq{}).}
    \PYG{c+c1}{\PYGZsh{}}
    \PYG{c+c1}{\PYGZsh{} \PYGZsq{}papersize\PYGZsq{}: \PYGZsq{}letterpaper\PYGZsq{},}

    \PYG{c+c1}{\PYGZsh{} The font size (\PYGZsq{}10pt\PYGZsq{}, \PYGZsq{}11pt\PYGZsq{} or \PYGZsq{}12pt\PYGZsq{}).}
    \PYG{c+c1}{\PYGZsh{}}
    \PYG{c+c1}{\PYGZsh{} \PYGZsq{}pointsize\PYGZsq{}: \PYGZsq{}10pt\PYGZsq{},}

    \PYG{c+c1}{\PYGZsh{} Additional stuff for the LaTeX preamble.}
    \PYG{c+c1}{\PYGZsh{}}
    \PYG{c+c1}{\PYGZsh{} \PYGZsq{}preamble\PYGZsq{}: \PYGZsq{}\PYGZsq{},}

    \PYG{c+c1}{\PYGZsh{} Latex figure (float) alignment}
    \PYG{c+c1}{\PYGZsh{}}
    \PYG{c+c1}{\PYGZsh{} \PYGZsq{}figure\PYGZus{}align\PYGZsq{}: \PYGZsq{}htbp\PYGZsq{},}
\PYG{p}{\PYGZcb{}}

\PYG{c+c1}{\PYGZsh{} Grouping the document tree into LaTeX files. List of tuples}
\PYG{c+c1}{\PYGZsh{} (source start file, target name, title,}
\PYG{c+c1}{\PYGZsh{}  author, documentclass [howto, manual, or own class]).}
\PYG{n}{latex\PYGZus{}documents} \PYG{o}{=} \PYG{p}{[}
    \PYG{p}{(}\PYG{n}{master\PYGZus{}doc}\PYG{p}{,} \PYG{l+s+s1}{\PYGZsq{}}\PYG{l+s+s1}{test.tex}\PYG{l+s+s1}{\PYGZsq{}}\PYG{p}{,} \PYG{l+s+sa}{u}\PYG{l+s+s1}{\PYGZsq{}}\PYG{l+s+s1}{test Documentation}\PYG{l+s+s1}{\PYGZsq{}}\PYG{p}{,}
     \PYG{l+s+sa}{u}\PYG{l+s+s1}{\PYGZsq{}}\PYG{l+s+s1}{test}\PYG{l+s+s1}{\PYGZsq{}}\PYG{p}{,} \PYG{l+s+s1}{\PYGZsq{}}\PYG{l+s+s1}{manual}\PYG{l+s+s1}{\PYGZsq{}}\PYG{p}{)}\PYG{p}{,}
\PYG{p}{]}

\PYG{c+c1}{\PYGZsh{} The name of an image file (relative to this directory) to place at the top of}
\PYG{c+c1}{\PYGZsh{} the title page.}
\PYG{c+c1}{\PYGZsh{}}
\PYG{c+c1}{\PYGZsh{} latex\PYGZus{}logo = None}

\PYG{c+c1}{\PYGZsh{} If true, show page references after internal links.}
\PYG{c+c1}{\PYGZsh{}}
\PYG{c+c1}{\PYGZsh{} latex\PYGZus{}show\PYGZus{}pagerefs = False}

\PYG{c+c1}{\PYGZsh{} If true, show URL addresses after external links.}
\PYG{c+c1}{\PYGZsh{}}
\PYG{c+c1}{\PYGZsh{} latex\PYGZus{}show\PYGZus{}urls = False}

\PYG{c+c1}{\PYGZsh{} Documents to append as an appendix to all manuals.}
\PYG{c+c1}{\PYGZsh{}}
\PYG{c+c1}{\PYGZsh{} latex\PYGZus{}appendices = []}

\PYG{c+c1}{\PYGZsh{} If false, no module index is generated.}
\PYG{c+c1}{\PYGZsh{}}
\PYG{c+c1}{\PYGZsh{} latex\PYGZus{}domain\PYGZus{}indices = True}


\PYG{c+c1}{\PYGZsh{} \PYGZhy{}\PYGZhy{} Options for manual page output \PYGZhy{}\PYGZhy{}\PYGZhy{}\PYGZhy{}\PYGZhy{}\PYGZhy{}\PYGZhy{}\PYGZhy{}\PYGZhy{}\PYGZhy{}\PYGZhy{}\PYGZhy{}\PYGZhy{}\PYGZhy{}\PYGZhy{}\PYGZhy{}\PYGZhy{}\PYGZhy{}\PYGZhy{}\PYGZhy{}\PYGZhy{}\PYGZhy{}\PYGZhy{}\PYGZhy{}\PYGZhy{}\PYGZhy{}\PYGZhy{}\PYGZhy{}\PYGZhy{}\PYGZhy{}\PYGZhy{}\PYGZhy{}\PYGZhy{}\PYGZhy{}\PYGZhy{}\PYGZhy{}\PYGZhy{}\PYGZhy{}\PYGZhy{}}

\PYG{c+c1}{\PYGZsh{} One entry per manual page. List of tuples}
\PYG{c+c1}{\PYGZsh{} (source start file, name, description, authors, manual section).}
\PYG{n}{man\PYGZus{}pages} \PYG{o}{=} \PYG{p}{[}
    \PYG{p}{(}\PYG{n}{master\PYGZus{}doc}\PYG{p}{,} \PYG{l+s+s1}{\PYGZsq{}}\PYG{l+s+s1}{test}\PYG{l+s+s1}{\PYGZsq{}}\PYG{p}{,} \PYG{l+s+sa}{u}\PYG{l+s+s1}{\PYGZsq{}}\PYG{l+s+s1}{test Documentation}\PYG{l+s+s1}{\PYGZsq{}}\PYG{p}{,}
     \PYG{p}{[}\PYG{n}{author}\PYG{p}{]}\PYG{p}{,} \PYG{l+m+mi}{1}\PYG{p}{)}
\PYG{p}{]}

\PYG{c+c1}{\PYGZsh{} If true, show URL addresses after external links.}
\PYG{c+c1}{\PYGZsh{}}
\PYG{c+c1}{\PYGZsh{} man\PYGZus{}show\PYGZus{}urls = False}


\PYG{c+c1}{\PYGZsh{} \PYGZhy{}\PYGZhy{} Options for Texinfo output \PYGZhy{}\PYGZhy{}\PYGZhy{}\PYGZhy{}\PYGZhy{}\PYGZhy{}\PYGZhy{}\PYGZhy{}\PYGZhy{}\PYGZhy{}\PYGZhy{}\PYGZhy{}\PYGZhy{}\PYGZhy{}\PYGZhy{}\PYGZhy{}\PYGZhy{}\PYGZhy{}\PYGZhy{}\PYGZhy{}\PYGZhy{}\PYGZhy{}\PYGZhy{}\PYGZhy{}\PYGZhy{}\PYGZhy{}\PYGZhy{}\PYGZhy{}\PYGZhy{}\PYGZhy{}\PYGZhy{}\PYGZhy{}\PYGZhy{}\PYGZhy{}\PYGZhy{}\PYGZhy{}\PYGZhy{}\PYGZhy{}\PYGZhy{}\PYGZhy{}\PYGZhy{}\PYGZhy{}\PYGZhy{}}

\PYG{c+c1}{\PYGZsh{} Grouping the document tree into Texinfo files. List of tuples}
\PYG{c+c1}{\PYGZsh{} (source start file, target name, title, author,}
\PYG{c+c1}{\PYGZsh{}  dir menu entry, description, category)}
\PYG{n}{texinfo\PYGZus{}documents} \PYG{o}{=} \PYG{p}{[}
    \PYG{p}{(}\PYG{n}{master\PYGZus{}doc}\PYG{p}{,} \PYG{l+s+s1}{\PYGZsq{}}\PYG{l+s+s1}{test}\PYG{l+s+s1}{\PYGZsq{}}\PYG{p}{,} \PYG{l+s+sa}{u}\PYG{l+s+s1}{\PYGZsq{}}\PYG{l+s+s1}{test Documentation}\PYG{l+s+s1}{\PYGZsq{}}\PYG{p}{,}
     \PYG{n}{author}\PYG{p}{,} \PYG{l+s+s1}{\PYGZsq{}}\PYG{l+s+s1}{test}\PYG{l+s+s1}{\PYGZsq{}}\PYG{p}{,} \PYG{l+s+s1}{\PYGZsq{}}\PYG{l+s+s1}{One line description of project.}\PYG{l+s+s1}{\PYGZsq{}}\PYG{p}{,}
     \PYG{l+s+s1}{\PYGZsq{}}\PYG{l+s+s1}{Miscellaneous}\PYG{l+s+s1}{\PYGZsq{}}\PYG{p}{)}\PYG{p}{,}
\PYG{p}{]}

\PYG{c+c1}{\PYGZsh{} Documents to append as an appendix to all manuals.}
\PYG{c+c1}{\PYGZsh{}}
\PYG{c+c1}{\PYGZsh{} texinfo\PYGZus{}appendices = []}

\PYG{c+c1}{\PYGZsh{} If false, no module index is generated.}
\PYG{c+c1}{\PYGZsh{}}
\PYG{c+c1}{\PYGZsh{} texinfo\PYGZus{}domain\PYGZus{}indices = True}

\PYG{c+c1}{\PYGZsh{} How to display URL addresses: \PYGZsq{}footnote\PYGZsq{}, \PYGZsq{}no\PYGZsq{}, or \PYGZsq{}inline\PYGZsq{}.}
\PYG{c+c1}{\PYGZsh{}}
\PYG{c+c1}{\PYGZsh{} texinfo\PYGZus{}show\PYGZus{}urls = \PYGZsq{}footnote\PYGZsq{}}

\PYG{c+c1}{\PYGZsh{} If true, do not generate a @detailmenu in the \PYGZdq{}Top\PYGZdq{} node\PYGZsq{}s menu.}
\PYG{c+c1}{\PYGZsh{}}
\PYG{c+c1}{\PYGZsh{} texinfo\PYGZus{}no\PYGZus{}detailmenu = False}

\PYG{c+c1}{\PYGZsh{} \PYGZhy{}\PYGZhy{} A random example \PYGZhy{}\PYGZhy{}\PYGZhy{}\PYGZhy{}\PYGZhy{}\PYGZhy{}\PYGZhy{}\PYGZhy{}\PYGZhy{}\PYGZhy{}\PYGZhy{}\PYGZhy{}\PYGZhy{}\PYGZhy{}\PYGZhy{}\PYGZhy{}\PYGZhy{}\PYGZhy{}\PYGZhy{}\PYGZhy{}\PYGZhy{}\PYGZhy{}\PYGZhy{}\PYGZhy{}\PYGZhy{}\PYGZhy{}\PYGZhy{}\PYGZhy{}\PYGZhy{}\PYGZhy{}\PYGZhy{}\PYGZhy{}\PYGZhy{}\PYGZhy{}\PYGZhy{}\PYGZhy{}\PYGZhy{}\PYGZhy{}\PYGZhy{}\PYGZhy{}\PYGZhy{}\PYGZhy{}\PYGZhy{}\PYGZhy{}\PYGZhy{}\PYGZhy{}\PYGZhy{}\PYGZhy{}\PYGZhy{}\PYGZhy{}\PYGZhy{}\PYGZhy{}\PYGZhy{}}

\PYG{k+kn}{import} \PYG{n+nn}{sys}\PYG{o}{,} \PYG{n+nn}{os}
\PYG{n}{sys}\PYG{o}{.}\PYG{n}{path}\PYG{o}{.}\PYG{n}{insert}\PYG{p}{(}\PYG{l+m+mi}{0}\PYG{p}{,} \PYG{n}{os}\PYG{o}{.}\PYG{n}{path}\PYG{o}{.}\PYG{n}{abspath}\PYG{p}{(}\PYG{l+s+s1}{\PYGZsq{}}\PYG{l+s+s1}{.}\PYG{l+s+s1}{\PYGZsq{}}\PYG{p}{)}\PYG{p}{)}
\PYG{n}{exclude\PYGZus{}patterns} \PYG{o}{=} \PYG{p}{[}\PYG{l+s+s1}{\PYGZsq{}}\PYG{l+s+s1}{zzz}\PYG{l+s+s1}{\PYGZsq{}}\PYG{p}{]}

\PYG{n}{numfig} \PYG{o}{=} \PYG{k+kc}{True}
\PYG{c+c1}{\PYGZsh{}language = \PYGZsq{}ja\PYGZsq{}}

\PYG{n}{extensions}\PYG{o}{.}\PYG{n}{append}\PYG{p}{(}\PYG{l+s+s1}{\PYGZsq{}}\PYG{l+s+s1}{sphinx.ext.todo}\PYG{l+s+s1}{\PYGZsq{}}\PYG{p}{)}
\PYG{n}{extensions}\PYG{o}{.}\PYG{n}{append}\PYG{p}{(}\PYG{l+s+s1}{\PYGZsq{}}\PYG{l+s+s1}{sphinx.ext.autodoc}\PYG{l+s+s1}{\PYGZsq{}}\PYG{p}{)}
\PYG{c+c1}{\PYGZsh{}extensions.append(\PYGZsq{}sphinx.ext.autosummary\PYGZsq{})}
\PYG{n}{extensions}\PYG{o}{.}\PYG{n}{append}\PYG{p}{(}\PYG{l+s+s1}{\PYGZsq{}}\PYG{l+s+s1}{sphinx.ext.intersphinx}\PYG{l+s+s1}{\PYGZsq{}}\PYG{p}{)}
\PYG{n}{extensions}\PYG{o}{.}\PYG{n}{append}\PYG{p}{(}\PYG{l+s+s1}{\PYGZsq{}}\PYG{l+s+s1}{sphinx.ext.mathjax}\PYG{l+s+s1}{\PYGZsq{}}\PYG{p}{)}
\PYG{n}{extensions}\PYG{o}{.}\PYG{n}{append}\PYG{p}{(}\PYG{l+s+s1}{\PYGZsq{}}\PYG{l+s+s1}{sphinx.ext.viewcode}\PYG{l+s+s1}{\PYGZsq{}}\PYG{p}{)}
\PYG{n}{extensions}\PYG{o}{.}\PYG{n}{append}\PYG{p}{(}\PYG{l+s+s1}{\PYGZsq{}}\PYG{l+s+s1}{sphinx.ext.graphviz}\PYG{l+s+s1}{\PYGZsq{}}\PYG{p}{)}


\PYG{n}{autosummary\PYGZus{}generate} \PYG{o}{=} \PYG{k+kc}{True}
\PYG{n}{html\PYGZus{}theme} \PYG{o}{=} \PYG{l+s+s1}{\PYGZsq{}}\PYG{l+s+s1}{default}\PYG{l+s+s1}{\PYGZsq{}}
\PYG{c+c1}{\PYGZsh{}source\PYGZus{}suffix = [\PYGZsq{}.rst\PYGZsq{}, \PYGZsq{}.txt\PYGZsq{}]}
\end{sphinxVerbatim}

\sphinxstepscope


\chapter{reStructureText语法}
\label{\detokenize{reStructureText_syntax:restructuretext}}\label{\detokenize{reStructureText_syntax::doc}}

\section{内联标记 Inline Markup}
\label{\detokenize{reStructureText_syntax:inline-markup}}

\begin{savenotes}\sphinxattablestart
\sphinxthistablewithglobalstyle
\centering
\begin{tabulary}{\linewidth}[t]{TTT}
\sphinxtoprule
\sphinxstyletheadfamily 
\sphinxAtStartPar
语法
&\sphinxstyletheadfamily 
\sphinxAtStartPar
效果
&\sphinxstyletheadfamily 
\sphinxAtStartPar
说明
\\
\sphinxmidrule
\sphinxtableatstartofbodyhook
\sphinxAtStartPar
\sphinxcode{\sphinxupquote{*emphasis*}}
&
\sphinxAtStartPar
\sphinxstyleemphasis{emphasis}
&
\sphinxAtStartPar
强调,斜体
\\
\sphinxhline
\sphinxAtStartPar
\sphinxcode{\sphinxupquote{**strong**}}
&
\sphinxAtStartPar
\sphinxstylestrong{strong}
&
\sphinxAtStartPar
加粗
\\
\sphinxhline
\sphinxAtStartPar
`interpreted\sphinxhyphen{}text`
&
\sphinxAtStartPar
\sphinxtitleref{interpreted\sphinxhyphen{}text}
&
\sphinxAtStartPar
使用两个反逗号包裹内容,表征对其解释
\\
\sphinxhline
\sphinxAtStartPar
``inline literal``
&
\sphinxAtStartPar
\sphinxcode{\sphinxupquote{inline literal}}
&
\sphinxAtStartPar
当包括内容包含空格时使用两个反逗号来包裹
\\
\sphinxhline
\sphinxAtStartPar
\sphinxcode{\sphinxupquote{ref\_  .. \_ref: 链接}}
&
\sphinxAtStartPar
\sphinxhref{https://docutils.sourceforge.io/docs/user/rst/quickref.html\#hyperlink-targets}{ref}
&
\sphinxAtStartPar
链接:纯文本,外部链接
\\
\sphinxhline
\sphinxAtStartPar
`phrase ref`\_
&
\sphinxAtStartPar
{\hyperref[\detokenize{reStructureText_syntax:id36}]{\sphinxcrossref{带空格 外链}}}
&
\sphinxAtStartPar
链接:文字间有空格标点,外部链接
\\
\sphinxhline
\sphinxAtStartPar
\sphinxcode{\sphinxupquote{anonymous\_\_}}
&
\sphinxAtStartPar
anonymous
&
\sphinxAtStartPar
链接: 匿名链接
\\
\sphinxhline
\sphinxAtStartPar
\sphinxcode{\sphinxupquote{\_`inline\_link}}
&
\sphinxAtStartPar
inline\_link
&
\sphinxAtStartPar
交叉引用链接
\\
\sphinxhline
\sphinxAtStartPar
\sphinxcode{\sphinxupquote{|substitution ref|}}
&
\sphinxAtStartPar
\sphinxcode{\sphinxupquote{|substitution ref|}}
&
\sphinxAtStartPar
指示链接(图片,链接等)
\\
\sphinxhline
\sphinxAtStartPar
\sphinxcode{\sphinxupquote{footnote {[}1{]}\_}}
&
\sphinxAtStartPar
footnote %
\begin{footnote}[1]\sphinxAtStartFootnote
脚注1
%
\end{footnote}
&
\sphinxAtStartPar
脚注(包括参考文献)
\\
\sphinxhline
\sphinxAtStartPar
\sphinxcode{\sphinxupquote{citation {[}CIT2002{]}\_}}
&
\sphinxAtStartPar
\sphinxcode{\sphinxupquote{citation {[}CIT2002{]}}}
&
\sphinxAtStartPar
引用
\\
\sphinxhline
\sphinxAtStartPar
\sphinxurl{http://docutils.sf.net/}
&
\sphinxAtStartPar
\sphinxurl{http://docutils.sf.net/}
&
\sphinxAtStartPar
独立链接
\\
\sphinxbottomrule
\end{tabulary}
\sphinxtableafterendhook\par
\sphinxattableend\end{savenotes}


\section{反斜杠转义 Escaping with Backslashes}
\label{\detokenize{reStructureText_syntax:escaping-with-backslashes}}
\sphinxAtStartPar
使用反斜杠来转义任意rst语法符号为符号本身,包括反斜杠自己
\begin{itemize}
\item {} 
\sphinxAtStartPar
转义内联标记:  \sphinxcode{\sphinxupquote{\textbackslash{}*去除斜体*}}

\item {} 
\sphinxAtStartPar
转义反斜杠(用两个反斜杠):  \sphinxcode{\sphinxupquote{\textbackslash{}\textbackslash{}}}

\end{itemize}

\sphinxAtStartPar
在python中,最简单的方式是在字符串外表示raw strings(加r)


\begin{savenotes}\sphinxattablestart
\sphinxthistablewithglobalstyle
\centering
\begin{tabulary}{\linewidth}[t]{TT}
\sphinxtoprule
\sphinxstyletheadfamily 
\sphinxAtStartPar
py字符串
&\sphinxstyletheadfamily 
\sphinxAtStartPar
显示结果
\\
\sphinxmidrule
\sphinxtableatstartofbodyhook
\sphinxAtStartPar
r”””*去除斜体*  “\textbackslash{}””””
&
\sphinxAtStartPar
r”””*去除斜体*  “\textbackslash{}””””
\\
\sphinxhline
\sphinxAtStartPar
“””\textbackslash{}*去除斜体*  “\textbackslash{}\textbackslash{}\textbackslash{}””””
&
\sphinxAtStartPar
“””*去除斜体*  “\textbackslash{}””””
\\
\sphinxhline
\sphinxAtStartPar
“””*去除斜体*  “\textbackslash{}””””
&
\sphinxAtStartPar
“””*去除斜体*  “\textbackslash{}””””
\\
\sphinxbottomrule
\end{tabulary}
\sphinxtableafterendhook\par
\sphinxattableend\end{savenotes}


\section{章节标记 Section Structure}
\label{\detokenize{reStructureText_syntax:section-structure}}
\sphinxAtStartPar
任意可打印的7个bit的ASCII码字符都可以作为章节标识符,它们是

\sphinxAtStartPar
\sphinxcode{\sphinxupquote{! " \# \$ \% \& ' ( ) * + , \sphinxhyphen{} . / : ; < = > ? @ {[} \textbackslash{} {]} \textasciicircum{} \_ ` \{ | \} \textasciitilde{}}}

\sphinxAtStartPar
不过有些可能会看起来比较奇怪,因此推荐使用其中的

\sphinxAtStartPar
\sphinxcode{\sphinxupquote{= \sphinxhyphen{} ` : . ' " \textasciitilde{} \textasciicircum{} \_ * + \#}}
\begin{itemize}
\item {} 
\sphinxAtStartPar
在reStructureText中未明确各个章节标识符层级的顺序,\sphinxstylestrong{它按照标识符在书写文本中的顺序来指定标识符指示的标题层级} 。

\item {} 
\sphinxAtStartPar
在标题上下,使用两行标识符;和只在标题下使用一行标识符。效果是一样的。

\item {} 
\sphinxAtStartPar
标题标识符的数量至少要和标题文本等长

\item {} 
\sphinxAtStartPar
建议定义如下标题标识符层级(从高到低)为  \sphinxcode{\sphinxupquote{= \sphinxhyphen{} , . *}}

\end{itemize}

\sphinxAtStartPar
可以使用如下标准定义各级标题

\begin{sphinxVerbatim}[commandchars=\\\{\}]
\PYG{n}{一级标题}
\PYG{o}{==}\PYG{o}{==}\PYG{o}{==}\PYG{o}{==}\PYG{o}{==}
\PYG{n}{二级标题}
\PYG{o}{\PYGZhy{}}\PYG{o}{\PYGZhy{}}\PYG{o}{\PYGZhy{}}\PYG{o}{\PYGZhy{}}\PYG{o}{\PYGZhy{}}\PYG{o}{\PYGZhy{}}\PYG{o}{\PYGZhy{}}\PYG{o}{\PYGZhy{}}\PYG{o}{\PYGZhy{}}\PYG{o}{\PYGZhy{}}
\PYG{n}{三级标题}
\PYG{p}{,}\PYG{p}{,}\PYG{p}{,}\PYG{p}{,}\PYG{p}{,}\PYG{p}{,}\PYG{p}{,}\PYG{p}{,}\PYG{p}{,}
\PYG{n}{四级标题}
\PYG{o}{.}\PYG{o}{.}\PYG{o}{.}\PYG{o}{.}\PYG{o}{.}\PYG{o}{.}\PYG{o}{.}\PYG{o}{.}\PYG{o}{.}\PYG{o}{.}\PYG{o}{.}\PYG{o}{.}
\PYG{n}{五级标题}
\PYG{o}{*}\PYG{o}{*}\PYG{o}{*}\PYG{o}{*}\PYG{o}{*}\PYG{o}{*}\PYG{o}{*}\PYG{o}{*}\PYG{o}{*}\PYG{o}{*}\PYG{o}{*}\PYG{o}{*}\PYG{o}{*}
\end{sphinxVerbatim}


\section{段落 Paragraphs}
\label{\detokenize{reStructureText_syntax:paragraphs}}
\sphinxAtStartPar
段落一般隶属于某个章节中,是一块左对齐并且没有其他元素体标记的块。
在.rst文件中,段落和其他内容的分割是靠 \sphinxstylestrong{空行} 来完成
\begin{quote}

\sphinxAtStartPar
如果段落相较于其他的段落有 \sphinxstylestrong{缩进**(这段缩进了4个空格),reStructuredText会解析为 **引用段落} ,样式上有些不同。
\end{quote}


\section{无序列表 Bullet Lists}
\label{\detokenize{reStructureText_syntax:bullet-lists}}
\sphinxAtStartPar
reStructuredText中 \sphinxhref{https://docutils.sourceforge.io/docs/ref/rst/restructuredtext.html\#bullet-lists}{无序列表} 的语法和Markdown中的是一样的。
一般使用 \sphinxcode{\sphinxupquote{"+ ","\sphinxhyphen{} ","* "}} 来作为无序列表的指示符,利用缩进来指示列表之间的嵌套关系。
\begin{itemize}
\item {} 
\sphinxAtStartPar
列表的起始项和终止项前后是需要空行的

\item {} 
\sphinxAtStartPar
同层级之间的列表项之间空行可加可不加,不同层级之间的列表项必须加空行

\item {} 
\sphinxAtStartPar
层级中内容有多段,则第二段(后续段落内容)无需指示列表标识符,只需保持同样缩进(即左对齐)。

\item {} 
\sphinxAtStartPar
指示标识符可以混用,但是不推荐,推荐同样层级使用同样符号,一般层级顺序就按照 \sphinxcode{\sphinxupquote{"+ ","\sphinxhyphen{} ","* "}}

\end{itemize}

\sphinxAtStartPar
\sphinxstylestrong{示例如下:}

\begin{sphinxVerbatim}[commandchars=\\\{\}]
+ 一级列表第一项
+ 一级列表第二项

  \PYGZhy{} 二级列表,换层级加空行

    二级列表的第二段内容,加空行,缩进对齐

  + 依然是二级列表,指示标识符可以混用,但不推荐

    * 第三级
\end{sphinxVerbatim}

\sphinxAtStartPar
\sphinxstylestrong{效果如下:}
\begin{itemize}
\item {} 
\sphinxAtStartPar
一级列表第一项

\item {} 
\sphinxAtStartPar
一级列表第二项
\begin{itemize}
\item {} 
\sphinxAtStartPar
二级列表,换层级加空行

\sphinxAtStartPar
二级列表的第二段内容,加空行,缩进对齐

\end{itemize}
\begin{itemize}
\item {} 
\sphinxAtStartPar
依然是二级列表,指示标识符可以混用,但不推荐
\begin{itemize}
\item {} 
\sphinxAtStartPar
第三级

\end{itemize}

\end{itemize}

\end{itemize}


\section{有序列表 Enumerated Lists}
\label{\detokenize{reStructureText_syntax:enumerated-lists}}
\sphinxAtStartPar
\sphinxhref{https://docutils.sourceforge.io/docs/ref/rst/restructuredtext.html\#enumerated-lists}{枚举列表} 即顺序列表(ordered\sphinxhyphen{}list),可以使用不同的枚举序号来表示列表。

\sphinxAtStartPar
\sphinxstylestrong{枚举指示符有:}
\begin{itemize}
\item {} 
\sphinxAtStartPar
阿拉伯数字: 1, 2, 3, … (无上限)。

\item {} 
\sphinxAtStartPar
大写字母: A\sphinxhyphen{}Z。

\item {} 
\sphinxAtStartPar
小写字母: a\sphinxhyphen{}z。

\item {} 
\sphinxAtStartPar
大写罗马数字: I, II, III, IV, …, MMMMCMXCIX (4999)。

\item {} 
\sphinxAtStartPar
小写罗马数字: i, ii, iii, iv, …, mmmmcmxcix (4999)。

\end{itemize}

\sphinxAtStartPar
并且可以使用 “\#” 来自动自增。

\sphinxAtStartPar
\sphinxstylestrong{支持添加的前缀和后缀:}
\begin{itemize}
\item {} 
\sphinxAtStartPar
. 后缀: “1.”, “A.”, “a.”, “I.”, “i.”。

\item {} 
\sphinxAtStartPar
() 包起来: “(1)”, “(A)”, “(a)”, “(I)”, “(i)”。

\item {} 
\sphinxAtStartPar
) 后缀: “1)”, “A)”, “a)”, “I)”, “i)”。

\end{itemize}

\sphinxAtStartPar
当正常的文本中包含可被识别为列表的内容时( \sphinxcode{\sphinxupquote{A. 1. (b) I}} 等),为了避免被识别,可以采取如下措施:
\begin{enumerate}
\sphinxsetlistlabels{\arabic}{enumi}{enumii}{}{.}%
\item {} 
\sphinxAtStartPar
将一行内容,折断为多行书写,这样会被识别为 \sphinxstylestrong{段落} 内容,而不会解析为列表;

\item {} 
\sphinxAtStartPar
使用反斜杠 “\textbackslash{}” 在段首进行转义。

\end{enumerate}

\sphinxAtStartPar
\sphinxstylestrong{示例如下:}

\begin{sphinxVerbatim}[commandchars=\\\{\}]
A. Einstein was a really
smart dude. (跨行避免)

A. Einstein was a really smart dude.(未避免)

\PYGZbs{}A. Einstein was a really smart dude.(使用\PYGZbs{}转义)
\end{sphinxVerbatim}

\sphinxAtStartPar
\sphinxstylestrong{效果如下:}

\sphinxAtStartPar
A. Einstein was a really
smart dude.
\begin{enumerate}
\sphinxsetlistlabels{\Alph}{enumi}{enumii}{}{.}%
\item {} 
\sphinxAtStartPar
Einstein was a really smart dude.

\end{enumerate}

\sphinxAtStartPar
A. Einstein was a really smart dude.

\sphinxAtStartPar
有序列表也支持嵌套,规则和无序列表一致

\begin{sphinxVerbatim}[commandchars=\\\{\}]
\PYG{l+m+mf}{1.} \PYG{n}{Item} \PYG{l+m+mi}{1} \PYG{n}{initial} \PYG{n}{text}\PYG{o}{.}

   \PYG{n}{a}\PYG{p}{)} \PYG{n}{Item} \PYG{l+m+mi}{1}\PYG{n}{a}\PYG{o}{.}
   \PYG{n}{b}\PYG{p}{)} \PYG{n}{Item} \PYG{l+m+mi}{1}\PYG{n}{b}\PYG{o}{.}

\PYG{c+c1}{\PYGZsh{}. a) Item 2a.使用\PYGZsh{}号自增}
   \PYG{n}{b}\PYG{p}{)} \PYG{n}{Item} \PYG{l+m+mi}{2}\PYG{n}{b}\PYG{o}{.}
\end{sphinxVerbatim}

\sphinxAtStartPar
\sphinxstylestrong{效果如下:}
\begin{enumerate}
\sphinxsetlistlabels{\arabic}{enumi}{enumii}{}{.}%
\item {} 
\sphinxAtStartPar
Item 1 initial text.
\begin{enumerate}
\sphinxsetlistlabels{\alph}{enumii}{enumiii}{}{)}%
\item {} 
\sphinxAtStartPar
Item 1a.

\item {} 
\sphinxAtStartPar
Item 1b.

\end{enumerate}

\item {} \begin{enumerate}
\sphinxsetlistlabels{\alph}{enumii}{enumiii}{}{)}%
\item {} 
\sphinxAtStartPar
Item 2a.使用\#号自增

\item {} 
\sphinxAtStartPar
Item 2b.

\end{enumerate}

\end{enumerate}


\section{定义列表Definition Lists}
\label{\detokenize{reStructureText_syntax:definition-lists}}
\sphinxAtStartPar
\sphinxhref{https://docutils.sourceforge.io/docs/ref/rst/restructuredtext.html\#definition-lists}{定义列表} 可以理解为解释列表,即名词解释(definition\_list, classifier, definition)。

\sphinxAtStartPar
条目占一行,解释文本要有缩进;多层可根据缩进实现。

\sphinxAtStartPar
各个条目由三部分组成,条目名称(term),条目属性(classifier),条目定义(definition), 条目名称和条目属性在同一行,使用空格、冒号、空格(” : “)连接,其中条目属性可以为空,也可以有多个
条目定义需要换行缩进。

\sphinxAtStartPar
\sphinxstylestrong{示例如下:}

\begin{sphinxVerbatim}[commandchars=\\\{\}]
\PYG{n}{term} \PYG{l+m+mi}{1}
  \PYG{n}{Definition} \PYG{l+m+mf}{1.}

\PYG{n}{term} \PYG{l+m+mi}{2}
    \PYG{n}{Definition} \PYG{l+m+mi}{2}\PYG{p}{,} \PYG{n}{paragraph} \PYG{l+m+mf}{1.}

    \PYG{n}{Definition} \PYG{l+m+mi}{2}\PYG{p}{,} \PYG{n}{paragraph} \PYG{l+m+mf}{2.}

\PYG{n}{term} \PYG{l+m+mi}{3} \PYG{p}{:} \PYG{n}{classifier}
    \PYG{n}{Definition} \PYG{l+m+mf}{3.}

\PYG{n}{term} \PYG{l+m+mi}{4} \PYG{p}{:} \PYG{n}{classifier} \PYG{n}{one} \PYG{p}{:} \PYG{n}{classifier} \PYG{n}{two}
    \PYG{n}{Definition} \PYG{l+m+mf}{4.}
\end{sphinxVerbatim}

\sphinxAtStartPar
\sphinxstylestrong{效果如下:}
\begin{description}
\sphinxlineitem{term 1}
\sphinxAtStartPar
Definition 1.

\sphinxlineitem{term 2}
\sphinxAtStartPar
Definition 2, paragraph 1.

\sphinxAtStartPar
Definition 2, paragraph 2.

\sphinxlineitem{term 3}{[}classifier{]}
\sphinxAtStartPar
Definition 3.

\sphinxlineitem{term 4}{[}classifier one{]}{[}classifier two{]}
\sphinxAtStartPar
Definition 4.

\end{description}


\section{字段列表 Field Lists}
\label{\detokenize{reStructureText_syntax:field-lists}}
\sphinxAtStartPar
\sphinxhref{https://docutils.sourceforge.io/docs/ref/rst/restructuredtext.html\#field-lists}{字段列表} 用于指令解释,或者数据库字段(记录)解释的场景。

\sphinxAtStartPar
它在形式上有点像两列的表格,因此在 field body中的功能是和在表格中一样的(即支持嵌套,跨行等等)。

\sphinxAtStartPar
\sphinxstylestrong{示例如下:}

\begin{sphinxVerbatim}[commandchars=\\\{\}]
:Date: 2020\PYGZhy{}02\PYGZhy{}02
:Version: 1
:Authors: \PYGZhy{} fire
          \PYGZhy{} firewang
          \PYGZhy{} firewang
:Indentation: Since the field marker may be quite long, the second
   and subsequent lines of the field body do not have to line up with first line.
   解释可能很长,第二行不用和第一行对齐,但是后续行必须和第二行对齐。
:Parameter i: field name可以是phrase,即可以带空格,但是不能带\PYGZdq{}:\PYGZdq{}
\end{sphinxVerbatim}

\sphinxAtStartPar
\sphinxstylestrong{效果如下:}
\begin{quote}\begin{description}
\sphinxlineitem{Date}
\sphinxAtStartPar
2020\sphinxhyphen{}02\sphinxhyphen{}02

\sphinxlineitem{Version}
\sphinxAtStartPar
1

\sphinxlineitem{Authors}\begin{itemize}
\item {} 
\sphinxAtStartPar
fire

\item {} 
\sphinxAtStartPar
firewang

\item {} 
\sphinxAtStartPar
firewang

\end{itemize}

\sphinxlineitem{Indentation}
\sphinxAtStartPar
Since the field marker may be quite long, the second
and subsequent lines of the field body do not have to line up with first line.
解释可能很长,第二行不用和第一行对齐,但是后续行必须和第二行对齐。

\sphinxlineitem{Parameter i}
\sphinxAtStartPar
field name可以是phrase,即可以带空格,但是不能带”:”

\end{description}\end{quote}


\section{参数(选项)列表 Option Lists}
\label{\detokenize{reStructureText_syntax:option-lists}}
\sphinxAtStartPar
选项列表是一个左列为参数,右列为参数说明的两列列表(无表头),用于command\sphinxhyphen{}line参数解释。

\sphinxAtStartPar
支持三种参数书写形式:
\begin{itemize}
\item {} 
\sphinxAtStartPar
由一个短横(Short dash)连接的 POSIX 式。

\item {} 
\sphinxAtStartPar
由两个短横(Short dash)连接的 长POSIX 式。

\item {} 
\sphinxAtStartPar
DOS/VMS参数形式,即由 \sphinxtitleref{/} 起始的参数形式。

\end{itemize}

\sphinxAtStartPar
\sphinxstylestrong{示例如下:}

\begin{sphinxVerbatim}[commandchars=\\\{\}]
\PYG{o}{\PYGZhy{}}\PYG{n}{a}            \PYG{n}{command}\PYG{o}{\PYGZhy{}}\PYG{n}{line} \PYG{n}{option} \PYG{l+s+s2}{\PYGZdq{}}\PYG{l+s+s2}{a}\PYG{l+s+s2}{\PYGZdq{}}
\PYG{o}{\PYGZhy{}}\PYG{n}{b} \PYG{n}{file}       \PYG{n}{options} \PYG{n}{can} \PYG{n}{have} \PYG{n}{arguments}
              \PYG{o+ow}{and} \PYG{n}{long} \PYG{n}{descriptions}
\PYG{o}{\PYGZhy{}}\PYG{o}{\PYGZhy{}}\PYG{n}{long}        \PYG{n}{options} \PYG{n}{can} \PYG{n}{be} \PYG{n}{long} \PYG{n}{also}
\PYG{o}{\PYGZhy{}}\PYG{o}{\PYGZhy{}}\PYG{n+nb}{input}\PYG{o}{=}\PYG{n}{file}  \PYG{n}{long} \PYG{n}{options} \PYG{n}{can} \PYG{n}{also} \PYG{n}{have}
              \PYG{n}{arguments}
\PYG{o}{/}\PYG{n}{V}            \PYG{n}{DOS}\PYG{o}{/}\PYG{n}{VMS}\PYG{o}{\PYGZhy{}}\PYG{n}{style} \PYG{n}{options} \PYG{n}{too}
\end{sphinxVerbatim}

\sphinxAtStartPar
\sphinxstylestrong{效果如下:}
\begin{optionlist}{3cm}
\item [\sphinxhyphen{}a]  
\sphinxAtStartPar
command\sphinxhyphen{}line option “a”
\item [\sphinxhyphen{}b file]  
\sphinxAtStartPar
options can have arguments
and long descriptions
\item [\sphinxhyphen{}\sphinxhyphen{}long]  
\sphinxAtStartPar
options can be long also
\item [\sphinxhyphen{}\sphinxhyphen{}input=file]  
\sphinxAtStartPar
long options can also have
arguments
\item [/V]  
\sphinxAtStartPar
DOS/VMS\sphinxhyphen{}style options too
\end{optionlist}


\section{文字块 Literal Blocks}
\label{\detokenize{reStructureText_syntax:literal-blocks}}
\sphinxAtStartPar
\sphinxhref{https://docutils.sourceforge.io/docs/ref/rst/restructuredtext.html\#literal-blocks}{文字块} 就是一段文字信息,指示符为连续两个冒号 \sphinxcode{\sphinxupquote{::}} ,支持文字块的嵌套。

\sphinxAtStartPar
文字块支持三种形式的语法(完全等价)
\begin{enumerate}
\sphinxsetlistlabels{\arabic}{enumi}{enumii}{}{.}%
\item {} 
\sphinxAtStartPar
起始新行,后接空行,块内容需缩进

\end{enumerate}

\sphinxAtStartPar
\sphinxstylestrong{示例如下:}

\begin{sphinxVerbatim}[commandchars=\\\{\}]
\PYG{p}{:}\PYG{p}{:}

  \PYG{n}{缩进后填写块内容}
\end{sphinxVerbatim}

\sphinxAtStartPar
\sphinxstylestrong{效果如下:}

\begin{sphinxVerbatim}[commandchars=\\\{\}]
\PYG{n}{缩进后填写块内容}
\end{sphinxVerbatim}
\begin{enumerate}
\sphinxsetlistlabels{\arabic}{enumi}{enumii}{}{.}%
\setcounter{enumi}{1}
\item {} 
\sphinxAtStartPar
部分简化,前文带一个冒号,加一个空格后,双冒号接在前文后面,不另起行,同时会显示单个冒号,块内容同样缩进

\end{enumerate}

\sphinxAtStartPar
\sphinxstylestrong{示例如下:}

\begin{sphinxVerbatim}[commandchars=\\\{\}]
这里是前面内容,下面引用: ::

  缩进后填写块内容
\end{sphinxVerbatim}

\sphinxAtStartPar
\sphinxstylestrong{效果如下:}

\sphinxAtStartPar
这里是前面内容,下面引用:

\begin{sphinxVerbatim}[commandchars=\\\{\}]
\PYG{n}{缩进后填写块内容}
\end{sphinxVerbatim}
\begin{enumerate}
\sphinxsetlistlabels{\arabic}{enumi}{enumii}{}{.}%
\setcounter{enumi}{2}
\item {} 
\sphinxAtStartPar
完全简化,双冒号接在前文后面,不另起行,同时会显示单个冒号,块内容同样缩进

\end{enumerate}

\sphinxAtStartPar
\sphinxstylestrong{示例如下:}

\begin{sphinxVerbatim}[commandchars=\\\{\}]
这里是前面内容,下面引用::

\PYGZgt{} 在(部分/完全)简化形势下支持单行引用形式的嵌套
\PYGZgt{} 再来一个单行引用
\end{sphinxVerbatim}

\sphinxAtStartPar
\sphinxstylestrong{效果如下:}

\sphinxAtStartPar
这里是前面内容,下面引用:

\begin{sphinxVerbatim}[commandchars=\\\{\}]
\PYG{o}{\PYGZgt{}} \PYG{n}{在简化形势下支持单行引用形式的嵌套}
\PYG{o}{\PYGZgt{}} \PYG{n}{再来一个单行引用}
\end{sphinxVerbatim}


\section{行块 Line Blocks}
\label{\detokenize{reStructureText_syntax:line-blocks}}
\begin{sphinxVerbatim}[commandchars=\\\{\}]
| 行块使用 | 指示符,
| 一般用于描述地址,歌
  词,诗歌,简单列表等。
\end{sphinxVerbatim}

\sphinxAtStartPar
\sphinxstylestrong{效果如下:}

\begin{DUlineblock}{0em}
\item[] 行块使用 “|” 指示符,
\item[] 一般用于描述地址,歌词,诗歌,简单列表等。
\end{DUlineblock}


\section{块引用 Block Quotes}
\label{\detokenize{reStructureText_syntax:block-quotes}}
\sphinxAtStartPar
块引用是 \sphinxstylestrong{通过缩进来实现} 的,引用块要在前面的段落基础上缩进。

\sphinxAtStartPar
通常引用结尾会加上出处(attribution),出处的文字块开头是两个或者三个连续短横(”–“,”—”)后面加上出处信息。

\sphinxAtStartPar
块引用可以使用空的注释 .. 分隔上下的块引用。

\sphinxAtStartPar
注意在新的块和出处都要添加一个空行。

\sphinxAtStartPar
\sphinxstylestrong{示例如下:}

\begin{sphinxVerbatim}[commandchars=\\\{\}]
实际效果:

    “真的猛士,敢于直面惨淡的人生,敢于正视淋漓的鲜血。”

    \PYGZhy{}\PYGZhy{}\PYGZhy{} 鲁迅

..

    “人生的意志和劳动将创造奇迹般的奇迹。”

    \PYGZhy{}\PYGZhy{} 涅克拉索
\end{sphinxVerbatim}

\sphinxAtStartPar
实际效果:
\begin{quote}

\sphinxAtStartPar
“真的猛士,敢于直面惨淡的人生,敢于正视淋漓的鲜血。”

\begin{flushright}
---鲁迅
\end{flushright}
\end{quote}
\begin{quote}

\sphinxAtStartPar
“人生的意志和劳动将创造奇迹般的奇迹。”

\begin{flushright}
---涅克拉索
\end{flushright}
\end{quote}


\section{文档测试块 Doctest Blocks}
\label{\detokenize{reStructureText_syntax:doctest-blocks}}
\sphinxAtStartPar
文档测试块是交互式的Python会话,以 \sphinxcode{\sphinxupquote{>>>}} 开始,一个空行结束,是一种特殊的文字块,\sphinxstylestrong{内容不需要缩进} 。

\sphinxAtStartPar
可直接复制到python的 docstrings中,用于为doctest模块提供测试环境。

\sphinxAtStartPar
当文字块语法和文档测试块语法同时出现时,文字块语法优先级更高。

\begin{sphinxVerbatim}[commandchars=\\\{\}]
\PYG{g+gp}{\PYGZgt{}\PYGZgt{}\PYGZgt{} }\PYG{n+nb}{print}\PYG{p}{(}\PYG{l+s+s1}{\PYGZsq{}}\PYG{l+s+s1}{this is a Doctest block}\PYG{l+s+s1}{\PYGZsq{}}\PYG{p}{)}
\PYG{g+go}{this is a Doctest block}
\end{sphinxVerbatim}


\section{表格 Tables}
\label{\detokenize{reStructureText_syntax:tables}}
\sphinxAtStartPar
reStructureText提供两种表格:网格表格(Grid Tables), 简单表格(Simple Tables)。

\sphinxAtStartPar
表格前后都需要空行


\subsection{网格表格}
\label{\detokenize{reStructureText_syntax:id9}}\begin{itemize}
\item {} 
\sphinxAtStartPar
“\sphinxhyphen{}” 分隔行(短破折号,减号)

\item {} 
\sphinxAtStartPar
“=” 分隔表头和表体行

\item {} 
\sphinxAtStartPar
“|” 分隔列

\item {} 
\sphinxAtStartPar
“+” 表示行和列相交的节点

\end{itemize}

\sphinxAtStartPar
\sphinxstylestrong{网格表格注意点}:
\begin{itemize}
\item {} 
\sphinxAtStartPar
网格表格编辑复杂,可以使用Emacs来编辑生成

\item {} 
\sphinxAtStartPar
行和列都支持并格

\item {} 
\sphinxAtStartPar
如果文本内包含”|” ,并且恰好与表格内分隔对齐了,那么会产生错误。\sphinxhref{https://docutils.sourceforge.io/docs/ref/rst/restructuredtext.html\#tables}{解决方案} : 方式一是加空格避免对齐,方式二是为该行增加一行

\item {} 
\sphinxAtStartPar
可以不包含表头。

\item {} 
\sphinxAtStartPar
列需要和”=”左对齐,不然可能会导致出错

\item {} 
\sphinxAtStartPar
如果碰到第一列为空,需要使用 “\textbackslash{}” 来转义, 不然会被视为是上一行的延续。

\end{itemize}

\sphinxAtStartPar
\sphinxstylestrong{示例:}

\begin{sphinxVerbatim}[commandchars=\\\{\}]
\PYG{o}{+}\PYG{o}{\PYGZhy{}}\PYG{o}{\PYGZhy{}}\PYG{o}{\PYGZhy{}}\PYG{o}{\PYGZhy{}}\PYG{o}{\PYGZhy{}}\PYG{o}{\PYGZhy{}}\PYG{o}{\PYGZhy{}}\PYG{o}{\PYGZhy{}}\PYG{o}{\PYGZhy{}}\PYG{o}{\PYGZhy{}}\PYG{o}{\PYGZhy{}}\PYG{o}{\PYGZhy{}}\PYG{o}{\PYGZhy{}}\PYG{o}{\PYGZhy{}}\PYG{o}{\PYGZhy{}}\PYG{o}{\PYGZhy{}}\PYG{o}{\PYGZhy{}}\PYG{o}{\PYGZhy{}}\PYG{o}{\PYGZhy{}}\PYG{o}{\PYGZhy{}}\PYG{o}{\PYGZhy{}}\PYG{o}{\PYGZhy{}}\PYG{o}{\PYGZhy{}}\PYG{o}{\PYGZhy{}}\PYG{o}{+}\PYG{o}{\PYGZhy{}}\PYG{o}{\PYGZhy{}}\PYG{o}{\PYGZhy{}}\PYG{o}{\PYGZhy{}}\PYG{o}{\PYGZhy{}}\PYG{o}{\PYGZhy{}}\PYG{o}{\PYGZhy{}}\PYG{o}{\PYGZhy{}}\PYG{o}{\PYGZhy{}}\PYG{o}{\PYGZhy{}}\PYG{o}{\PYGZhy{}}\PYG{o}{\PYGZhy{}}\PYG{o}{+}\PYG{o}{\PYGZhy{}}\PYG{o}{\PYGZhy{}}\PYG{o}{\PYGZhy{}}\PYG{o}{\PYGZhy{}}\PYG{o}{\PYGZhy{}}\PYG{o}{\PYGZhy{}}\PYG{o}{\PYGZhy{}}\PYG{o}{\PYGZhy{}}\PYG{o}{\PYGZhy{}}\PYG{o}{\PYGZhy{}}\PYG{o}{+}\PYG{o}{\PYGZhy{}}\PYG{o}{\PYGZhy{}}\PYG{o}{\PYGZhy{}}\PYG{o}{\PYGZhy{}}\PYG{o}{\PYGZhy{}}\PYG{o}{\PYGZhy{}}\PYG{o}{\PYGZhy{}}\PYG{o}{\PYGZhy{}}\PYG{o}{\PYGZhy{}}\PYG{o}{\PYGZhy{}}\PYG{o}{+}
\PYG{o}{|} \PYG{n}{Header} \PYG{n}{row}\PYG{p}{,} \PYG{n}{column} \PYG{l+m+mi}{1}   \PYG{o}{|} \PYG{n}{Header} \PYG{l+m+mi}{2}   \PYG{o}{|} \PYG{n}{Header} \PYG{l+m+mi}{3} \PYG{o}{|} \PYG{n}{Header} \PYG{l+m+mi}{4} \PYG{o}{|}
\PYG{o}{|} \PYG{p}{(}\PYG{n}{header} \PYG{n}{rows} \PYG{n}{optional}\PYG{p}{)} \PYG{o}{|}            \PYG{o}{|}          \PYG{o}{|}          \PYG{o}{|}
\PYG{o}{+}\PYG{o}{==}\PYG{o}{==}\PYG{o}{==}\PYG{o}{==}\PYG{o}{==}\PYG{o}{==}\PYG{o}{==}\PYG{o}{==}\PYG{o}{==}\PYG{o}{==}\PYG{o}{==}\PYG{o}{==}\PYG{o}{+}\PYG{o}{==}\PYG{o}{==}\PYG{o}{==}\PYG{o}{==}\PYG{o}{==}\PYG{o}{==}\PYG{o}{+}\PYG{o}{==}\PYG{o}{==}\PYG{o}{==}\PYG{o}{==}\PYG{o}{==}\PYG{o}{+}\PYG{o}{==}\PYG{o}{==}\PYG{o}{==}\PYG{o}{==}\PYG{o}{==}\PYG{o}{+}
\PYG{o}{|} \PYG{n}{body} \PYG{n}{row} \PYG{l+m+mi}{1}\PYG{p}{,} \PYG{n}{column} \PYG{l+m+mi}{1}   \PYG{o}{|} \PYG{n}{column} \PYG{l+m+mi}{2}   \PYG{o}{|} \PYG{n}{column} \PYG{l+m+mi}{3} \PYG{o}{|} \PYG{n}{column} \PYG{l+m+mi}{4} \PYG{o}{|}
\PYG{o}{+}\PYG{o}{\PYGZhy{}}\PYG{o}{\PYGZhy{}}\PYG{o}{\PYGZhy{}}\PYG{o}{\PYGZhy{}}\PYG{o}{\PYGZhy{}}\PYG{o}{\PYGZhy{}}\PYG{o}{\PYGZhy{}}\PYG{o}{\PYGZhy{}}\PYG{o}{\PYGZhy{}}\PYG{o}{\PYGZhy{}}\PYG{o}{\PYGZhy{}}\PYG{o}{\PYGZhy{}}\PYG{o}{\PYGZhy{}}\PYG{o}{\PYGZhy{}}\PYG{o}{\PYGZhy{}}\PYG{o}{\PYGZhy{}}\PYG{o}{\PYGZhy{}}\PYG{o}{\PYGZhy{}}\PYG{o}{\PYGZhy{}}\PYG{o}{\PYGZhy{}}\PYG{o}{\PYGZhy{}}\PYG{o}{\PYGZhy{}}\PYG{o}{\PYGZhy{}}\PYG{o}{\PYGZhy{}}\PYG{o}{+}\PYG{o}{\PYGZhy{}}\PYG{o}{\PYGZhy{}}\PYG{o}{\PYGZhy{}}\PYG{o}{\PYGZhy{}}\PYG{o}{\PYGZhy{}}\PYG{o}{\PYGZhy{}}\PYG{o}{\PYGZhy{}}\PYG{o}{\PYGZhy{}}\PYG{o}{\PYGZhy{}}\PYG{o}{\PYGZhy{}}\PYG{o}{\PYGZhy{}}\PYG{o}{\PYGZhy{}}\PYG{o}{+}\PYG{o}{\PYGZhy{}}\PYG{o}{\PYGZhy{}}\PYG{o}{\PYGZhy{}}\PYG{o}{\PYGZhy{}}\PYG{o}{\PYGZhy{}}\PYG{o}{\PYGZhy{}}\PYG{o}{\PYGZhy{}}\PYG{o}{\PYGZhy{}}\PYG{o}{\PYGZhy{}}\PYG{o}{\PYGZhy{}}\PYG{o}{+}\PYG{o}{\PYGZhy{}}\PYG{o}{\PYGZhy{}}\PYG{o}{\PYGZhy{}}\PYG{o}{\PYGZhy{}}\PYG{o}{\PYGZhy{}}\PYG{o}{\PYGZhy{}}\PYG{o}{\PYGZhy{}}\PYG{o}{\PYGZhy{}}\PYG{o}{\PYGZhy{}}\PYG{o}{\PYGZhy{}}\PYG{o}{+}
\PYG{o}{|} \PYG{n}{body} \PYG{n}{row} \PYG{l+m+mi}{2}             \PYG{o}{|} \PYG{n}{Cells} \PYG{n}{may} \PYG{n}{span} \PYG{n}{columns}\PYG{o}{.}          \PYG{o}{|}
\PYG{o}{+}\PYG{o}{\PYGZhy{}}\PYG{o}{\PYGZhy{}}\PYG{o}{\PYGZhy{}}\PYG{o}{\PYGZhy{}}\PYG{o}{\PYGZhy{}}\PYG{o}{\PYGZhy{}}\PYG{o}{\PYGZhy{}}\PYG{o}{\PYGZhy{}}\PYG{o}{\PYGZhy{}}\PYG{o}{\PYGZhy{}}\PYG{o}{\PYGZhy{}}\PYG{o}{\PYGZhy{}}\PYG{o}{\PYGZhy{}}\PYG{o}{\PYGZhy{}}\PYG{o}{\PYGZhy{}}\PYG{o}{\PYGZhy{}}\PYG{o}{\PYGZhy{}}\PYG{o}{\PYGZhy{}}\PYG{o}{\PYGZhy{}}\PYG{o}{\PYGZhy{}}\PYG{o}{\PYGZhy{}}\PYG{o}{\PYGZhy{}}\PYG{o}{\PYGZhy{}}\PYG{o}{\PYGZhy{}}\PYG{o}{+}\PYG{o}{\PYGZhy{}}\PYG{o}{\PYGZhy{}}\PYG{o}{\PYGZhy{}}\PYG{o}{\PYGZhy{}}\PYG{o}{\PYGZhy{}}\PYG{o}{\PYGZhy{}}\PYG{o}{\PYGZhy{}}\PYG{o}{\PYGZhy{}}\PYG{o}{\PYGZhy{}}\PYG{o}{\PYGZhy{}}\PYG{o}{\PYGZhy{}}\PYG{o}{\PYGZhy{}}\PYG{o}{+}\PYG{o}{\PYGZhy{}}\PYG{o}{\PYGZhy{}}\PYG{o}{\PYGZhy{}}\PYG{o}{\PYGZhy{}}\PYG{o}{\PYGZhy{}}\PYG{o}{\PYGZhy{}}\PYG{o}{\PYGZhy{}}\PYG{o}{\PYGZhy{}}\PYG{o}{\PYGZhy{}}\PYG{o}{\PYGZhy{}}\PYG{o}{\PYGZhy{}}\PYG{o}{\PYGZhy{}}\PYG{o}{\PYGZhy{}}\PYG{o}{\PYGZhy{}}\PYG{o}{\PYGZhy{}}\PYG{o}{\PYGZhy{}}\PYG{o}{\PYGZhy{}}\PYG{o}{\PYGZhy{}}\PYG{o}{\PYGZhy{}}\PYG{o}{\PYGZhy{}}\PYG{o}{\PYGZhy{}}\PYG{o}{+}
\PYG{o}{|} \PYG{n}{body} \PYG{n}{row} \PYG{l+m+mi}{3}             \PYG{o}{|} \PYG{n}{Cells} \PYG{n}{may}  \PYG{o}{|} \PYG{o}{\PYGZhy{}} \PYG{n}{Table} \PYG{n}{cells}       \PYG{o}{|}
\PYG{o}{+}\PYG{o}{\PYGZhy{}}\PYG{o}{\PYGZhy{}}\PYG{o}{\PYGZhy{}}\PYG{o}{\PYGZhy{}}\PYG{o}{\PYGZhy{}}\PYG{o}{\PYGZhy{}}\PYG{o}{\PYGZhy{}}\PYG{o}{\PYGZhy{}}\PYG{o}{\PYGZhy{}}\PYG{o}{\PYGZhy{}}\PYG{o}{\PYGZhy{}}\PYG{o}{\PYGZhy{}}\PYG{o}{\PYGZhy{}}\PYG{o}{\PYGZhy{}}\PYG{o}{\PYGZhy{}}\PYG{o}{\PYGZhy{}}\PYG{o}{\PYGZhy{}}\PYG{o}{\PYGZhy{}}\PYG{o}{\PYGZhy{}}\PYG{o}{\PYGZhy{}}\PYG{o}{\PYGZhy{}}\PYG{o}{\PYGZhy{}}\PYG{o}{\PYGZhy{}}\PYG{o}{\PYGZhy{}}\PYG{o}{+} \PYG{n}{span} \PYG{n}{rows}\PYG{o}{.} \PYG{o}{|} \PYG{o}{\PYGZhy{}} \PYG{n}{contain}           \PYG{o}{|}
\PYG{o}{|} \PYG{n}{body} \PYG{n}{row} \PYG{l+m+mi}{4}             \PYG{o}{|}            \PYG{o}{|} \PYG{o}{\PYGZhy{}} \PYG{n}{body} \PYG{n}{elements}\PYG{o}{.}    \PYG{o}{|}
\PYG{o}{+}\PYG{o}{\PYGZhy{}}\PYG{o}{\PYGZhy{}}\PYG{o}{\PYGZhy{}}\PYG{o}{\PYGZhy{}}\PYG{o}{\PYGZhy{}}\PYG{o}{\PYGZhy{}}\PYG{o}{\PYGZhy{}}\PYG{o}{\PYGZhy{}}\PYG{o}{\PYGZhy{}}\PYG{o}{\PYGZhy{}}\PYG{o}{\PYGZhy{}}\PYG{o}{\PYGZhy{}}\PYG{o}{\PYGZhy{}}\PYG{o}{\PYGZhy{}}\PYG{o}{\PYGZhy{}}\PYG{o}{\PYGZhy{}}\PYG{o}{\PYGZhy{}}\PYG{o}{\PYGZhy{}}\PYG{o}{\PYGZhy{}}\PYG{o}{\PYGZhy{}}\PYG{o}{\PYGZhy{}}\PYG{o}{\PYGZhy{}}\PYG{o}{\PYGZhy{}}\PYG{o}{\PYGZhy{}}\PYG{o}{+}\PYG{o}{\PYGZhy{}}\PYG{o}{\PYGZhy{}}\PYG{o}{\PYGZhy{}}\PYG{o}{\PYGZhy{}}\PYG{o}{\PYGZhy{}}\PYG{o}{\PYGZhy{}}\PYG{o}{\PYGZhy{}}\PYG{o}{\PYGZhy{}}\PYG{o}{\PYGZhy{}}\PYG{o}{\PYGZhy{}}\PYG{o}{\PYGZhy{}}\PYG{o}{\PYGZhy{}}\PYG{o}{+}\PYG{o}{\PYGZhy{}}\PYG{o}{\PYGZhy{}}\PYG{o}{\PYGZhy{}}\PYG{o}{\PYGZhy{}}\PYG{o}{\PYGZhy{}}\PYG{o}{\PYGZhy{}}\PYG{o}{\PYGZhy{}}\PYG{o}{\PYGZhy{}}\PYG{o}{\PYGZhy{}}\PYG{o}{\PYGZhy{}}\PYG{o}{\PYGZhy{}}\PYG{o}{\PYGZhy{}}\PYG{o}{\PYGZhy{}}\PYG{o}{\PYGZhy{}}\PYG{o}{\PYGZhy{}}\PYG{o}{\PYGZhy{}}\PYG{o}{\PYGZhy{}}\PYG{o}{\PYGZhy{}}\PYG{o}{\PYGZhy{}}\PYG{o}{\PYGZhy{}}\PYG{o}{\PYGZhy{}}\PYG{o}{+}
\end{sphinxVerbatim}

\sphinxAtStartPar
\sphinxstylestrong{结果:}


\begin{savenotes}\sphinxattablestart
\sphinxthistablewithglobalstyle
\centering
\begin{tabular}[t]{*{4}{\X{1}{4}}}
\sphinxtoprule
\sphinxstyletheadfamily 
\sphinxAtStartPar
Header row, column 1
(header rows optional)
&\sphinxstyletheadfamily 
\sphinxAtStartPar
Header 2
&\sphinxstyletheadfamily 
\sphinxAtStartPar
Header 3
&\sphinxstyletheadfamily 
\sphinxAtStartPar
Header 4
\\
\sphinxmidrule
\sphinxtableatstartofbodyhook
\sphinxAtStartPar
body row 1, column 1
&
\sphinxAtStartPar
column 2
&
\sphinxAtStartPar
column 3
&
\sphinxAtStartPar
column 4
\\
\sphinxhline
\sphinxAtStartPar
body row 2
&\sphinxstartmulticolumn{3}%
\begin{varwidth}[t]{\sphinxcolwidth{3}{4}}
\sphinxAtStartPar
Cells may span columns.
\par
\vskip-\baselineskip\vbox{\hbox{\strut}}\end{varwidth}%
\sphinxstopmulticolumn
\\
\sphinxhline
\sphinxAtStartPar
body row 3
&\sphinxmultirow{2}{12}{%
\begin{varwidth}[t]{\sphinxcolwidth{1}{4}}
\sphinxAtStartPar
Cells may
span rows.
\par
\vskip-\baselineskip\vbox{\hbox{\strut}}\end{varwidth}%
}%
&\sphinxstartmulticolumn{2}%
\sphinxmultirow{2}{13}{%
\begin{varwidth}[t]{\sphinxcolwidth{2}{4}}
\begin{itemize}
\item {} 
\sphinxAtStartPar
Table cells

\item {} 
\sphinxAtStartPar
contain

\item {} 
\sphinxAtStartPar
body elements.

\end{itemize}
\par
\vskip-\baselineskip\vbox{\hbox{\strut}}\end{varwidth}%
}%
\sphinxstopmulticolumn
\\
\sphinxcline{1-1}\sphinxvlinecrossing{2}\sphinxfixclines{4}
\sphinxAtStartPar
body row 4
&\sphinxtablestrut{12}&\multicolumn{2}{l}{\sphinxtablestrut{13}}\\
\sphinxbottomrule
\end{tabular}
\sphinxtableafterendhook\par
\sphinxattableend\end{savenotes}


\subsection{简单表格}
\label{\detokenize{reStructureText_syntax:id11}}
\sphinxAtStartPar
简单表格使用 “=” 和 “\_” 来进行绘制,其中”=” 放置于表格的最外两行(首行和末行),如果有表头,则表头也用该符号进行分隔,”\_”用于跨列合并(column span)。

\sphinxAtStartPar
简单表格需要各列首字母与该列指示的”=”对齐(表头可不对齐,为了保持统一,尽量保持左对齐),每列的”=”需要覆盖该列字符的长度


\subsubsection{包含表头的简单表格}
\label{\detokenize{reStructureText_syntax:id12}}
\sphinxAtStartPar
\sphinxstylestrong{语法如下:}

\begin{sphinxVerbatim}[commandchars=\\\{\}]
\PYG{o}{==}\PYG{o}{==}\PYG{o}{=}  \PYG{o}{==}\PYG{o}{==}\PYG{o}{=}  \PYG{o}{==}\PYG{o}{==}\PYG{o}{==}\PYG{o}{=}
\PYG{n}{A}      \PYG{n}{B}      \PYG{n}{A} \PYG{o+ow}{and} \PYG{n}{B}
\PYG{o}{==}\PYG{o}{==}\PYG{o}{=}  \PYG{o}{==}\PYG{o}{==}\PYG{o}{=}  \PYG{o}{==}\PYG{o}{==}\PYG{o}{==}\PYG{o}{=}
\PYG{k+kc}{False}  \PYG{k+kc}{False}  \PYG{k+kc}{False}
\PYG{k+kc}{True}   \PYG{k+kc}{False}  \PYG{k+kc}{False}
\PYG{k+kc}{False}  \PYG{k+kc}{True}   \PYG{k+kc}{False}
\PYG{k+kc}{True}   \PYG{k+kc}{True}   \PYG{k+kc}{True}
\PYG{o}{==}\PYG{o}{==}\PYG{o}{=}  \PYG{o}{==}\PYG{o}{==}\PYG{o}{=}  \PYG{o}{==}\PYG{o}{==}\PYG{o}{==}\PYG{o}{=}
\end{sphinxVerbatim}

\sphinxAtStartPar
\sphinxstylestrong{效果如下:}


\begin{savenotes}\sphinxattablestart
\sphinxthistablewithglobalstyle
\centering
\begin{tabulary}{\linewidth}[t]{TTT}
\sphinxtoprule
\sphinxstyletheadfamily 
\sphinxAtStartPar
A
&\sphinxstyletheadfamily 
\sphinxAtStartPar
B
&\sphinxstyletheadfamily 
\sphinxAtStartPar
A and B
\\
\sphinxmidrule
\sphinxtableatstartofbodyhook
\sphinxAtStartPar
False
&
\sphinxAtStartPar
False
&
\sphinxAtStartPar
False
\\
\sphinxhline
\sphinxAtStartPar
True
&
\sphinxAtStartPar
False
&
\sphinxAtStartPar
False
\\
\sphinxhline
\sphinxAtStartPar
False
&
\sphinxAtStartPar
True
&
\sphinxAtStartPar
False
\\
\sphinxhline
\sphinxAtStartPar
True
&
\sphinxAtStartPar
True
&
\sphinxAtStartPar
True
\\
\sphinxbottomrule
\end{tabulary}
\sphinxtableafterendhook\par
\sphinxattableend\end{savenotes}


\subsubsection{无表头的简单表格}
\label{\detokenize{reStructureText_syntax:id13}}
\sphinxAtStartPar
\sphinxstylestrong{语法如下:}

\begin{sphinxVerbatim}[commandchars=\\\{\}]
\PYG{o}{==}\PYG{o}{==}\PYG{o}{=}  \PYG{o}{==}\PYG{o}{==}\PYG{o}{=}  \PYG{o}{==}\PYG{o}{==}\PYG{o}{==}\PYG{o}{=}
\PYG{k+kc}{False}  \PYG{k+kc}{False}  \PYG{k+kc}{False}
\PYG{k+kc}{True}   \PYG{k+kc}{False}  \PYG{k+kc}{False}
\PYG{k+kc}{False}  \PYG{k+kc}{True}   \PYG{k+kc}{False}
\PYG{k+kc}{True}   \PYG{k+kc}{True}   \PYG{k+kc}{True}
\PYG{o}{==}\PYG{o}{==}\PYG{o}{=}  \PYG{o}{==}\PYG{o}{==}\PYG{o}{=}  \PYG{o}{==}\PYG{o}{==}\PYG{o}{==}\PYG{o}{=}
\end{sphinxVerbatim}

\sphinxAtStartPar
\sphinxstylestrong{效果如下:}


\begin{savenotes}\sphinxattablestart
\sphinxthistablewithglobalstyle
\centering
\begin{tabulary}{\linewidth}[t]{TTT}
\sphinxtoprule
\sphinxtableatstartofbodyhook
\sphinxAtStartPar
False
&
\sphinxAtStartPar
False
&
\sphinxAtStartPar
False
\\
\sphinxhline
\sphinxAtStartPar
True
&
\sphinxAtStartPar
False
&
\sphinxAtStartPar
False
\\
\sphinxhline
\sphinxAtStartPar
False
&
\sphinxAtStartPar
True
&
\sphinxAtStartPar
False
\\
\sphinxhline
\sphinxAtStartPar
True
&
\sphinxAtStartPar
True
&
\sphinxAtStartPar
True
\\
\sphinxbottomrule
\end{tabulary}
\sphinxtableafterendhook\par
\sphinxattableend\end{savenotes}


\subsubsection{跨列合并}
\label{\detokenize{reStructureText_syntax:id14}}
\sphinxAtStartPar
“\_”用于跨列合并,\sphinxstylestrong{仅支持在表头使用},”\_”长度需要从起始列的第一个指示符”=”到终止列的最后一个指示符”=”

\sphinxAtStartPar
\sphinxstylestrong{语法如下:}

\begin{sphinxVerbatim}[commandchars=\\\{\}]
\PYG{o}{==}\PYG{o}{==}\PYG{o}{=}  \PYG{o}{==}\PYG{o}{==}\PYG{o}{=}  \PYG{o}{==}\PYG{o}{==}\PYG{o}{==}
\PYG{n}{合并两列}      \PYG{n}{单独列}
\PYG{o}{\PYGZhy{}}\PYG{o}{\PYGZhy{}}\PYG{o}{\PYGZhy{}}\PYG{o}{\PYGZhy{}}\PYG{o}{\PYGZhy{}}\PYG{o}{\PYGZhy{}}\PYG{o}{\PYGZhy{}}\PYG{o}{\PYGZhy{}}\PYG{o}{\PYGZhy{}}\PYG{o}{\PYGZhy{}}\PYG{o}{\PYGZhy{}}\PYG{o}{\PYGZhy{}}  \PYG{o}{\PYGZhy{}}\PYG{o}{\PYGZhy{}}\PYG{o}{\PYGZhy{}}\PYG{o}{\PYGZhy{}}\PYG{o}{\PYGZhy{}}\PYG{o}{\PYGZhy{}}
  \PYG{n}{A}      \PYG{n}{B}    \PYG{n}{A} \PYG{o+ow}{or} \PYG{n}{B}
\PYG{o}{==}\PYG{o}{==}\PYG{o}{=}  \PYG{o}{==}\PYG{o}{==}\PYG{o}{=}  \PYG{o}{==}\PYG{o}{==}\PYG{o}{==}
\PYG{k+kc}{False}  \PYG{k+kc}{False}  \PYG{k+kc}{False}
\PYG{k+kc}{True}   \PYG{k+kc}{False}  \PYG{k+kc}{True}
\PYG{k+kc}{False}  \PYG{k+kc}{True}   \PYG{k+kc}{True}
\PYG{k+kc}{True}   \PYG{k+kc}{True}   \PYG{k+kc}{True}
\PYG{o}{==}\PYG{o}{==}\PYG{o}{=}  \PYG{o}{==}\PYG{o}{==}\PYG{o}{=}  \PYG{o}{==}\PYG{o}{==}\PYG{o}{==}
\end{sphinxVerbatim}

\sphinxAtStartPar
\sphinxstylestrong{效果如下:}


\begin{savenotes}\sphinxattablestart
\sphinxthistablewithglobalstyle
\centering
\begin{tabulary}{\linewidth}[t]{TTT}
\sphinxtoprule
\sphinxstartmulticolumn{2}%
\begin{varwidth}[t]{\sphinxcolwidth{2}{3}}
\sphinxstyletheadfamily \sphinxAtStartPar
合并两列
\par
\vskip-\baselineskip\vbox{\hbox{\strut}}\end{varwidth}%
\sphinxstopmulticolumn
&\sphinxstyletheadfamily 
\sphinxAtStartPar
单独列
\\
\sphinxhline\sphinxstyletheadfamily 
\sphinxAtStartPar
A
&\sphinxstyletheadfamily 
\sphinxAtStartPar
B
&\sphinxstyletheadfamily 
\sphinxAtStartPar
A or B
\\
\sphinxmidrule
\sphinxtableatstartofbodyhook
\sphinxAtStartPar
False
&
\sphinxAtStartPar
False
&
\sphinxAtStartPar
False
\\
\sphinxhline
\sphinxAtStartPar
True
&
\sphinxAtStartPar
False
&
\sphinxAtStartPar
True
\\
\sphinxhline
\sphinxAtStartPar
False
&
\sphinxAtStartPar
True
&
\sphinxAtStartPar
True
\\
\sphinxhline
\sphinxAtStartPar
True
&
\sphinxAtStartPar
True
&
\sphinxAtStartPar
True
\\
\sphinxbottomrule
\end{tabulary}
\sphinxtableafterendhook\par
\sphinxattableend\end{savenotes}


\subsubsection{单个表格中可以多行}
\label{\detokenize{reStructureText_syntax:id15}}\begin{itemize}
\item {} 
\sphinxAtStartPar
简单表格的单个格子中可以包含多行的内容(比如列表),但是不支持行合并;

\item {} 
\sphinxAtStartPar
增加空行可以进行换行,否则会自动将文本连接在一起。

\item {} 
\sphinxAtStartPar
首列不能为空,为空时使用 \textbackslash{} 进行占位。

\end{itemize}

\sphinxAtStartPar
\sphinxstylestrong{语法如下:}

\begin{sphinxVerbatim}[commandchars=\\\{\}]
\PYG{o}{==}\PYG{o}{==}\PYG{o}{=}  \PYG{o}{==}\PYG{o}{==}\PYG{o}{==}\PYG{o}{==}\PYG{o}{==}\PYG{o}{==}\PYG{o}{==}\PYG{o}{==}\PYG{o}{==}\PYG{o}{==}\PYG{o}{==}\PYG{o}{==}\PYG{o}{==}\PYG{o}{==}\PYG{o}{==}\PYG{o}{==}\PYG{o}{==}\PYG{o}{=}
\PYG{n}{col} \PYG{l+m+mi}{1}  \PYG{n}{col} \PYG{l+m+mi}{2}
\PYG{o}{==}\PYG{o}{==}\PYG{o}{=}  \PYG{o}{==}\PYG{o}{==}\PYG{o}{==}\PYG{o}{==}\PYG{o}{==}\PYG{o}{==}\PYG{o}{==}\PYG{o}{==}\PYG{o}{==}\PYG{o}{==}\PYG{o}{==}\PYG{o}{==}\PYG{o}{==}\PYG{o}{==}\PYG{o}{==}\PYG{o}{==}\PYG{o}{==}\PYG{o}{=}
\PYG{l+m+mi}{1}      \PYG{n}{Second} \PYG{n}{column} \PYG{n}{of} \PYG{n}{row} \PYG{l+m+mf}{1.}
\PYG{l+m+mi}{2}      \PYG{n}{Second} \PYG{n}{column} \PYG{n}{of} \PYG{n}{row} \PYG{l+m+mf}{2.}
       \PYG{n}{Second} \PYG{n}{line} \PYG{n}{of} \PYG{n}{paragraph}\PYG{o}{.}
\PYG{l+m+mi}{3}      \PYG{o}{\PYGZhy{}} \PYG{n}{Second} \PYG{n}{column} \PYG{n}{of} \PYG{n}{row} \PYG{l+m+mf}{3.}

       \PYG{o}{\PYGZhy{}} \PYG{n}{Second} \PYG{n}{item} \PYG{o+ow}{in} \PYG{n}{bullet}
         \PYG{n+nb}{list} \PYG{p}{(}\PYG{n}{row} \PYG{l+m+mi}{3}\PYG{p}{,} \PYG{n}{column} \PYG{l+m+mi}{2}\PYG{p}{)}\PYG{o}{.}
\PYGZbs{}      \PYG{n}{Row} \PYG{l+m+mi}{4}\PYG{p}{;} \PYG{n}{column} \PYG{l+m+mi}{1} \PYG{n}{will} \PYG{n}{be} \PYG{n}{empty}\PYG{o}{.}
\PYG{o}{==}\PYG{o}{==}\PYG{o}{=}  \PYG{o}{==}\PYG{o}{==}\PYG{o}{==}\PYG{o}{==}\PYG{o}{==}\PYG{o}{==}\PYG{o}{==}\PYG{o}{==}\PYG{o}{==}\PYG{o}{==}\PYG{o}{==}\PYG{o}{==}\PYG{o}{==}\PYG{o}{==}\PYG{o}{==}\PYG{o}{==}\PYG{o}{==}\PYG{o}{=}
\end{sphinxVerbatim}

\sphinxAtStartPar
\sphinxstylestrong{效果如下:}


\begin{savenotes}\sphinxattablestart
\sphinxthistablewithglobalstyle
\centering
\begin{tabular}[t]{*{2}{\X{1}{2}}}
\sphinxtoprule
\sphinxstyletheadfamily 
\sphinxAtStartPar
col 1
&\sphinxstyletheadfamily 
\sphinxAtStartPar
col 2
\\
\sphinxmidrule
\sphinxtableatstartofbodyhook
\sphinxAtStartPar
1
&
\sphinxAtStartPar
Second column of row 1.
\\
\sphinxhline
\sphinxAtStartPar
2
&
\sphinxAtStartPar
Second column of row 2.
Second line of paragraph.
\\
\sphinxhline
\sphinxAtStartPar
3
&\begin{itemize}
\item {} 
\sphinxAtStartPar
Second column of row 3.

\item {} 
\sphinxAtStartPar
Second item in bullet
list (row 3, column 2).

\end{itemize}
\\
\sphinxhline
\sphinxAtStartPar

&
\sphinxAtStartPar
Row 4; column 1 will be empty.
\\
\sphinxbottomrule
\end{tabular}
\sphinxtableafterendhook\par
\sphinxattableend\end{savenotes}


\section{Transitions}
\label{\detokenize{reStructureText_syntax:transitions}}
\sphinxAtStartPar
转换分隔用于段与段之间的分隔,相当于html中的<hr>,就是跨屏的一个横线。

\sphinxAtStartPar
使用4个及以上的标点符号(推荐使用短横 “\sphinxhyphen{}“)就可以生成,同样需要前后空行,另外,\sphinxstylestrong{不能连续出现} ,不能在文档结尾使用。

\sphinxAtStartPar
\sphinxstylestrong{示例如下:}

\begin{sphinxVerbatim}[commandchars=\\\{\}]
\PYG{n}{前后需要空行}

\PYG{p}{,}\PYG{p}{,}\PYG{p}{,}\PYG{p}{,}\PYG{p}{,}\PYG{p}{,}\PYG{p}{,}\PYG{p}{,}

\PYG{n}{使用标点符号}

\PYG{o}{.}\PYG{o}{.}\PYG{o}{.}\PYG{o}{.}\PYG{o}{.}\PYG{o}{.}\PYG{o}{.}\PYG{o}{.}\PYG{o}{.}\PYG{o}{.}\PYG{o}{.}\PYG{o}{.}\PYG{o}{.}

\PYG{n}{不能连续出现}

\PYG{o}{\PYGZhy{}}\PYG{o}{\PYGZhy{}}\PYG{o}{\PYGZhy{}}\PYG{o}{\PYGZhy{}}\PYG{o}{\PYGZhy{}}\PYG{o}{\PYGZhy{}}\PYG{o}{\PYGZhy{}}\PYG{o}{\PYGZhy{}}\PYG{o}{\PYGZhy{}}\PYG{o}{\PYGZhy{}}\PYG{o}{\PYGZhy{}}\PYG{o}{\PYGZhy{}}\PYG{o}{\PYGZhy{}}\PYG{o}{\PYGZhy{}}\PYG{o}{\PYGZhy{}}

\PYG{n}{不能在结尾使用}
\end{sphinxVerbatim}

\sphinxAtStartPar
\sphinxstylestrong{效果如下:}

\sphinxAtStartPar
前后需要空行


\bigskip\hrule\bigskip


\sphinxAtStartPar
使用标点符号


\bigskip\hrule\bigskip


\sphinxAtStartPar
不能连续出现


\bigskip\hrule\bigskip


\sphinxAtStartPar
不能在结尾使用


\section{脚注 Footnotes}
\label{\detokenize{reStructureText_syntax:footnotes}}
\sphinxAtStartPar
\sphinxhref{https://docutils.sourceforge.io/docs/ref/rst/restructuredtext.html\#footnotes}{脚注} 有三种形式,
手工序号(标记序号123之类)、自动序号(填入 \# 号会自动填充序号)、自动符号(填入 * 会自动生成符号)

\sphinxAtStartPar
手工序号可以和 \# 结合使用,会自动延续手工的序号。

\sphinxAtStartPar
\#表示的方法可以在后面加上一个名称,这个名称就会生成一个链接。
\begin{enumerate}
\sphinxsetlistlabels{\arabic}{enumi}{enumii}{}{.}%
\item {} 
\sphinxAtStartPar
手工标序(标记序号123之类)

\end{enumerate}

\sphinxAtStartPar
\sphinxstylestrong{示例如下:}

\begin{sphinxVerbatim}[commandchars=\\\{\}]
\PYG{n}{Footnote} \PYG{n}{references}\PYG{p}{,} \PYG{n}{like} \PYG{p}{[}\PYG{l+m+mi}{5}\PYG{p}{]}\PYG{n}{\PYGZus{}}\PYG{o}{.}
\PYG{n}{Note} \PYG{n}{that} \PYG{n}{footnotes} \PYG{n}{may} \PYG{n}{get} \PYG{p}{[}\PYG{l+m+mi}{3}\PYG{p}{]}\PYG{n}{\PYGZus{}}
\PYG{n}{rearranged}\PYG{p}{,} \PYG{n}{e}\PYG{o}{.}\PYG{n}{g}\PYG{o}{.}\PYG{p}{,} \PYG{n}{to} \PYG{n}{the} \PYG{n}{bottom} \PYG{n}{of}
\PYG{n}{the} \PYG{l+s+s2}{\PYGZdq{}}\PYG{l+s+s2}{page}\PYG{l+s+s2}{\PYGZdq{}}\PYG{o}{.}

\PYG{o}{.}\PYG{o}{.} \PYG{p}{[}\PYG{l+m+mi}{5}\PYG{p}{]} \PYG{n}{A} \PYG{n}{numerical} \PYG{n}{footnote}\PYG{o}{.} \PYG{n}{Note}
   \PYG{n}{there}\PYG{l+s+s1}{\PYGZsq{}}\PYG{l+s+s1}{s no colon after the ].}
\PYG{o}{.}\PYG{o}{.} \PYG{p}{[}\PYG{l+m+mi}{3}\PYG{p}{]} \PYG{n}{脚注3}
\end{sphinxVerbatim}

\sphinxAtStartPar
\sphinxstylestrong{效果如下:}

\sphinxAtStartPar
Footnote references, like %
\begin{footnote}[5]\sphinxAtStartFootnote
A numerical footnote. Note
there’s no colon after the \sphinxcode{\sphinxupquote{{]}}}.
%
\end{footnote}.
Note that footnotes may get %
\begin{footnote}[3]\sphinxAtStartFootnote
脚注3
%
\end{footnote}
rearranged, e.g., to the bottom of
the “page”.
\begin{enumerate}
\sphinxsetlistlabels{\arabic}{enumi}{enumii}{}{.}%
\setcounter{enumi}{1}
\item {} 
\sphinxAtStartPar
自动序号(填入 \# 号会自动填充序号)

\end{enumerate}

\sphinxAtStartPar
\sphinxstylestrong{示例如下:}

\begin{sphinxVerbatim}[commandchars=\\\{\}]
\PYG{n}{自动排序脚注}\PYG{p}{,} \PYG{n}{like} \PYG{n}{using} \PYG{p}{[}\PYG{c+c1}{\PYGZsh{}]\PYGZus{} and [\PYGZsh{}]\PYGZus{}.}
\PYG{o}{.}\PYG{o}{.} \PYG{p}{[}\PYG{c+c1}{\PYGZsh{}] This is the first one.}
\PYG{o}{.}\PYG{o}{.} \PYG{p}{[}\PYG{c+c1}{\PYGZsh{}] This is the second one.}
\end{sphinxVerbatim}

\sphinxAtStartPar
\sphinxstylestrong{效果如下:}

\sphinxAtStartPar
自动排序脚注, like using %
\begin{footnote}[2]\sphinxAtStartFootnote
This is the first one.
%
\end{footnote} and %
\begin{footnote}[4]\sphinxAtStartFootnote
This is the second one.
%
\end{footnote}.

\sphinxAtStartPar
可以添加别名,即可同时实现自动排序,又带有自定义名称,\sphinxstylestrong{这个功能相当于实现了文献引用功能} ;

\sphinxAtStartPar
\sphinxstylestrong{示例如下:}

\begin{sphinxVerbatim}[commandchars=\\\{\}]
\PYG{n}{They} \PYG{n}{may} \PYG{n}{be} \PYG{n}{assigned} \PYG{l+s+s1}{\PYGZsq{}}\PYG{l+s+s1}{autonumber}
\PYG{n}{labels}\PYG{l+s+s1}{\PYGZsq{}}\PYG{l+s+s1}{ \PYGZhy{} for instance,}
\PYG{p}{[}\PYG{c+c1}{\PYGZsh{}fourth]\PYGZus{} and [\PYGZsh{}third]\PYGZus{}.}

\PYG{o}{.}\PYG{o}{.} \PYG{p}{[}\PYG{c+c1}{\PYGZsh{}third] a.k.a. third\PYGZus{}}

\PYG{o}{.}\PYG{o}{.} \PYG{p}{[}\PYG{c+c1}{\PYGZsh{}fourth] a.k.a. fourth\PYGZus{}}
\end{sphinxVerbatim}

\sphinxAtStartPar
\sphinxstylestrong{效果如下:}

\sphinxAtStartPar
They may be assigned ‘autonumber
labels’ \sphinxhyphen{} for instance,
%
\begin{footnote}[7]\sphinxAtStartFootnote
a.k.a. {\hyperref[\detokenize{reStructureText_syntax:fourth}]{\sphinxcrossref{fourth}}}
%
\end{footnote} and %
\begin{footnote}[6]\sphinxAtStartFootnote
a.k.a. {\hyperref[\detokenize{reStructureText_syntax:third}]{\sphinxcrossref{third}}}
%
\end{footnote}.
\begin{enumerate}
\sphinxsetlistlabels{\arabic}{enumi}{enumii}{}{.}%
\setcounter{enumi}{2}
\item {} 
\sphinxAtStartPar
自动符号(填入 * 会自动生成符号)。

\end{enumerate}

\sphinxAtStartPar
自动填符号功能上和自动填序号是一样的,只是换了一种辨识符号。

\sphinxAtStartPar
\sphinxstylestrong{示例如下:}

\begin{sphinxVerbatim}[commandchars=\\\{\}]
\PYG{n}{自动脚注符号}\PYG{p}{,} \PYG{n}{like} \PYG{n}{this}\PYG{p}{:} \PYG{p}{[}\PYG{o}{*}\PYG{p}{]}\PYG{n}{\PYGZus{}} \PYG{p}{,}\PYG{p}{[}\PYG{o}{*}\PYG{p}{]}\PYG{n}{\PYGZus{}} \PYG{p}{,} \PYG{p}{[}\PYG{o}{*}\PYG{p}{]}\PYG{n}{\PYGZus{}} \PYG{o+ow}{and} \PYG{p}{[}\PYG{o}{*}\PYG{p}{]}\PYG{n}{\PYGZus{}}\PYG{o}{.}

\PYG{o}{.}\PYG{o}{.} \PYG{p}{[}\PYG{o}{*}\PYG{p}{]} \PYG{n}{This} \PYG{o+ow}{is} \PYG{n}{the} \PYG{n}{first} \PYG{n}{one}\PYG{o}{.}
\PYG{o}{.}\PYG{o}{.} \PYG{p}{[}\PYG{o}{*}\PYG{p}{]} \PYG{n}{This} \PYG{o+ow}{is} \PYG{n}{the} \PYG{n}{second} \PYG{n}{one}\PYG{o}{.}
\PYG{o}{.}\PYG{o}{.} \PYG{p}{[}\PYG{o}{*}\PYG{p}{]} \PYG{n}{This} \PYG{o+ow}{is} \PYG{n}{the} \PYG{n}{third} \PYG{n}{one}\PYG{o}{.}
\PYG{o}{.}\PYG{o}{.} \PYG{p}{[}\PYG{o}{*}\PYG{p}{]} \PYG{n}{This} \PYG{o+ow}{is} \PYG{n}{the} \PYG{n}{fourth} \PYG{n}{one}\PYG{o}{.}
\end{sphinxVerbatim}

\sphinxAtStartPar
\sphinxstylestrong{效果如下:}

\sphinxAtStartPar
自动脚注符号, like this: %
\begin{footnote}[*]\sphinxAtStartFootnote
This is the first one.
%
\end{footnote} ,{[}*{]}\_ , %
\begin{footnote}[†]\sphinxAtStartFootnote
This is the second one.
%
\end{footnote} and %
\begin{footnote}[‡]\sphinxAtStartFootnote
This is the third one.
%
\end{footnote}.


\section{引用Citations}
\label{\detokenize{reStructureText_syntax:citations}}
\sphinxAtStartPar
引用和脚注是一样的,只不过引用只能用文本而不能用数字。

\sphinxAtStartPar
\sphinxstylestrong{示例如下:}

\begin{sphinxVerbatim}[commandchars=\\\{\}]
引用参考的内容通常放在页面结尾处,比如 [One]\PYGZus{},Two\PYGZus{}

.. [One] 参考引用一
.. [Two] 参考引用二
\end{sphinxVerbatim}

\sphinxAtStartPar
\sphinxstylestrong{效果如下:}

\sphinxAtStartPar
引用参考的内容通常放在页面结尾处,比如 \sphinxcite{reStructureText_syntax:one},\sphinxcite{reStructureText_syntax:two}


\section{超链接Hyperlink Targets}
\label{\detokenize{reStructureText_syntax:hyperlink-targets}}
\sphinxAtStartPar
\sphinxhref{https://docutils.sourceforge.io/docs/ref/rst/restructuredtext.html\#hyperlink-targets}{超链接Hyperlink} 有三种:

\phantomsection\label{\detokenize{reStructureText_syntax:id36}}\begin{itemize}
\item {} 
\sphinxAtStartPar
带别名的超链接 ,语法为 \sphinxcode{\sphinxupquote{.. \_hyperlink\sphinxhyphen{}name: link\sphinxhyphen{}address}} ;由 \sphinxcode{\sphinxupquote{..}},空格,短下划线”\_”,别名,冒号,空格和链接地址构成。
在原文引用处书写语法为 \sphinxcode{\sphinxupquote{hyperlink\sphinxhyphen{}name\_}} (特别注意原文中”\_”在别名后,而在指示链接出,”\_”在别名前)。

\item {} 
\sphinxAtStartPar
匿名anonymous的超链接,即不带别名的超链接,语法为 \sphinxcode{\sphinxupquote{.. \_\_: link\sphinxhyphen{}address}} ; 由 \sphinxcode{\sphinxupquote{..}},空格,两个短下划线”\_\_”,冒号,空格和链接地址构成。

\item {} 
\sphinxAtStartPar
匿名的超链接,另一种语法形式,语法为 \sphinxcode{\sphinxupquote{\_\_ link\sphinxhyphen{}address}}  。

\end{itemize}


\subsection{外部链接 External Hyperlink Targets}
\label{\detokenize{reStructureText_syntax:external-hyperlink-targets}}
\sphinxAtStartPar
外部链接有两种方式,需要引用的话,使用上述带别名的超链接的语法形式,即

\sphinxAtStartPar
\sphinxstylestrong{示例如下:}

\begin{sphinxVerbatim}[commandchars=\\\{\}]
这是我的 reStructureText\PYGZus{} 实践笔记。

.. \PYGZus{}reStructureText: https://sphinx\PYGZhy{}practise.readthedocs.io/zh\PYGZus{}CN/latest/index.html
\end{sphinxVerbatim}

\sphinxAtStartPar
\sphinxstylestrong{效果如下:}

\sphinxAtStartPar
这是我的 \sphinxhref{https://sphinx-practise.readthedocs.io/zh\_CN/latest/index.html}{reStructureText} 实践笔记。

\sphinxAtStartPar
另一种是直接在名称后附加地址, 语法为 `别名 <链接>`\_

\sphinxAtStartPar
\sphinxstylestrong{示例如下:}

\begin{sphinxVerbatim}[commandchars=\\\{\}]
这是我的 `reStructureText \PYGZlt{}https://sphinx\PYGZhy{}practise.readthedocs.io/zh\PYGZus{}CN/latest/index.html\PYGZgt{}`\PYGZus{} 实践笔记。
\end{sphinxVerbatim}

\sphinxAtStartPar
\sphinxstylestrong{效果如下:}

\sphinxAtStartPar
这是我的 \sphinxhref{https://sphinx-practise.readthedocs.io/zh\_CN/latest/index.html}{reStructureText} 实践笔记。


\subsection{锚点链接 Internal Hyperlink Targets}
\label{\detokenize{reStructureText_syntax:internal-hyperlink-targets}}
\sphinxAtStartPar
内部超链接,即锚点。

\sphinxAtStartPar
锚点的语法即外部超链接中 {\hyperref[\detokenize{reStructureText_syntax:id36}]{\sphinxcrossref{带别名的超链接}}} 去除外部链接,其他语法一致。


\subsection{间接链接 Indirect Hyperlink Targets}
\label{\detokenize{reStructureText_syntax:indirect-hyperlink-targets}}
\sphinxAtStartPar
间接超链接是基于匿名链接的基础上的,就是将匿名链接地址换成了外部引用名。

\sphinxAtStartPar
\sphinxstylestrong{示例如下:}

\begin{sphinxVerbatim}[commandchars=\\\{\}]
Python\PYGZus{} is `my favourite
programming language`\PYGZus{}\PYGZus{}.

.. \PYGZus{}Python: http://www.python.org/

\PYGZus{}\PYGZus{} Python\PYGZus{}
\end{sphinxVerbatim}

\sphinxAtStartPar
\sphinxstylestrong{效果如下:}

\sphinxAtStartPar
\sphinxhref{http://www.python.org/}{Python} is \sphinxhref{http://www.python.org/}{my favourite
programming language}.

\sphinxAtStartPar
其中 python\_ 就是一个正常的外部链接,而后面那句话是一个匿名链接,
对这个匿名链接使用间接链接方式链接到 Python这个外部链接的链接地址上去。


\subsection{Implicit Hyperlink Targets}
\label{\detokenize{reStructureText_syntax:implicit-hyperlink-targets}}
\sphinxAtStartPar
隐式超链接

\sphinxAtStartPar
小节标题、脚注和引用参考会自动生成超链接地址,使用小节标题、脚注或引用参考名称作为超链接名称就可以生成隐式链接。

\sphinxAtStartPar
本质上它们的写法都是和 {\hyperref[\detokenize{reStructureText_syntax:external-hyperlink-targets}]{\sphinxcrossref{外部链接 External Hyperlink Targets}}} 相一致的, 只是做了一些微小改动,以做出区别。

\sphinxAtStartPar
例如链接到 {\hyperref[\detokenize{reStructureText_syntax:hyperlink-targets}]{\sphinxcrossref{超链接Hyperlink Targets}}} 这个章节目录去

\begin{sphinxVerbatim}[commandchars=\\\{\}]
`超链接Hyperlink Targets`\PYGZus{}
\end{sphinxVerbatim}


\section{扩展指令 Directives}
\label{\detokenize{reStructureText_syntax:directives}}
\sphinxAtStartPar
指令(Directives)是reStructureText的扩展机制。
可以在不增加新语法的情况下,增加新的结构性支持(a way of adding support for new constructs)。

\sphinxAtStartPar
指令由三部分组成

\begin{sphinxVerbatim}[commandchars=\\\{\}]
\PYG{o}{.}\PYG{o}{.} \PYG{n}{directive}\PYG{o}{\PYGZhy{}}\PYG{n+nb}{type}\PYG{p}{:}\PYG{p}{:} \PYG{n}{directive}\PYG{o}{\PYGZhy{}}\PYG{n}{block}
\end{sphinxVerbatim}

\sphinxAtStartPar
其中指令类型(directive\sphinxhyphen{}type)指明指令的类型,指令内容体又由三部分组成
\begin{itemize}
\item {} 
\sphinxAtStartPar
指令作用对象Directive arguments:指明该指令针对哪个对象作用

\item {} 
\sphinxAtStartPar
指令选项参数Directive options:该指令的可选参数(可选),是一个参数列表

\item {} 
\sphinxAtStartPar
指令内容说明Directive content:说明文档(可选)

\end{itemize}

\sphinxAtStartPar
比如插入一个图片

\begin{sphinxVerbatim}[commandchars=\\\{\}]
\PYG{o}{.}\PYG{o}{.} \PYG{n}{figure}\PYG{p}{:}\PYG{p}{:} \PYG{n}{图片名}\PYG{o}{.}\PYG{n}{png}  \PYG{c+c1}{\PYGZsh{} 这里是指令作用对象}
   \PYG{p}{:}\PYG{n}{scale}\PYG{p}{:} \PYG{l+m+mi}{50}     \PYG{c+c1}{\PYGZsh{} 这里是指令参数}
   \PYG{p}{:}\PYG{n}{width}\PYG{p}{:} \PYG{l+m+mi}{100}

   \PYG{n}{这是一个图片}   \PYG{c+c1}{\PYGZsh{} 这里是说明}
\end{sphinxVerbatim}

\sphinxAtStartPar
已在 \sphinxhref{https://docutils.sourceforge.io/docs/ref/rst/directives.html}{reference reStructuredText parser} 中实现的指令。


\subsection{警告信息Admonitions}
\label{\detokenize{reStructureText_syntax:admonitions}}

\subsubsection{特定的警告信息}
\label{\detokenize{reStructureText_syntax:id40}}
\sphinxAtStartPar
格式为

\begin{sphinxVerbatim}[commandchars=\\\{\}]
\PYG{o}{.}\PYG{o}{.} \PYG{n}{admonition}\PYG{p}{:}\PYG{p}{:} \PYG{n}{admonition}\PYG{o}{\PYGZhy{}}\PYG{n}{title}\PYG{p}{(}\PYG{n}{可空}\PYG{p}{)}
   \PYG{p}{:}\PYG{n}{class}\PYG{p}{:} \PYG{n}{class}\PYG{o}{\PYGZhy{}}\PYG{n}{name}\PYG{p}{(}\PYG{n}{可选}\PYG{p}{)}
       \PYG{p}{:}\PYG{n}{name}\PYG{p}{:} \PYG{n}{name}\PYG{p}{(}\PYG{n}{可选}\PYG{p}{)}
   \PYG{n}{admonition}\PYG{o}{\PYGZhy{}}\PYG{n}{content说明信息}
\end{sphinxVerbatim}

\sphinxAtStartPar
admonition\sphinxhyphen{}title和admonition\sphinxhyphen{}content显示效果是一体的,
但是admonition\sphinxhyphen{}title(可空)会在html中单独存在一个title标签中。

\sphinxAtStartPar
支持如下特定警告信息
\begin{itemize}
\item {} 
\sphinxAtStartPar
attention

\item {} 
\sphinxAtStartPar
caution

\item {} 
\sphinxAtStartPar
danger

\item {} 
\sphinxAtStartPar
error

\item {} 
\sphinxAtStartPar
hint

\item {} 
\sphinxAtStartPar
important

\item {} 
\sphinxAtStartPar
note

\item {} 
\sphinxAtStartPar
tip

\item {} 
\sphinxAtStartPar
warning

\end{itemize}

\sphinxAtStartPar
\sphinxstylestrong{示例如下:}

\begin{sphinxVerbatim}[commandchars=\\\{\}]
\PYG{o}{.}\PYG{o}{.} \PYG{n}{attention}\PYG{p}{:}\PYG{p}{:} \PYG{n}{This} \PYG{o+ow}{is} \PYG{n}{a} \PYG{n}{attention} \PYG{n}{admonition}\PYG{o}{.}
   \PYG{n}{second} \PYG{n}{attention} \PYG{n}{paragraph}\PYG{o}{.}

\PYG{o}{.}\PYG{o}{.} \PYG{n}{caution}\PYG{p}{:}\PYG{p}{:} \PYG{n}{This} \PYG{o+ow}{is} \PYG{n}{a} \PYG{n}{caution} \PYG{n}{admonition}\PYG{o}{.}
   \PYG{n}{second} \PYG{n}{caution} \PYG{n}{paragraph}\PYG{o}{.}

\PYG{o}{.}\PYG{o}{.} \PYG{n}{danger}\PYG{p}{:}\PYG{p}{:} \PYG{n}{This} \PYG{o+ow}{is} \PYG{n}{a} \PYG{n}{danger} \PYG{n}{admonition}\PYG{o}{.}
   \PYG{n}{second} \PYG{n}{danger} \PYG{n}{paragraph}\PYG{o}{.}

\PYG{o}{.}\PYG{o}{.} \PYG{n}{error}\PYG{p}{:}\PYG{p}{:} \PYG{n}{This} \PYG{o+ow}{is} \PYG{n}{a} \PYG{n}{error} \PYG{n}{admonition}\PYG{o}{.}
   \PYG{n}{second} \PYG{n}{error} \PYG{n}{paragraph}\PYG{o}{.}

\PYG{o}{.}\PYG{o}{.} \PYG{n}{hint}\PYG{p}{:}\PYG{p}{:} \PYG{n}{This} \PYG{o+ow}{is} \PYG{n}{a} \PYG{n}{hint} \PYG{n}{admonition}\PYG{o}{.}
   \PYG{n}{second} \PYG{n}{hint} \PYG{n}{paragraph}\PYG{o}{.}

\PYG{o}{.}\PYG{o}{.} \PYG{n}{important}\PYG{p}{:}\PYG{p}{:} \PYG{n}{This} \PYG{o+ow}{is} \PYG{n}{a} \PYG{n}{important} \PYG{n}{admonition}\PYG{o}{.}
   \PYG{n}{second} \PYG{n}{important} \PYG{n}{paragraph}\PYG{o}{.}

\PYG{o}{.}\PYG{o}{.} \PYG{n}{note}\PYG{p}{:}\PYG{p}{:} \PYG{n}{This} \PYG{o+ow}{is} \PYG{n}{a} \PYG{n}{note} \PYG{n}{admonition}\PYG{o}{.}
   \PYG{n}{This} \PYG{o+ow}{is} \PYG{n}{the} \PYG{n}{second} \PYG{n}{line} \PYG{n}{of} \PYG{n}{the} \PYG{n}{first} \PYG{n}{paragraph}\PYG{o}{.}

   \PYG{o}{\PYGZhy{}} \PYG{n}{The} \PYG{n}{note} \PYG{n}{contains} \PYG{n+nb}{all} \PYG{n}{indented} \PYG{n}{body} \PYG{n}{elements}
     \PYG{n}{following}\PYG{o}{.}
   \PYG{o}{\PYGZhy{}} \PYG{n}{It} \PYG{n}{includes} \PYG{n}{this} \PYG{n}{bullet} \PYG{n+nb}{list}\PYG{o}{.}

\PYG{o}{.}\PYG{o}{.} \PYG{n}{tip}\PYG{p}{:}\PYG{p}{:} \PYG{n}{This} \PYG{o+ow}{is} \PYG{n}{a} \PYG{n}{tip} \PYG{n}{admonition}\PYG{o}{.}
   \PYG{n}{second} \PYG{n}{tip} \PYG{n}{paragraph}\PYG{o}{.}

\PYG{o}{.}\PYG{o}{.} \PYG{n}{warning}\PYG{p}{:}\PYG{p}{:} \PYG{n}{This} \PYG{o+ow}{is} \PYG{n}{a} \PYG{n}{warning} \PYG{n}{admonition}\PYG{o}{.}
   \PYG{n}{second} \PYG{n}{warning} \PYG{n}{paragraph}\PYG{o}{.}
\end{sphinxVerbatim}

\sphinxAtStartPar
\sphinxstylestrong{效果如下:}

\begin{sphinxadmonition}{attention}{注意:}
\sphinxAtStartPar
This is a attention admonition.
second attention paragraph.
\end{sphinxadmonition}

\begin{sphinxadmonition}{caution}{小心:}
\sphinxAtStartPar
This is a caution admonition.
second caution paragraph.
\end{sphinxadmonition}

\begin{sphinxadmonition}{danger}{危险:}
\sphinxAtStartPar
This is a danger admonition.
second danger paragraph.
\end{sphinxadmonition}

\begin{sphinxadmonition}{error}{错误:}
\sphinxAtStartPar
This is a error admonition.
second error paragraph.
\end{sphinxadmonition}

\begin{sphinxadmonition}{hint}{提示:}
\sphinxAtStartPar
This is a hint admonition.
second hint paragraph.
\end{sphinxadmonition}

\begin{sphinxadmonition}{important}{重要:}
\sphinxAtStartPar
This is a important admonition.
second important paragraph.
\end{sphinxadmonition}

\begin{sphinxadmonition}{note}{备注:}
\sphinxAtStartPar
This is a note admonition.
This is the second line of the first paragraph.
\begin{itemize}
\item {} 
\sphinxAtStartPar
The note contains all indented body elements
following.

\item {} 
\sphinxAtStartPar
It includes this bullet list.

\end{itemize}
\end{sphinxadmonition}

\begin{sphinxadmonition}{tip}{小技巧:}
\sphinxAtStartPar
This is a tip admonition.
second tip paragraph.
\end{sphinxadmonition}

\begin{sphinxadmonition}{warning}{警告:}
\sphinxAtStartPar
This is a warning admonition.
second warning paragraph.
\end{sphinxadmonition}


\subsubsection{通用警告信息Generic Admonition}
\label{\detokenize{reStructureText_syntax:generic-admonition}}
\sphinxAtStartPar
通用警告信息即不指定为特定的警告类别,使用admonition指代警告。
与特定警告不同的是,特定警告的admonition\sphinxhyphen{}title在通用警告中为admonition\sphinxhyphen{}name,
这是我们自定义的警告名,用于和特定警告(danger,hint,important等)提供同等标识。

\sphinxAtStartPar
\sphinxstylestrong{示例如下:}

\begin{sphinxVerbatim}[commandchars=\\\{\}]
\PYG{o}{.}\PYG{o}{.} \PYG{n}{admonition}\PYG{p}{:}\PYG{p}{:} \PYG{n}{And}\PYG{p}{,} \PYG{n}{by} \PYG{n}{the} \PYG{n}{way}\PYG{o}{.}\PYG{o}{.}\PYG{o}{.}

   \PYG{n}{You} \PYG{n}{can} \PYG{n}{make} \PYG{n}{up} \PYG{n}{your} \PYG{n}{own} \PYG{n}{admonition} \PYG{n}{too}\PYG{o}{.}
\end{sphinxVerbatim}

\sphinxAtStartPar
\sphinxstylestrong{结果如下:}

\begin{sphinxadmonition}{note}{And, by the way…}

\sphinxAtStartPar
You can make up your own admonition too.
\end{sphinxadmonition}


\subsection{图片Images}
\label{\detokenize{reStructureText_syntax:images}}
\sphinxAtStartPar
使用image

\begin{sphinxVerbatim}[commandchars=\\\{\}]
.. image:: picture.jpeg
   :class: class\PYGZhy{}name
       :name: name
   :height: 100 px(长度)
   :width: 200 px (长度或者百分比)
   :scale: 50 \PYGZpc{} (百分比,百分号可省略)
   :alt: alternate text
   :align: right
       :target: https://www.baidu.com
\end{sphinxVerbatim}

\sphinxAtStartPar
align可选top,middle,bottom,left,center,right

\sphinxAtStartPar
target使得图片可点击跳转。

\sphinxAtStartPar
scale表示等比例伸缩(放大或者缩小)

\begin{sphinxadmonition}{important}{重要:}
\sphinxAtStartPar
scale需要和width或者height(或者2者)一起使用。
\end{sphinxadmonition}

\sphinxAtStartPar
使用figure

\begin{sphinxVerbatim}[commandchars=\\\{\}]
.. figure:: picture.png
   :figwidth: 200 px (长度或者百分比)
   :scale: 50 \PYGZpc{}
       :align: center
       :figclass: figure\PYGZhy{}class
   :alt: map to buried treasure

    +\PYGZhy{}\PYGZhy{}\PYGZhy{}\PYGZhy{}\PYGZhy{}\PYGZhy{}\PYGZhy{}\PYGZhy{}\PYGZhy{}\PYGZhy{}\PYGZhy{}\PYGZhy{}\PYGZhy{}\PYGZhy{}\PYGZhy{}\PYGZhy{}\PYGZhy{}\PYGZhy{}\PYGZhy{}\PYGZhy{}\PYGZhy{}\PYGZhy{}\PYGZhy{}\PYGZhy{}\PYGZhy{}\PYGZhy{}\PYGZhy{}+
    |        figure             |
    |                           |
    |\PYGZlt{}\PYGZhy{}\PYGZhy{}\PYGZhy{}\PYGZhy{}\PYGZhy{}\PYGZhy{} figwidth \PYGZhy{}\PYGZhy{}\PYGZhy{}\PYGZhy{}\PYGZhy{}\PYGZhy{}\PYGZhy{}\PYGZhy{}\PYGZhy{}\PYGZgt{}|
    |                           |
    |  +\PYGZhy{}\PYGZhy{}\PYGZhy{}\PYGZhy{}\PYGZhy{}\PYGZhy{}\PYGZhy{}\PYGZhy{}\PYGZhy{}\PYGZhy{}\PYGZhy{}\PYGZhy{}\PYGZhy{}\PYGZhy{}\PYGZhy{}\PYGZhy{}\PYGZhy{}\PYGZhy{}\PYGZhy{}\PYGZhy{}\PYGZhy{}+  |
    |  |     image           |  |
    |  |                     |  |
    |  |\PYGZlt{}\PYGZhy{}\PYGZhy{}\PYGZhy{} width \PYGZhy{}\PYGZhy{}\PYGZhy{}\PYGZhy{}\PYGZhy{}\PYGZhy{}\PYGZhy{}\PYGZhy{}\PYGZhy{}\PYGZgt{}|  |
    |  +\PYGZhy{}\PYGZhy{}\PYGZhy{}\PYGZhy{}\PYGZhy{}\PYGZhy{}\PYGZhy{}\PYGZhy{}\PYGZhy{}\PYGZhy{}\PYGZhy{}\PYGZhy{}\PYGZhy{}\PYGZhy{}\PYGZhy{}\PYGZhy{}\PYGZhy{}\PYGZhy{}\PYGZhy{}\PYGZhy{}\PYGZhy{}+  |
    |                           |
    |The figure\PYGZsq{}s caption should|
    |wrap at this width.        |
    +\PYGZhy{}\PYGZhy{}\PYGZhy{}\PYGZhy{}\PYGZhy{}\PYGZhy{}\PYGZhy{}\PYGZhy{}\PYGZhy{}\PYGZhy{}\PYGZhy{}\PYGZhy{}\PYGZhy{}\PYGZhy{}\PYGZhy{}\PYGZhy{}\PYGZhy{}\PYGZhy{}\PYGZhy{}\PYGZhy{}\PYGZhy{}\PYGZhy{}\PYGZhy{}\PYGZhy{}\PYGZhy{}\PYGZhy{}\PYGZhy{}+
\end{sphinxVerbatim}

\sphinxAtStartPar
figure相当于一个画布(类似于html中的一个div或者一个canvas),
它对处于其内的内容进行样式统一管理。相比image可以包含除图片外的更多内容。

\sphinxAtStartPar
figure支持image的所有指令选项参数,除了align在figure中指示整个画布的对齐方式。
且它只能选择为left,center,right。

\begin{sphinxadmonition}{important}{重要:}
\sphinxAtStartPar
和image一致,要使得scale(这里是对整个画布作用)起作用需要和figwidth一起使用
\end{sphinxadmonition}


\subsection{页面元素Body Elements}
\label{\detokenize{reStructureText_syntax:body-elements}}

\subsection{表格Tables}
\label{\detokenize{reStructureText_syntax:id41}}

\subsection{文档Documents}
\label{\detokenize{reStructureText_syntax:documents}}

\subsection{References}
\label{\detokenize{reStructureText_syntax:references}}

\subsection{HTML\sphinxhyphen{}Specific}
\label{\detokenize{reStructureText_syntax:html-specific}}

\subsection{Substitution Definitions}
\label{\detokenize{reStructureText_syntax:substitution-definitions}}

\subsection{其他}
\label{\detokenize{reStructureText_syntax:id42}}

\subsection{通用指令选项参数}
\label{\detokenize{reStructureText_syntax:id43}}

\begin{savenotes}\sphinxattablestart
\sphinxthistablewithglobalstyle
\centering
\begin{tabulary}{\linewidth}[t]{TT}
\sphinxtoprule
\sphinxtableatstartofbodyhook
\sphinxAtStartPar
:class:
&
\sphinxAtStartPar
得到
\\
\sphinxhline
\sphinxAtStartPar
:name:
&
\sphinxAtStartPar
为指令设置名称(可用于简化别名链接)
\\
\sphinxbottomrule
\end{tabulary}
\sphinxtableafterendhook\par
\sphinxattableend\end{savenotes}

\begin{sphinxVerbatim}[commandchars=\\\{\}]
\PYG{o}{.}\PYG{o}{.} \PYG{n}{image}\PYG{p}{:}\PYG{p}{:} \PYG{n}{build}\PYG{o}{.}\PYG{n}{png}
   \PYG{p}{:}\PYG{n}{name}\PYG{p}{:} \PYG{n}{my} \PYG{n}{pic}
\PYG{n}{与下列方式等价}
\PYG{o}{.}\PYG{o}{.} \PYG{n}{\PYGZus{}my} \PYG{n}{pic}

\PYG{o}{.}\PYG{o}{.} \PYG{n}{image}\PYG{p}{:}\PYG{p}{:} \PYG{n}{build}\PYG{o}{.}\PYG{n}{png}
\end{sphinxVerbatim}


\section{Substitution References and Definitions}
\label{\detokenize{reStructureText_syntax:substitution-references-and-definitions}}

\section{Comments}
\label{\detokenize{reStructureText_syntax:comments}}
\sphinxAtStartPar
非上述语法,则都作为Comments处理。

\sphinxstepscope


\chapter{Reference}
\label{\detokenize{reference:reference}}\label{\detokenize{reference::doc}}

\section{编辑工具}
\label{\detokenize{reference:id1}}
\sphinxAtStartPar
\sphinxurl{https://github.com/steenhulthin/reStructuredText\_NPP} notepad++插件

\sphinxAtStartPar
\sphinxurl{http://rst.ninjs.org/}  在线


\section{参考内容}
\label{\detokenize{reference:id2}}

\subsection{sphinx}
\label{\detokenize{reference:sphinx}}
\sphinxAtStartPar
\sphinxurl{https://www.sphinx.org.cn/}

\sphinxAtStartPar
\sphinxurl{https://www.sphinx-doc.org/en/master/}

\sphinxAtStartPar
\sphinxurl{https://zh-sphinx-doc.readthedocs.io/en/latest/}


\subsection{reStructureText}
\label{\detokenize{reference:restructuretext}}
\sphinxAtStartPar
\sphinxurl{https://docutils.sourceforge.io/docs/user/rst/quickref.html}

\sphinxAtStartPar
\sphinxurl{https://blog.csdn.net/liuskyter/article/details/86570790}

\sphinxstepscope


\chapter{Pygments}
\label{\detokenize{pygments_intro:pygments}}\label{\detokenize{pygments_intro::doc}}

\chapter{Indices and tables}
\label{\detokenize{index:indices-and-tables}}\begin{itemize}
\item {} 
\sphinxAtStartPar
\DUrole{xref,std,std-ref}{genindex}

\item {} 
\sphinxAtStartPar
\DUrole{xref,std,std-ref}{modindex}

\item {} 
\sphinxAtStartPar
\DUrole{xref,std,std-ref}{search}

\end{itemize}

\begin{sphinxthebibliography}{One}
\bibitem[One]{reStructureText_syntax:one}
\sphinxAtStartPar
参考引用一
\bibitem[Two]{reStructureText_syntax:two}
\sphinxAtStartPar
参考引用二
\end{sphinxthebibliography}



\renewcommand{\indexname}{索引}
\printindex
\end{document}